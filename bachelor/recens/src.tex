\documentclass[12pt]{article}
\usepackage[T2A]{fontenc}
\usepackage[utf8]{inputenc}
\usepackage[ukrainian]{babel}
\usepackage{pscyr}
\usepackage[top=2cm,left=3cm,bottom=2cm,right=2cm]{geometry}

\begin{document}
\renewcommand{\baselinestretch}{1.5}\thispagestyle{empty}

\begin{center}
  \bfseries
  {\Large РЕЦЕНЗІЯ}

  \large
  на дипломну роботу

  освітньо"=кваліфікаційного рівня <<Бакалавр>>
\end{center}
\renewcommand{\baselinestretch}{1.2}\thispagestyle{empty}
\large
виконану на тему:\\
<<ППП деконволюції абстрактного сигналу як метод відновлення зображень>>
студентом Погодою Михайлом Володимировичем.

Кваліфікаційна робота студента Погоди~М.\,В. повністю відповідає затвердженій
темі та завданню.
Ця робота присвячена вивченю процесів, що стоять за спотворенням зображеннь
(змазів або розфокусувань).

У рамках дипломної роботи було розглянута математична модель, що досить добре
описує багато видів спотворень (не лише двовимірних зображень).
Були проаналізовані найпоширеніші методи деконволюції.
Оскільки не всі методи дозволяють отримати гарні результаті за реальних умов,
було обрано оптимальний метод, який і був використаний при розробці пакету
прикладних програм.
Написання програми було виконано з використанням ідіоми
об’єктно"=орієнтованого програмування.
В програмній реалізації використована високоефективна бібліотека, що дозволяє
отримувати результат найшвидшим образом.

Розрахунково"=пояснювальну записку до дипломної роботи виконано з дотриманням
необхідних вимог.
Матеріал викладено чітко та цілісно, зручно структуровано у розділи й секції.

У даній роботі суттєвих недоліків не виявлено.

Вважаю, що дана дипломна робота відповідає усім вимога щодо робіт
освітньо"=кваліфікаційного рівня <<Бакалавр>> із напряму підготовки
<<Прикладна математика>> та заслуговує на оцінку <<відмінно>>, а її автор,
Погода~М.\,В., --- присвоєння кваліфікації <<Фахівець (Прикладна
математика)>>.
\vfill
Рецензент\\
доцент кафедри СПіСПС\\\indent НТУУ ,,КПІ'', к.т.н.\hfill\hfill\underline{\hspace{3cm}}\hfill
А.\,В.~Петрашенко

\end{document}
