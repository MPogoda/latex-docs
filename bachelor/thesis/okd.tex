\documentclass{diploma}
\usepackage{bm}
\usepackage{pscyr}
\usepackage{amsmath}

\begin{document}
\renewcommand{\baselinestretch}{1.5}\setfontsize{14pt}
\thispagestyle{empty}
\begin{center}
\textbf{АНОТАЦІЯ}
\end{center}

Дана дипломна робота присвячена розробленню математичної моделі кластеризованої лінгвістичної бази даних і знань для розпізнавання мовленнєвого сигналу.

В роботі виконаний порівняльний аналіз методів розпізнавання звуку, зокрема застосування прихованих марківських моделей та нейронних мереж, та методів побудови лінгвістичної моделі, а саме --- проста побудова лінгвістичної моделі та кластеризована. Запропоновано математичну модель кластеризації, що не включає до одного класу фонетично схожі слова, та розроблено відповідний модуль. 

Також в роботі проведено тестування розробленого модуля кластеризації лінгвістичної бази даних.

\newpage
\thispagestyle{empty}
\begin{center}
\textbf{АНОТАЦИЯ}
\end{center}

Данная дипломная работа посвящена разработке математической модели кластеризированной лингвистической базы данных и знаний для распознавание речевого сигнала.

В работе выполнен сравнительный анализ методов распознавания звука, в частности применение скрытых марковских моделей и нейронных сетей, и методов построения лингвистической модели, а именно --- построение простой лингвистической модели и кластеризированной. Предложена математическая модель кластеризации, которая не группирует в один кластер фонетически близкие слова, и разработано соответствующий модуль.

Также в работе проведено тестирование разработанного модуля кластеризации лингвистической базы данных.
	
\newpage
\thispagestyle{empty}
\begin{center}
\textbf{ABSTRACT}
\end{center}

This diploma thesis deals with the development of the mathematical model for class-based language model for speech signal recognition.
	
The comparative analysis of mathematics methods of speech signal recognition, in particular, the application of hidden markov models and neural networks, and methods of constructing language models --- such as easy construction of language models and class-based language model, is fulfilled and discussed. Mathematical model of making classes for language model that does not combine phonetically similar words to one class, is proposed.

Testing the developed module for class-based language database was also fulfilled.	
\newpage

\maketitlepage
  	
\shortings
	
ПЗ --- програмне забезпечення;

CMED --- computing minimal edition distance;

ПММ ---  приховані марківські моделі;

к-ть --- кількість;

Інформаційна ентропія --- міра невизначеності випадкової величини.



\intro Системи розпізнавання злитого мовленнєвого сигналу набули широкого застосування в наш час: голосові команди, пошук аудіофайлів за їх змістом, запис озвученого тексту, ідентифікація диктора та інше. Таким чином, будучи посередником взаємодії комп'ютера та людини, дані системи поступово витісняють інші види вводу інформації.

Дана тема є і буде актуальною, так як системи розпізнавання звуку вводяться в нових галузях, тим самим полегшують управління, створюють автоматизацію. 

В наш час такі системи існують і повноцінно функціонують лише для англійської мови. Для слов'янських мов загальноприйняті алгоритми  та методи розпізнавання злитого мовленнєвого сигналу не дають точного результату, в силу того, що в слов'янських мовах у 8-10 разів більше словоформ, ніж в англійській та достатньо вільний порядок слів у реченні. Таким чином сильно зростає робочий словник та зменшується точність прогнозування в лінгвістичній моделі.

Аналіз публікацій та відкритих для користування програм провідних компаній та наукових центрів з розпізнавання мовленнєвого сигналу показує, що найпоширенішого застосування для розпізнавання мовлення набула схема неявних (прихованих) марківських моделей. Тому взявши дану схему за основу будемо намагатись вдосконалити розпізнавання сигналу українською мовою.


\newpage





\chapter{ПОСТАНОВКА ЗАДАЧІ}

В даній роботі ставиться задача розробки кластеризованої лінгвістичної бази даних для розпізнавання мовленнєвого сигналу, яка буде враховувати фонетичичну схожість слів й не допускатиме їх групування до одного кластеру. Дана задача включає наступні завдання:

\begin{enumerate}
\item аналіз існуючих методів побудови лінгвістичної моделі;
\item розробка алгоритму побудови кластеризованої лінгвістичної бази, з урахуванням фонетичної схожості слів;
\item розробка програмного модуля для програми кластеризації \emph{clastlm};
\item проведення порівняльного аналізу та оцінка роботи ПЗ з вдосконаленою лінгвістичною базою;
\end{enumerate}

Функціонал програми не повинен обмежуват користувача --- користувачу потрібно надати можливість налаштування параметрів запуску програми та керування процесом кластеризації.

Всі можливі параметри кластеризації в \emph{clastlm} повинні працювати і для роботи з підключеним модулем:

\begin{itemizer} 
\item можливість задати наступні параметри: максимальна кількість ітерацій та кластерів; 
\item можливість збільшення к-ть кластерів починаючи з певної ітерації; 
\item можливість збільшення к-ть кластерів починаючи з певним інкрементом;
\item запис кластерів до файлу;
\item зчитування кластерів з файлу;
\item зчитування кластерів з log файлу;
\end{itemizer}

Користувачу надається можливість управління процесом роботи програми, а саме можливість зупинити кластеризацію в певний момент часу та продовжити з того ж місця в інший час, при цьому, за бажанням, змінивши параметри кластеризації.





\chapter{ІНФОРМАЦІЙНЕ ЗАБЕЗПЕЧЕННЯ}

\Section{Аналіз та огляд існуючих методів}

\Subsection{Аналіз методів розпізнавання мовленнєвого сигналу.}

\textbf{Застосування прихованих марківських моделей}

Прихованою марківською моделлю (ПММ) називається модель, що складається з $M$ станів ($N-gram$ мадель), в кожному з яких деяка система може приймати одне з $N$ значень якого-небудь параметру.

Ймовірність переодів між станами задається матрицею ймовірностей $A=\{a_{ij}\}$, де $a_{ij}$ - ймовірніст переходу з $i$-го в $j$-ий стан. 

Ймовірність випадіння кожного з $M$ значень параметрів в кожному з $N$ станів задається вектором $B=\{b_j(k)\}$, де $b_j(k)$ - ймовірнісь випадіння $k$-го значення параметру в $j$-м стані.

Ймовірність настання початкового стану задається вектором $\pi = {\pi_i}$, де $\pi_i$ - ймовірність того, що в початковий момент система опинеться в $i$-му стані.

Таким чином, прихованою марківською моделлю називают трійку $\lambda = \{A,B,\pi\}$. Використання прихованих марковських моделей для розпізнавання мови базується на двох засадах:

\begin{enumerate}
\item Звуковий сигнал може бути розбитий на фрагменти, що відповідають станам в ПММ, параметри звукового сигналу в межах кожного фрагменту вважаються постійними.
\item Ймовірність кожного фрагменту залежить тільки від поточного стану системи і не залежить від попередніх станів.
\end{enumerate}

Модель називається ,,прихованою'', так як нас, як правило, не цікавить конкретна послідовність станів, в якій перебуває система. Ми або подаємо на вхід системи послідовність типу $O = \{o_1,o_2, \dots o_n\}$, де кожне $o_i$ - значення параметру (одне з $M$), яке приймається в $i$-тий момент часу, а на виході очікуємо модель $\lambda = \{A,B,\pi\}$ з максимальною ймовірністю генеруючу таку послідовність, або ж навпаки --- подаємо на вхід параметри моделі і генеруємо нею породжену послідовність. І в тому і в іншому випадку система виступає як ,,'чорний ящик'', в якому приховані поточні стани системи, а пов'язана знею модель заслуговує на назву ,,прихована''.


Формування генеративної моделі починається з наступного: отриманий звуковий сигнал в результаті препроцесингу перетворюється в послідовність акустичних векторів фіксованого виміру $\boldsymbol Y_{i:T} = y_1,\dots,y_T$. Далі декодер намагається встановити послідовність слів $\boldsymbol\omega_{i:L} = \omega_1,\dots,\omega_L$, які найімовірніше утворили $Y$, тобто декодер шукає

\begin{equation}
\label{eq:1}
\boldsymbol{\hat{\omega}} = \arg \max_{\boldsymbol\omega}\{P(\boldsymbol\omega|\boldsymbol Y)\}. 
\end{equation}

Такі $\boldsymbol{\hat{\omega}}$ є основними складовими словника з розпізнавання злитого мовлення.

Насправді завдання змоделювати безпосередньо $P(\boldsymbol\omega|\boldsymbol Y)$ є досить складним, тому застосувавши правило Байєса, отримаємо еквівалентну задачу пошуку 

\begin{equation}
\label{eq:2}
\boldsymbol{\hat{\omega}} = \arg \max_{\boldsymbol\omega}\{p(\boldsymbol Y|\boldsymbol\omega)P(\boldsymbol\omega)\}.
\end{equation}

Акустична модель визначає міру схожості $p(\boldsymbol Y|\boldsymbol\omega)$, а ймовірність $P(\boldsymbol\omega)$ визначає лінгвістична модель. Ці дві моделі в сукупності формують генеративну модель розпізнавання мовленнєвого сигналу.

В акустичній моделі кожне слово $\omega$ розкладається на послідовність $K_\omega$ фонем, тобто послідовність $q^{(\omega)}_{1:K_\omega} = q_1,\dots,q_{K_\omega}$ є фонемною транскрипцією слова. Зважаючи на те, що вимова може бути різною, міру схожості $p(\boldsymbol Y|\boldsymbol\omega)$ можна обчислити наступним чином, врахувавши багато фонемних транскрипцій:
 
\begin{equation}
\label{eq:3}
p(\boldsymbol Y|\boldsymbol\omega) = \sum_{\boldsymbol Q} p(\boldsymbol Y|\boldsymbol Q)P(\boldsymbol Q|\boldsymbol\omega),
\end{equation} 

де сума обчислюється по всім послідовностям вимов $\boldsymbol\omega$, а $\boldsymbol Q$ ~--- послідовність вимов, для котрої виконується 

\begin{equation}
\label{eq:4}
P(\boldsymbol Q|\boldsymbol\omega) = \prod_{l=1}^L P(\boldsymbol q^{\omega_l}|\omega_l), 
\end{equation} 

де кожна $\boldsymbol q^{(\omega_l)}$ ~--- допустима вимова слова $\omega_l$~.\cite{l1}

Кожна фонема $q$ представляється акустичною генеративною моделлю, як показано на рисунку ~\ref{acusticmodel}. Вона має наступні параметри: ймовірності переходу зі стану в стан $\{a_{ij}\}$ та ~$\{b_j()\}$--- розподіли у просторі первинних ознак для робочих станів. 

\begin{figure}[h]
\noindent
\centering{
\includegraphics[width=75mm]{11.jpg}
}
\caption{Акустична генеративна модель}\label{acusticmodel}
\end{figure}

Ці розподіли фактично апроксимують у просторі первинних ознак ті області, через які проходять траєкторії, що відповідають акустичній реалізації фонеми $q$. Такий загальний вигляд має базова ПММ. Технічно, перехід від робочого стану генеративної моделі до одного зі станів, з яким робочий стан пов’язаний, здійснюється за одиницю відліку часу, а матриця $\{a_{ij}\}$ залежить від топології ПММ та має вигляд стохастичної матриці, що формує ланцюг Маркова.~\cite{l2}


\textbf{Використання нейронних мереж}

Цей підхід моделює процес розпізнавання в біологічних системах. Нейронна мережа --- апаратні чи програмні засоби, що моделюють роботу мозку людини. Як і будь-яка модель, вони є лише наближенням. Але навіть не дивлячись на те, що в подібних засобах імітуються лише окремі сторони бібліотечного прототипу, вони вже зараз дають змогу досягти певних успіхів в багатьох областях, особливо в пов'язаних з класифікацією і розпізнанням образів.

Як відомо, нервова система людини складається з великої кількості елементів --- нейронів, які поєднуються між собою ниткоподібними відростками --- дендритами. Збудження чи заторможення передається від нейрона до нейрона по дендритам, де ті приймають сигнали в точках з'єднання, так званих синапсами. Прийняті синапсом сигнали передаються нейрону, де суммуються. Якщо рівень збудження перевищує деяку порогову величину, збудження передається з тіла нейрону у вихідну точку, яка зветься аксоном, звідки по дендритам поступає в інші нейрони.

Саме наведені вище характеристики і стали суттєвими при створенні штучних нейронних мереж.

Основу нейронної мережі складає, як правило, однотипні елементи, імітуючі роботу біологічного нейрону, які так само й називаються. Кожен з нейронів в кожен момент часу знаходиться, як і біологічний нейрон, в деякому стані. Він має групу однонаправлених вхідних зв'язків --- синапсів, які проходять від входу в мережу чи від інших нейронів. Крім того він має один однонапрямлений зв'язок --- аксон.

Синаптические связи характеризуються вагою wi. Поточний стан S нейрона дорівнює зважуваній сумі входів: $$S = \sum_{i=1}^{n}X_i\omega_i.$$
В векторному вигляді це можна записати як $S = XW$, тобто вектор $S$ є добутком вектора вхідних значень $X$ і матриці вагів $W$ , в якій рядки відповідають шарам, а стовпці --- нейронав в середині кожного шару.

Функція $S$ далі перетворюється активаційною функцією $F$ і дає вихідний сигнал $Y$ нейрона.

$Y = F(S)$ --- активаційна функція мусить мати таку властивість, як різкий зріст на короткому інтервалі аргументу в околі крайнього значення $T$, приймати приблизно одне значення до цього інтервалу і приблизно одне (велике) значення - після цього інтервалу. Цим вимогам відповідає, наприклад, функція $Y$, що дорівнюю 1 при $S>T$, s 0 ghb $S<=T$. Ця функція також називається функцією одиничкого стрибка.~\cite{u1}


\Subsection{Аналіз методів побудови лінгвістичної моделі.} %\label{sec:exist:focuscontext}

\textbf{Побудова лінгвістичної моделі без кластеризації.}

Отриманий звуковий сигнал в результаті препроцесингу перетворюється в послідовність акустичних векторів фіксованого виміру $\boldsymbol Y_{i:T} = y_1,\dots,y_T$. Далі декодер намагається встановити послідовність слів $\boldsymbol\omega_{i:L} = \omega_1,\dots,\omega_L$, які найімовірніше утворили $Y$, тобто декодер шукає

\begin{equation}
\label{eq:5}
\boldsymbol{\hat{\omega}} = \arg \max_{\boldsymbol\omega}\{P(\boldsymbol\omega|\boldsymbol Y)\}. 
\end{equation}

Такі $\boldsymbol{\hat{\omega}}$ є основними складовими словника з розпізнавання злитого мовлення.

Насправді завдання змоделювати безпосередньо $P(\boldsymbol\omega|\boldsymbol Y)$ є досить складним, тому застосувавши правило Байєса, отримаємо еквівалентну задачу пошуку 

\begin{equation}
\label{eq:6}
\boldsymbol{\hat{\omega}} = \arg \max_{\boldsymbol\omega}\{p(\boldsymbol Y|\boldsymbol\omega)P(\boldsymbol\omega)\}.
\end{equation}


В лінгвістичній моделі ймовірність послідовності слів $\boldsymbol\omega = \omega_1,\dots,\omega_K$, що зазначалась в~\eqref{eq:2}, визначається як 
\begin{equation}
\label{eq:7}
P(\boldsymbol\omega) = \prod_{k=1}^K P(\omega_k|\omega_{k-1},\dots,\omega_1).
\end{equation}

У великих лінгвістичних базах розпізнавання кількість попередніх слів у~\eqref{eq:5} може бути досить великою, тому її скорочують до $N-1$ для можливості подальших обчислень і формування ~$N$-грамної лінгвістичної моделі:

\begin{equation}
\label{eq:8}
P(\boldsymbol\omega) = \prod_{k=1}^K P(\omega_k|\omega_{k-1},\omega_{k-2},\dots,\omega_{k-N+1}),
\end{equation}

де $N$ лежить в межах~2--4. Ймовірності~$N$-грам оцінюються за текстовим корпусом, підраховуючи кількість входжень ~$N$-грам.~\cite{l1}


\textbf{Побудова кластиризованої лінгвістичної бази.} 


Нехай $W$ --- послідовність всіх реалізацій слів $(w_1, w_2, w_3,\dots)$ з вибірки на основі деякого текстового корпусу, $V$ --- словник, або множина слів з $W$. Тоді для біграмного контексту найкраще об'єднання слів в класи призводить до максимума

\begin{equation}
\label{a}
P_{class}W = \prod_{x,y \in V}  P_{class}(x|y)^{C(x,y)}
\end{equation}


де $(x,y)$ означає пару слів, в котрій слово $x$ слідує за словом $y$ в послідовності $W$, а функція $C(\cdot)$ визначає частоту спостереження аргументу у виборці. Для уникнення проблем з малими величинами використовується логарифмування

\begin{equation}
\label{b}
\log P_{class}W = \sum_{x,y \in V}C(x,y) \log P_{class}(x|y)
\end{equation}

Після перетворень над $P_{class}(x|y)$ маємо:

\begin{equation}
\label{c}
F_G =\sum_{g,h \in G}C(g,h) \log C(g,h)- 2\sum_{g \in G}C(g) \log C(g)
\end{equation}

де $(hg,)$ означає слідування класу $g$ за класом $h$.

Ідея пошуку деякого найкращого об'єднання слів в задане число класів полягає в обчисленні зміни критерія $F_G$ при гіпотетичному відношенні кожного зі слів у альтернативні класи з наступним переміщенням до класу з найбільшим критерієм. Таким чином, класи обмінюються словами, тобто здійснюють певний словообмін до тих пір, доки критерій не перестає покращуватись. 

Даний алгоритм відноситься до сімейства ,,жадібних'' і не гарантує глобального екстремума.~\cite{l3}



\Section{Висновок 1}
Як було з'ясовано, більшість алгоритмів нейронних мереж погано працюють або взагалі не працюють з лінійними функціями. На відміну від нейронних мереж, математична структура скритих марковських моделей дуже велика і дозволяє вирішувати математичні проблеми різних областей науки. Правильно спроектована марковська модель дає гарні результати роботи. Перевага моделі заключається в тому, що розмітка і кожний елемент характеризується своїми марковськими випадковими полями, не залежними один від іншого. Це дозволяє легко модифікувати розпізнавальну систему.

Питання використання кластеризованої лінгвістичної бази даних чи звичайної лишається відкритим, так як немає однозначної оцінки лінгвістичної бази.









\newpage
\chapter{МАТЕМАТИЧНЕ МОДЕЛЮВАННЯ}
\Section{Модель багатозначного перетворення послідовностей символів}

Нехай маємо скінченну послідовність символів
\begin{equation}
\label{eq1}
(a_1, a_2,\dots , a_n,\dots ,a_N)\equiv a_1^N, a_n\in \text{\textbf{A}},
\end{equation}

де \textbf{А} --- алфавіт вхідних символів. Сконструюємо відображення цієї послідовності на множину послідовностей вихідних символів із деякого іншого алфавіту \textbf{В}.

Розглянемо функцію $f$, що відображає послідовність  $a_1^N$, починаючи з її $n$-го символу, у символ алфавіту \textbf{В} або порожню множину:

\begin{equation}
\label{eq2}
f:a_n^N \rightarrow b, b\in \textbf{B} \cup \varnothing , 1\le n\le N  
\end{equation}

Зауважимо, що \eqref{eq2} має місце лише у випадку, коли вхідна послідовність належить області визначення $f$, тобто $a_n^N \in Def(f)$. Множина послідовних застосувань таких функцій переводить $a_n^N$ у послідовності символів з алфавіту \textbf{В}, утворюючи таким чином мультифункцію: 

\begin{equation}
\label{eq3}
F\left(a_n^N\right)= \{ \left(f_1^k(a_n^N)\right),\left(f_2^k(a_n^N)\right), \cdots ,\left(f_{L_k}^k(a_n^N)\right) \in \textbf{B}^{L_k} \cup \varnothing, 1\leqslant k \leqslant K_F \},
\end{equation}

де $L_k$ --- довжина $k$-ї вихідної послідовності, загальна кількість яких, $K_F$, своя для кожного  $F\in \textbf{F}$

Визначимо аналог прямого добутку над множинами, отриманими внаслідок дії мультифункцій з \textbf{F}, як перебір усіх варіантів об'єднання скінченних послідовностей символів з алфавіту \textbf{B}. Тобто, опускаючи аргументи мультифункцій:

\begin{equation}
\label{eq4}
F\otimes G = \{ \left(f_1^u,f_2^u,\dots ,f_{L_k}^u, g_1^v,g_2^v,\dots ,g_{L_k}^v\right), 1\leqslant u\leqslant K_F, 1\leqslant v\leqslant K_G \}. 
\end{equation}

Припускаємо за визначенням, що якщо результат дії $F$ або $G$ є порожньою множиною, то результатом їх добутку буде порожня множина. На відміну від декартового добутку для визначеного нами аналогу виконується властивість асоціативності. 

Розглянемо впорядковану множину $\tilde{F}$ мультифункцій $F\in \textbf{F}$, які супроводимо додатковими параметрами:

\begin{equation}
\label{eq5}
\tilde{F}=\left(F_{i,d_i,\delta_i}\right), 1\leqslant i \leqslant |\tilde{F}|, d_i>0,\delta_i=\{0,1\}, 
\end{equation}

де $i$ є індексом мультифуннкцій у впорядкованій множині $\tilde{F}$; параметр $d_i$ назвемо шириною кроку аналізу, $\delta_i$ --- ,,умовою виключності''. Через ці параметри конструюємо обмеження при обчисленні добутку 

\begin{equation}
\label{eq6}
\otimes_{i,n}F_{i,d_i,\delta_i}\left(a_n^N \right), 1 \leqslant i \leqslant |\tilde{F}|, 1\leqslant n\leqslant N.
\end{equation}

Припустимо, що ми вже обчислили вираз \eqref{eq6} на деяких упорядкованих індексних множинах $J$ і $M$ і отримали деяку непорожню множину 

\begin{equation}
\label{eq7}
G_{J,M} = \underset{\substack{u\in J\\ v\in M}}{\otimes} F_{u,d_u,\delta_u}\left(a_v^N\right).
\end{equation}

Нехай $j$ та $m$ є останніми елементами індексних множин $J$ і $M$ відповідно. Тоді при розгляді наступної компоненти добутку,  $F_{i,d_i,\delta_i}\left(a_n^N \right)$, проводимо обчислення згідно з визначенням \eqref{eq4}, якщо виконуються такі умови:

\begin{equation}
\label{eq8}
\begin{cases}
m + d_i = n; \\
\left[\begin{array}{l l}
\delta_r \not= 1, 1\leqslant r < 1;\\
\underset{\substack{u\in J\\ v\in M}}{\otimes} F_{u,d_u,\delta_u}\left(a_v^N\right) \otimes F_{r,d_r,\delta_r}\left(a_v^N\right) \not= \varnothing, 1\leqslant r < i, \text{якщо} \delta_i =1 
\end{array}\right.
\end{cases}
\end{equation}

В іншому випадку, при надходженні наступрої компоненти добутку отримуємо порожню множину.

Виразом \eqref{eq6} породжуються послідовності вихідних символів за деякою послідовністю вхідних символів. Якщо вхідний алфавіт збігається з алфавітом літер певної мови, а вихідний алфавіт складається з фонем, то маємо багатозначний транскриптор орфографічного тексту. І навпаки, якщо на вході --- фонемний алфавіт, а на виході --- алфавіт літер, то отримаємо багатозначне перетворення з фонемного тексту на орфографічний.~\cite{l4}



\Section{Визначення міри схожості фонем за CMED}

Computing minimal edition distance (CMED) --- у теорії інформації і комп'ютерній лінгвістиці міра відмінності двох послідовностей символів (рядків).~\cite{u1}

Для пофонемного порівняння двох рядків символів можна застосувати наступний метод: будуємо граф порівняння символів (рис. \ref{p1}) з накладанням ,,штрафів'' за розміром яких і визначаємо CMED.

,,Штрафи'' будемо накладати згідно наступних критеріїв: 

\begin{itemizer}
\item ступінь огубленості (є, немає);
\item місце творення (передня, середня, задня частини ротової порожнини);
\item назалізованість;
\item палаталізація (пом'якшення);
\item вокалізація або наявність ,,голосу'' при творенні.
\end{itemizer}

Наприклад, якщо фонеми різні --- ,,штраф'' = 1, якщо фонеми різні, але відносяться до одного типу (голосні, приголосні) --- 0.9, якщо це одна й та ж голосна фонем, але в різних станах наголошеності --- 0.2, якщо це одна й та ж приголосна фонем, але в різних станах пом'якшення --- 0.3 і т. д.

\begin{figure}[hb] 
 	\centering 
 	\includegraphics[scale=0.7]{g1.png} 
 	\caption{--- Граф порівняння двох фонетичних послідовностей з накладанням ,,штрафів''}\label{p1}
\end{figure}



\Section{Оцінка лінгвістичної бази}

Для оцінки лінгвістичних бази пропонується використовувати perplexity --- міра в теорії інформації. 

Perplexity показує як точно ми можемо прогнозувати розпізнання наступних bgram. Чим меньший даний показник, тим вірогідніше розпіпізнання послідовності слів, що утворюють наступну bgram.

Perplexity визначається, як $b$ піднесене до степеня ентропії з базою $b$, або, частіше, $b$ піднесене до степеня перехресної ентропії з базою $b$. Perplexity дискретного розподілу $p$ визначається як:

\begin{equation}
\label{eq9}
2^{H(p)}=2^{-\sum_x \log_2 p(x)}
\end{equation}

де $H(p)$ --- ентропія розподілу, а $x$ приймає всі можливі значення.~\cite{u1}

В даній роботі для знаходження perplexity використовується наступне програмне забезпечення: The CMU Statistical Language Modeling (SLM) Toolkit.

 







\chapter{ПРОГРАМНА РЕАЛІЗАЦІЯ}
\Section{Опис розроблених програмових засобів}

В даній дипломній роботі було спроектовано та розроблено програмний модуль для програми кластеризації лінгвістичної бази даних і знань \emph{clastlm}.

Даний модуль складається з наступних підпограм:

\begin{itemizer}
\item \emph{grapheme-to-phoneme conversion} --- підпрограма, яка отримує на вхід 2 значення типу string, проводить їх претворення та повертає набір фонем для кожного з вхідних параметрів;
\item \emph{phoneme comparison} --- здійснює порівняння двох фонем, та згідно критеріям штрафів визначає штраф типу float;
\item \emph{calculation CMED} --- отримує на вхід штраф, сумує їх, та повертає значення типу boolean, відповідно до наявності штрафу.
\end{itemizer}

\Section{Архітектура розроблених програмових засобів}

Розроблений модуль містить наступні програмні блоки (рис. \ref{b1}):

\begin{figure}[hb] 
 	\centering 
 	\includegraphics[scale=0.7]{b1.png} 
 	\caption{--- Структурна схема взаємодії програм}\label{b1}
\end{figure}

\begin{itemizer}
\item головна програма --- керування всим розробленим програмним засобом;
\item ввід даних --- підпрограма зчитування вхідних даних, переданих від основної програми;
\item орфографічно-фонемне перетворення --- підпрограма перетворення орфографічних символів до фонемних;
\item пофонемне порівняння з накладанням штрафу --- підпрограма реалізації пошуку CMED;
\item ввивід даних --- підпрограма, що передає результат роботи модуля основній програмі.
\end{itemizer}

На рисунку \ref{b2} показаний алгоритм роботи модуля кластеризації лінгвістичної бази.

\begin{figure}[!hb] 
 	\centering 
 	\includegraphics[scale=0.55]{b2.png} 
 	\caption{--- Алгоритм роботи модуля} \label{b2}
\end{figure}

\Section{Засоби керування програмою}

\begin{figure}[hb] 
 	\centering 
 	\includegraphics[scale=0.7]{s1.png} 
 	\caption{--- Вікно запуску програми з відкритою ,,Довідкою''}\label{s1}
\end{figure}
 
 Для роботи з програмою \emph{clastlm} користувачу необхідно в командному рядку вказати деякі з наступних наступні параметри:
 \begin{itemizer}
\item -c <str> - file of uni- and bi- gram counters
\item -pc <str> - make classes with phoneme comparison
\item -mi int - max iterations
\item -mc int - max clusters
\item -vc int - vary (grow) classes each iteration starting from int
\item -vi int - grow classes each iteration with int increment
\item -va float - grow classes each iteration with float accelleration
\item -vm float - multiply classes each iteration with float
\item -ic int - set initial iteration class count to int
\item -rg <str> - read classes from file <str>
\item -wg <str> - write classes to file <str>
\item -wi <str> - after iterations write classes to file prepended with <str>
\item -rl <str> - read classes from a log file
 \end{itemizer}
 Вхідний файл для програми --- файл з розширенням \emph{.cnt}, що містить unigram та bgram та їх частоту зустрічань в текстовому корпусі.
 
 Щоб запустити програму \emph{clastlm} з модулем для розділення фонетично близьких слів до різних класів, користувачу потрібно обов'язково вказати параметр -pc (див. рис. \ref{s1}).
 
 
\Section{Засоби розробки програмного забезпечення}

Дана курсова робота була розроблена за допомогою QT --- крос-платформений інструментарій розробки ПЗ на мові програмування C++. 

Даний інструментарій для реалізації програми був обраний тому що, дозволяє реалізувати програму з високою суміснітю з різними операційними системами, з високим рівнем захисту інформації, масштабування.

\Section{Вимоги до складу й параметрів технічних засобів}

Вимоги до персонального комп’ютера, на якому буде використовуватись розроблене програмне забезпечення:

\begin{itemizer}
\item Intel Core2Duo 2ГГц або вище;
\item 3 Гб ОЗУ;
\item 10 Гб вільного місця на диску;
\item монітор, клавіатура, миша.
\end{itemizer}

\Section{Вимоги до програмної та інформаційної сумісності}

Розроблена програма може працювати на таких операційних системах, як Microsoft Windows 2000/XP/Vista/7 та GNU/Linux.


\chapter{ОХОРОНА ПРАЦІ}
\label{chap:ot}

В даній дипломній роботі було розроблено програму для користувачв ПК, тому в даному розділі описане робоче місце користувача цією програмою, всі небезпечні виробничі фактори та інструкція з техніки безпеки на цьому робочому місці.

\Section{Характеристики робочого місця}

\Subsection{Гігієнічні вимоги до виробничих приміщень з ЕОМ}
	
Розглянемо санітарні норми на прикладі окремого приміщення, що зображене на Рис.\ref{fig:plan1}. На малюнку позначено два робочі місця.
 
\begin{figure}[ht] 
 	\centering 
 	\includegraphics[scale=0.6]{plan1.png} 
 	\caption{--- Схема робочого приміщення} \label{fig:plan1} 
\end{figure}


Приміщення відповідає вимогам НПАОП 0.00-1.28-10 \cite{oh1}, і, за виключенням недостатньої інтенсивності природного освітлення, вимогам ДБН В.2.5-28-2006\cite{oh2}.

Дане приміщення призначене для роботи з ЕОМ, і тому знаходиться на другому поверсі (відповідно до НПАОП 0.00-1.28-10, такі приміщення не повинні знаходитись в підвальних чи цокольних приміщеннях). Площа на одне робоче в даному приміщенні становить 7.5 кв. м., що більше встановленої нормами мінімальної площі - 6,0 кв. м. Об\rq{}єм на одне робоче місце (22.5 куб. м.) теж задовольняє нормам (не менше 20,0 куб. м). Приміщення має природне та штучне освітлення (відповідно до ДБН В.2.5-28-2006).

Для внутрішнього оздоблення приміщення використані дифузно-відбивні матеріали з коефіцієнтами відбиття для стелі 0.7 (вимоги --- 0.7-0.8), для стін і підлоги 0.5 (0.5 --- 0.6 і 0.3 --- 0.5 відповідно). Поверхня підлоги є матовою, рівною, неслизькою, з антистатичними властивостями. Для оздоблення інтер'єру не використовуються полімерні матеріали (ДСП, шпалери, що миються, рулонні синтетичні матеріали, шаруватий паперовий пластик тощо). 

В приміщенні встановлені системи опалення, кондиціювання повітря. Приміщення оснащене аптечками першої медичної допомоги.

У приміщенні щоденно проводиться вологе прибирання.

\Subsection{Організація робочих місць та вимоги до розміщення ЕОМ}

Обладнання і організація робочих місця в приміщенні забезпечують відповідність конструкції всіх елементів робочого місця та їх взаємного розташування ергономічним вимогам з урахуванням характеру і особливостей трудової діяльності (НПАОП 0.00-1.28-10)~\cite{oh1}.

Робочі місця з так розташовані відносно світових прорізів, щоб природне світло падало збоку. При розміщенні робочих столів з ВДТ відстані між бічними поверхнями ВДТ перевищує 1,2 м, відстань від тильної поверхні одного ВДТ до екрана іншого --- 2,5 м.

Конструкція робочих столів відповідає сучасним вимогам ергономіки і забезпечує оптимальне розміщення на робочій поверхні використовуваного обладнання (дисплея і клавіатури).

Висота робочої поверхні робочого столу з ВДТ регулюється в межах 680 - 800 мм. Ширина і глибина робочої поверхні - 1300мм і 1000мм відповідно, що забезпечує можливість виконання операцій у зоні досяжності моторного поля.
Робочі столи мають простір для ніг заввишки 650 мм, завширшки 1000 мм, завглибшки 800 мм.

Робочі стільці обладнані підйомно-поворотним механізмом, мають регульований кут нахилу сидіння та спинки. Висота поверхні сидінь регулюється в межах 400 - 500 мм. Кут нахилу сидіння - до 15 град. вперед і до 5 град. назад. 

Екран ВДТ розташовується на відстані 650 мм, від очей користувача, що відноситься до оптимального інтервалу відстаней. Розташування екрана ВДТ забезпечує зручність зорового спостереження у вертикальній площині під кутом + 30 град. до нормальної лінії погляду працюючого.

Клавіатури розташовуються на поверхні столу на відстані 200 мм від краю, звернутого до працюючого. Поверхня клавіатур є матовою з коефіцієнтом відбиття 0,4.

Для забезпечення захисту і досягнення нормованих рівнів комп'ютерних випромінювань застосовуються приекранні фільтри.

\Section{Аналіз шкідливих і небезпечних виробничих факторів}

\Subsection{Мікроклімат виробничих приміщень}

Згідно з санітарними нормами мікроклімату виробничих приміщень ДСН 3.3.6.042-99 \cite{oh3} 
показники температури повітря в робочій зоні по висоті та по горизонталі, а також протягом робочої зміни не повинні виходити за межі нормованих величин оптимальної температури для даної категорії робіт (Іб) --- $21–23^{\circ}C$ у холодний період року та $22–24^{\circ}C$ у теплий, відносна вологість мусить бути в межах 60-40\%, а швидкість руху повітря мусить становити 0.1 м/с у холодний період року та 0.2 м/с у теплий.

Температура внутрішніх поверхонь робочої зони (стіни, підлога, стеля), технологічного обладнання (екрани і т. ін.), зовнішніх поверхонь технологічного устаткування, огороджуючих конструкцій не повинна виходити більш ніж на 2 град.C за межі оптимальних величин температури повітря для даної категорії робіт (Іб), тобто бути в межах $19-25^{\circ}C$ у холодний період року та $20–26^{\circ}C$ у теплий. 

Для дотримання оптимальних значень температури в холодний період року слід застосовувати секційні радіатори водяного опалення.
Для дотримання оптимальних значень температури в теплий період року приміщення має бути додатково обладнане системою опалення-охолодження повітря. Для цього встановлюється кондиціонер типу ,,Toshiba RAS-18SKHPE'' з потужність охолодження 5,1 кВт
та потужністю обігрівуч 5,45 кВт. В міжсезонний перехідний період кондиціонер використовується як обігрівач.

\Subsection{Виробничий шум}
Згідно ДСН 3.3.6.037-99\cite{oh4} допустимі  рівні звуку, еквівалентні рівні звуку та рівні звукового тиску в октавних смугах частот повинні  бути такими, як в табл.\ref{t1}.


\begin{table}[ht]
\centering
\begin{supertable}{|c|c|c|c|c|c|c|c|c|c|c|}{11}{{\large Нормовані рівні звукового тиску (дБ) та рівні шуму (дБА) на робочих місцях}}{t1}
\hline
Вид трудової & \multicolumn{9}{|c|}{Рівні звукового тиску в октавних смугах} &   Рівень звуку \\
діяльності & \multicolumn{9}{|c|}{з середніми геометричними частотами} &  в дБА \\ \cline{2-10}
& 31.5 & 63 & 125 & 250 & 500 & 1000 & 2000 & 4000 & 8000 &\\
\hline	
Програмування на ЕОМ & 86 & 71 & 61 & 54 & 49 & 45 & 42 & 40 & 38 & 50\\
\hline	
\end{supertable}
\end{table}

Основними джерелами шуму на робочому місці є: вентилятори охолодження, жорсткий диск, приводи дисководів, CD-ROM, шум клавіатури і шум периферійних пристроїв.

Рівень шуму на робочому місці не повинен перевищувати 50 дБА. 

За принципом енергетичного підсумовування рівня інтенсивності окремих джерел, визначимо середній рівень шуму на робочому місці:

$$L_{\sum} = 10\cdot lg\sum_{i=1}^n 10^{0,1L_i},$$

де $L_i$ --- рівень звукового тиску $i$-го джерела шуму.

Рівні звукового тиску кожного джерела шуму:

\begin{itemize}
\item жорсткий диск --- 26 дБА;
\item вентилятор охолодження --- 27 дБА;
\item удари по клавіатурі --- 30 дБА;
\item клік миші --- 31 дБА;
\end{itemize}

$$L_{\sum} = 10\cdot lg(10^{0,1\cdot 26} + 10^{0,1\cdot 27} + 10^{0,1\cdot 30} + 10^{0,1\cdot 31}) = 35 дБА,$$

Таким чином ми отримали допустимий рівень шуму, тому засоби захисту не потрібні.

\Subsection{Освітлення виробничого приміщення}

При проектуванні приміщень з постійним перебуванням людей необхідно дотримуватися ДБН В.2.5-28-2006 \cite{oh2}: у приміщеннях з комп'ютерним обладнанням необхідно застосувати систему комбінованого освітлення.

Природне освітлення має здійснюватись через світлові прорізи, 
орієнтовані переважно на північ чи північний схід і забезпечувати коефіцієнт 
природної освітленості (КПО) не нижче ніж 1.5\%.

Визначимо, чи відповідає природне освітлення приміщення, вказаного на рис.~\ref{fig:plan1} необхідним межам.
Дане приміщення має дає 3м довжини, 5 м ширини та 3 м висоти. Наявні два подвійних вікна з віконним листовим склом  шириною та висотою 1,5 м. Рами є дерев’яними, подвійними та окремими.
	
Знайдемо нормоване значення коефіцієнта природного освітлення $e_N$ (КПО) для даного приміщення (з рис.\ref{fig:plan1}) за формулою (\ref{eq:kpon}): 
\begin{equation}
e_{N} = e_{n}\cdot m_{N}, \label{eq:kpon}
\end{equation}
де $e_{n}$ --- значення КПО, що залежить від типу зорової роботи та типу освітлення, $m_{N}$ --- коефіцієнт світлового клімату.

Оскільки ми розглядаємо роботу з ПК, маємо $ e_{n} = 1.5\%. $
 
Потрібно знайти мінімальну кількість ламп, необхідну для забезпечення освітлення приміщення, що буде відповідати встановленим нормам.

Розглянемо робоче приміщення зображене на рис.~\ref{fig:plan1}. Визначимо тип приміщення як кабінет, відповідний розряд та підрозряд зорової роботи визначимо як 1б \cite{oh2}. Вирішено використовувати дволампові світильники типу ШОД, в кожному з яких знаходяться люмінесцентні лампи ЛДЦ-P потужністю 40~Вт. Нехай також середньо-виважений коефіцієнт відбиття стін та підлоги дорівнює 0.5. 
		
Для вказаного типу зорових робіт (1б) необхідна загальна освітленість величиною 300лк: $E=300 лк$.

Світловий потік кожної лампи типу ЛДЦ-P дорівнює: $F=2200 ЛМ$.	

Кількість ламп у світильнику: $n=2$.	
	
Довжина приміщення: a=5, ширина приміщення: b=3, висота приміщення: h=3.

Необхідну кількість світильників можна знайти зі співвідношення (\ref{eq:mainlight}).
	
		\begin{equation}
		\label{eq:mainlight}
		N= \frac{EK_зS_z}{nF\eta},	
	\end{equation}

де $K_з$ --- коефіцієнт запасу,

$S$ --- площа приміщення,

$\eta$ ---  коефіцієнт використання світлового потоку, що залежить від індексу приміщення. 

	Знайдемо індекс приміщення.
	
	\begin{equation}
		\label{eq:index}
		i= \frac{ab}{h(a+b)}=\frac{5\cdot3}{3(5+3)}=0.7	
	\end{equation}
	

	Для вказаного індексу приміщення та середньозваженого коефіцієнта відбиття стелі, стін та підлоги можна знайти коефіцієнт використання світлового потоку:
	
		\begin{equation}
		\label{eq:eta2}
		\eta =0.42.	
	\end{equation}

За співвідношенням (\ref{eq:mainlight}) знаходимо кількість світильників:

	\begin{equation}
		\label{eq:mainlightnum}
		N= \frac{300\cdot1.2\cdot15\cdot1.1}{2\cdot2200\cdot0.42}=3.21\approx4.	
	\end{equation}

Отже, для досягнення необхідного показника освітлення необхідно використати 4 світильника.


\Subsection{Електробезпека у виробничому приміщення}

Робоче приміщення повинно відповідати вимогам НПАОП 0.00-1.28-10.
При проектуванні системелектропостачання, монтажі силового електроустаткування і електричногоосвітлення в будівлях і приміщеннях для ЕОМ необхідно дотримуватися вимог нормативно-технічній документація. Комплекс необхідних заходів з технікибезпеки визначається, виходячи з видів електроустановки, її номінальної напруги, умов середи, типа приміщення і доступності електроустаткування.
 
ЕОМ є однофазним споживачем електроенергії, що живиться відзмінного струму 220В від мережі із заземленою нейтраллю. IBM РСвідноситься до електроустановок до 1000В закритого виконання, всіструмопровідні частини знаходяться в кожухах. За способом захисту людинивід ураження електричним струмом, ЕОМ і периферійна техніка повиннівідповідати 1 класу захисту.
 
НПАОП 0.00-1.28-10 передбачені такі заходи електробезпеки: конструктивні,схемно-конструктивні, експлуатаційні.
 
Конструктивні заходи забезпечують захист від випадкового дотику дострумопровідних частин за допомогою захисних оболонок і ізоляції струмоведучих частин. Ступінь захисту оболонки повинен відповідати класу пожежонебезпечної зони приміщення П-IIа. Для ступеня захисту оболонки IP-44.

Схемно-конструктивні заходи призначені для забезпечення захисту відураження електричним струмом при дотику до металевих оболонок, які можутьопинитися під напругою в результаті аварії. У даному приміщенні в 71 комп'ютерах застосовується занулення. Біля монітора передбачена подвійнаізоляція.

Експлуатаційні заходи. Необхідно дотримувати правила техніки безпеки при роботі з високоюнапругою і наступних запобіжних засобів:

 - монтаж, обслуговування, ремонт і наладка ЕОМ, заміна деталей,пристосувань, блоків повинна здійснюватися тільки при повному відключенніживлення;
 
 - заземлені конструкції приміщення мають бути надійно захищенідіелектричними щитками або сітками від випадкового дотику.


\Section{Пожежна безпека}

Відповідно до НАПБ Б.07.005186 приміщення за вибухопожежною та пожежною небезпекою відноситься до категорії В --- пожежонебезпечні. У приміщенні є пожежонебезпечна зона класу П-ІІа – простір у приміщенні, у якому знаходяться тверді горючі речовини та матеріали та вибухонебезпечна зона класу 20 – простір, у якому під час нормальної
експлуатації вибухонебезпечний пил у вигляді хмари присутній постійно або
часто у кількості, достатній для утворення небезпечної концентрації суміші з
повітрям, і простір, де можуть утворюватися пилові шари непередбаченої або
надмірної товщини.

В приміщенні посередині стелі має бути встановлений один димовий 
пожежний сповіщувач СПД-3 відповідно до ДБН В.1.1.-7-2002~\cite{oh6} –-- з 
розрахунку один на висоту до 3,5 м та загальною площею не більше ніж 86 . 
Сигналізація виведена на диспетчерський пожежний пульт.

Приміщення слід оснащувати вуглекислотними вогнегасниками ВВ-5, 
з розрахунку 2 штуки на кожні 20 м2,
в даному приміщенны їх необхідно 2 штуки.

Максимально допустима відстань від можливого осередку пожежі
до місця розташування вогнегасника має бути: 30 м – для приміщень даної категорії(В).

Як правило, первинні засоби пожежогасіння розміщуються на пожежних
щитах або стендах, які встановлюються на території об’єкта з розра1
хунку один щит (стенд) на площу 5000 м2.

План евакуації з приміщення розташований біля вхідних дверей.
 площі приміщення. 
 
\Section{Інструкція з техніки безпеки}
	
\begin{enumerator}
\item Виконувати умови інструкції з експлуатації ПК.
\item При експлуатації ПК  необхідно  пам'ятати,  що  первинні мережі  електроспоживання під час роботи знаходяться під напругою, яка є небезпечною для життя людини,  тому необхідно  користуватися справними    розетками,    відгалужувальними    та  з'єднувальними коробками, вимикачами та іншими електроприладами.
\item До  роботи  з  ПК  допускаються  працівники,   з   якими проведений   вступний   інструктаж  та  первинний  інструктаж  (на робочому місці) з питань охорони праці,  техніки безпеки, пожежної безпеки  та  зроблений  запис  про  їх  проведення  у спеціальному журналі інструктажів.
\item Працівники при роботі з ПК повинні  дотримуватися  вимог техніки безпеки, пожежної безпеки.
\item При   виявленні   в  обладнанні  ПК  ознак  несправності (іскріння,  пробоїв,  підвищення температури,  запаху гару,  ознак горіння)   необхідно  негайно  припинити  роботи,  відключити  усе обладнання  від  електромережі  і  терміново  повідомити  про   це відповідних посадових осіб, спеціалістів.
\item Вміти   діяти   в   разі   ураження   інших  працівників електричним струмом або виникнення пожежі.
\item Знати    місця    розташування     первинних     засобів пожежегасіння,  план евакуації працівників, матеріальних цінностей з приміщення в разі виникнення пожежі.
\end{enumerator}	
	
	
	
\uchapter{ВИСНОВКИ}

Метою даної роботи було вдосконалення лінгвістичної бази даних та знань для розпізнавання мовленнєвого сигналу. 
Для цього в роботі було розглянуто та проаналізовано методи розпізнавання звуку та методи побудови лінгвістичної бази даних для розпізнавання мовлення, на основі чого обрано базові методи для розробки модуля. запропоновано й розроблено математичну модель, яка дозволяє побудувати кластеризовану лінгвістичну базу даних для розпізнавання мовленнєвого сигналу з урахуванням фонетичної схожості слів. застостосовано наступні існуючі методм: орфографічно-фонемного перетворення та computing minimal edition distance. Також розроблено метод накладання “штрафів” при порівнянні фонем. Реалізовано модуль для розробленої математичної моделі, який працює з програмою clastlm, та проведено його тестування.








	
	
\begin{references}
\bibitem{l1} Gales M. The Application of Hidden Markov Models in Speech Recognition / Gales M., Young S. --- Foundations and Trends in Signal Processing, 2007. --- 124p.
\bibitem{l2} ~Робейко ~В.В. Розпізнавання спонтанного мовлення на основі акустичних композитних моделей слів у реальному часі / Робейко В.В., Сажок М.М. // Штучний інтелект. --- \No 4'2011. --- Донецьк, 2011. --- 12c.
\bibitem{l3} Сажок~Н.Н. Кластеризация слов при построении лингвистической модели для автоматического распознавания речевого сигнала // Информационные технологии и системы.~--- 2012. --- с.59-66
\bibitem{l4} Сажок~М.М. Багаторівнева багатозначна модель перетворення орфографічного тексту на  фонемний // Штучний інтелект.~--- 2012. --- с.65-75


\bibitem{oh1} Правила охорони праці під час експлуатації електронно-обчислювальних машин НПАОП 0.00-1.28-10     
\bibitem{oh2} Державні будівельні норми України. Природне і штучне освітлення ДБН В.2.5-28-2006 
\bibitem{oh3} Санітарні норми мікроклімату виробничих приміщень ДСН 3.3.6.042-99
\bibitem{oh4} Санітарні норми виробничого шуму, ультразвуку та інфразвуку ДСН 3.3.6.037-99

\bibitem{oh6} Пожежна безпека об’єктів будівництва ДБН В.1.1.7–2002 

\bibitem{u1} \url{http://uk.wikipedia.org/wiki/}


\end{references}	
	
	
\newpage
\uchapter{Додатки}

\append{1}{Лістинги програм}
	
\lstset{language=C++,basicstyle=\ttfamily\scriptsize}
	Файл clustlm.cpp
	\lstinputlisting{src/clustlm.cpp}



\end{document}