% vim:spelllang=ru,en
\documentclass[a4paper,12pt,notitlepage,headsepline,pdftex]{scrartcl}

\usepackage{cmap} % чтобы работал поиск по PDF
\usepackage[T2A]{fontenc}
\usepackage[utf8]{inputenc}
\usepackage[english,russian]{babel}
\usepackage{concrete}
\usepackage{cite}
\usepackage{url}

\usepackage{textcase}
\usepackage[pdftex]{graphicx}

\usepackage{lscape}

\pdfcompresslevel=9 % сжимать PDF
\usepackage{pdflscape} % для возможности альбомного размещения некоторых страниц
\usepackage[pdftex]{hyperref}
% настройка ссылок в оглавлении для pdf формата
\hypersetup{unicode=true,
            pdftitle={2ой этап выполнения проекта по ПО ЭВМ},
            pdfauthor={Погода Михаил},
            pdfcreator={pdflatex},
            pdfsubject={},
            pdfborder    = {0 0 0},
            bookmarksopen,
            bookmarksnumbered,
            bookmarksopenlevel = 2,
            pdfkeywords={},
            colorlinks=true, % установка цвета ссылок в оглавлении
            citecolor=black,
            filecolor=black,
            linkcolor=black,
            urlcolor=blue}

\usepackage{amsmath}
\usepackage{amssymb}
\usepackage{moreverb}
\usepackage{indentfirst}
\usepackage{misccorr}

\usepackage{xtab}
\usepackage{nccfoots}

\begin{document}
\begin{titlepage}
  \begin{center}
    \large
    \MakeUppercase{Министерство образования и науки,}

    \MakeUppercase{молодёжи и спорта Украины}

    \MakeUppercase{Национальный технический университет Украины}

    \MakeUppercase{,,Киевский политехнический институт''}

    \addvspace{6pt}

    \normalsize
    Кафедра прикладной математики

    \vfill

    \textbf{Етап №2}

    выполнения курсового проекта

    по дисциплине ,,Программное обеспечение ЭВМ''

    \emph{Обзор литературы. Изучение методов решения задачи}
  \end{center}

  \vfill

  \noindent
  Выполнил\\
  студент группы КМ-92\\
  Погода~М.\,В.\\
  \vfill

  \begin{center}
    КИЕВ

    2012
  \end{center}
\end{titlepage}


  При сооружении ограждений зачастую решающим фактором является количество
  израсходованных материалов.
  Поэтому желательно использовать такое ограждение, которые бы охватывало всю
  необходимую территорию, и притом длина его была бы оптимальной.

  Самым простым подходом является сооружение ограждение в форме такого
  прямоугольника, который бы касался бы самой нижней, самой верхней, самой
  левой и самой правой точек.
  Тогда, независимо от формы территории, все её точки окажутся внутри этого
  ограждения.

  Однако, этот метод не даёт оптимальную форму ограждения.
  Как показали исследования, минимальную длину ограждения имеет ограждение в
  форме выпуклой оболочки, построенной для точек территории, которую нужно
  оградить.\cite{book1}

  \emph{Выпуклой оболочкой} множества $\mathbb{X}$ называется наименьшее
  выпуклое множество, содержащее $\mathbb{X}$.
  ,,Наименьшее множество'' здесь означает наименьший элемент по отношению к
  вложению множеств, то есть такое выпуклое множество, содержащее данную
  фигуру, что оно содержится в любом другом выпуклом множестве, содержащем
  данную фигуру.\cite{book2}

  Множество в аффинном пространстве называется \emph{выпуклым}, если оно
  содержит вместе с любыми двумя точками соединяющий их отрезок.

  Существуют несколько методов (алгоритмов), позволяющих построить выпуклую
  оболочку для множества вершин.
  Вот некоторые из них:
  \begin{itemize}
    \item \emph{Алгоритм Грэхема}.
      В этом алгоритме задача о выпуклой оболочке решается с помощью стека,
      сформированного из точек"=кандидатов.
      Все точки входного множества заносятся в стек, а потом точки, не
      являющиеся вершинами выпуклой оболочки, со временем удаляются из него.
      По завершении работы алгоритма в стеке остаются только вершины оболочки
      в порядке их обхода против часовой стрелки.\cite{book3}
    \item \emph{Алгоритм Джарвиса}.
      Метод можно представить как обтягивание верёвкой множества вбитых в
      доску гвоздей.
      Алгоритм работает за время \verb'O(nh)', где \verb'n' --- общее число
      точек на плоскости, \verb'h' --- число точек в выпуклой
      оболочке.\cite{book4}
    \item \emph{Алгоритм Чана}.
      Является комбинацией двух более медленных алгоритмов (сканирование по
      Грэхему  и заворачивание по Джарвису).
      Недостатком сканирования по Грэхему является необходимость сортировки
      всех точек по полярному углу, что занимает достаточно много времени.
      Заворачивание по Джарвису требует перебора всех точек для каждой из
      точек выпуклой оболочки, что в худшем случае занимает квадратичное
      время.\cite{book5}
    \item \emph{Алгоритм быстрой оболочки}.
      Использует идею быстрой сортировки Хоара.
    \item \emph{Алгоритм Киркпатрика}
  \end{itemize}

\bibliographystyle{ugost2008ls}
\bibliography{src}
\end{document}
