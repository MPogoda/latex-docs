% vim:spelllang=uk
\documentclass[a4paper,10pt,notitlepage,pdftex,headsepline]{scrartcl}

\usepackage{a4wide}
\usepackage{cmap} % чтобы работал поиск по PDF
\usepackage[utf8]{inputenc}
\usepackage[ukrainian]{babel}
\usepackage[T2A]{fontenc}
\usepackage{concrete}

\pdfcompresslevel=9 % сжимать PDF
\usepackage{pdflscape} % для возможности альбомного размещения некоторых страниц
\usepackage[pdftex]{hyperref}
% настройка ссылок в оглавлении для pdf формата
\hypersetup{unicode=true,
            pdftitle={Основи охорони праці},
            pdfauthor={Михайло Погода},
            pdfcreator={pdflatex},
            pdfsubject={},
            pdfborder    = {0 0 0},
            bookmarksopen,
            bookmarksnumbered,
            bookmarksopenlevel = 2,
            pdfkeywords={},
            colorlinks=true, % установка цвета ссылок в оглавлении
            citecolor=black,
            filecolor=black,
            linkcolor=black,
            urlcolor=blue}

\author{Михайло Погода}
\title{Основи охорони праці}
\date{\today}

\begin{document}
\begin{titlepage}
  \maketitle
\end{titlepage}

\tableofcontents
\newpage

\section{Загальні питання охорони праці}

  Шкідливий виробничий фактор "--- небажане явище, яке супроводжує виробничий
  процес і вплив якого на працюючого може привести до погіршення самопочуття,
  зниження працездатності, захворювання та навіть смерті як результату
  захворювання.

  Небезпечний виробничий фактор "--- небажане явище, яке супроводжує
  виробничий процес і дія якого за певних умов може привести до травми або
  іншого раптового погіршення здоров’я працівника й навіть до раптової смерті.

  За походженням ці фактори можна поділити на:
  \begin{itemize}
    \item Фізичні.
    \item Хімічні.
      Їх поділяють на:
      \begin{itemize}
        \item токсичні,
        \item задушуючі,
        \item наркотичні,
        \item подразнюючи,
        \item сенсибілізуючи,
        \item канцерогенні,
        \item мутагенні,
        \item такі, що впливають на репродуктивну функцію.
      \end{itemize}

      По шляхам проникнення їх поділяють на:
      \begin{itemize}
        \item органи дихання,
        \item шлунково"=кишечний тракт,
        \item шкіряні покриви та слизові оболонки.
      \end{itemize}
    \item Біологічні "--- патогенні мікроорганізми та продукти їх
      життєдіяльності, а також макроорганізми.
    \item Психо"=фізіологічні.
      Їх поділяють на:
      \begin{itemize}
        \item фізичні перенавантаження,
        \item нервово-психічні перенавантаження,
        \item емоційні перенавантаження,
        \item монотонні перенавантаження.
      \end{itemize}
    \item Cоціальні:
      \begin{itemize}
        \item неякісна організація роботи,
        \item понаднормовона работа,
        \item погані відносини в колективі,
        \item зміни біоритмів,
        \item насильство.
      \end{itemize}
  \end{itemize}

  Охорона праці --- система правових, соціально-економічних,
  організаційно-технічних, санітарно-гігієнічних і лікувально-профілактичних
  заходів і засобів спрямованих на збереження життя, здоров’я та
  працездатності людини в процесі трудової діяльності.

  Структура до модулю:
  \begin{itemize}
    \item правові та організаційні основи,
    \item фізіологія, гігієна праці та виробнича санітарія,
    \item виробнича безпека,
    \item пожежна безпека на виробництві.
  \end{itemize}

\section{Правові та організаційні основи охорони праці}
  \begin{itemize}
    \item Закон України про охорону праці.
      Був введений в 1992 року.
      Має 8 розділів та 48 статей.
      \begin{itemize}
        \item Гарантії прав на охорону праці (розділ 2, статті 5--12).
        \item Організація охорони праці (розділ 3, статті 13--24).
        \item Стимулювання охорони праці (розділ 4, статті 25--26).
        \item Нормативно-правові акти з охорони праці (розділ 5, статті
          27--30).
        \item Державне управління охороною праці (розділ 6, статті 31--37).
        \item Державний нагляд і громадський контроль за охороною праці
          (розділ 7, статті 38--42).
        \item Відповідальність працівників за порушення законодавства про
          охорону праці (розділ 8, статті 43--44).
        \item Система стандартів безпеки праці.
      \end{itemize}
    \item Конституція України.
      \begin{description}
        \item[43] Право на належні безпечні умови праці.
        \item[46] Соціальних захист, що включає право забезпечення у разі
          повної, часткової або тимчасової втрати працездатності.
        \item[49] Право на охорону здоров’я, медичну допомогу та медичне
          страхування.
        \item[57] Право знати свої права.
      \end{description}
    \item Кодекс законів про працю.
    \item Закон України про пожежну безпеку.
    \item Закон о радіаційній безпеки України.
    \item Закон України про загально"=обов’язкове державне соціальне страхування
      від нещасного випадку на виробництві та професійного захворювання, які
      спричинили втрату працездатності.
  \end{itemize}

  Закон про охорону праці за порушення норм охорони праці передбачає:
  \begin{itemize}
    \item Адміністративну відповідальність --- відповідно статті 41 Кодексу
      України тягне за собою адміністративну відповідальність у вигляді
      накладання штрафів на працівників та посадових осіб підприємства.
    \item Дисциплінарна --- відповідно статті 47 КЗпПУ\footnote{Кодекс Законів
      про Працю України} встановлює два види дисциплінарного стягнення: догана
      та звільнення з роботи.
    \item Матеріальна відповідальність --- регламентуються КЗпПУ.
      Загальними підставами для накладання матеріальної відповідальності на
      працівника є наявність прямої дійсної шкоди, провина працівника,
      протиправні дії працівника, \ldots
    \item Кримінальна відповідальність --- ККУ\footnote{Кримінальний Кодекс
      України} 271--275.
      Якщо була загроза загибелі людей або інших тяжких наслідків.
  \end{itemize}

\section{Організація охорони праці на підприємстві}
  Три центри управління охороною праці:
  \begin{itemize}
    \item Державне управління.
    \item Управління з боку власника підприємства.
    \item Управління з боку працівника підприємства.
  \end{itemize}

  СУОП\footnote{Система управління охороною праці} має наступні основні
  функції:
  \begin{itemize}
    \item прогнозування та планування робіт, їх фінансування;
    \item організація та координація робіт (розробка стандарту СУОП);
    \item аналіз та оцінка стану умов і безпеки праці;
    \item стимулювання робіт по вдосконалення охорони праці.
  \end{itemize}

  Основні завдання управління охороною праці:
  \begin{itemize}
    \item навчання питань з охорони праці;
    \item забезпечення безпечності технологічних процесів, виробничого
      устаткування, будівель і споруд;
    \item забезпечення працівників засобами індивідуального захисту.
  \end{itemize}

  Основною формую планування є розроблення комплексного плану підприємства
  щодо покращення стану охорони праці.

  До основних форм контролю за станом охорони праці належать:
  \begin{itemize}
    \item оперативний контроль;
    \item контроль, що проводиться службою охорони праці підприємства;
    \item громадській контроль;
    \item адміністративно"=громадській трьох"=ступеневий контроль;
    \item відомчий контроль.
  \end{itemize}

  Відповідно до типового положення на підприємстві з числом працюючих 50 і
  більше чоловік створюється служба охорони праці.

  Служба охорони праці підпорядковується безпосередньо керівнику підприємства.

  Служба охорони праці виконує такі основні функції:
  \begin{itemize}
    \item опрацьовує ефективну цілісну систему управління охороною праці;
    \item проводить оперативно"=методичне керівництво роботи з охорони праці;
    \item складає разом зі структурним підрозділом підприємства комплексні
      заходи щодо до досягнення встановлених норм;
    \item проводить для працівників вступний інструктаж з охорони праці;
    \item організує забезпечення правилами, нормами, стандартами всі
      структурні підрозділи.
  \end{itemize}

\section{Навчання з питань охорони праці}
  Навчання і перевірка знань з охорони праці відповідно до ДНАОП000-4.12-99
  --- перелік осіб та посад що зобов’язані проходити перевірку знань з охорони
  праці.

  Інструктажі з охорони праці:
  \begin{itemize}
    \item Вступний інструктаж.
      Проводиться з:
      \begin{itemize}
        \item з усіма працівниками, що приймаються на постійну чи тимчасову
          роботу;
        \item з працівниками інших організацій, які прибули на підприємство та
          беруть участь у виробничому процесі або виконують інші роботи для
          підприємства;
        \item з учнями та студентами, які прибули на підприємства для
          проходження практики;
        \item з відвідувачами у разі екскурсії на підприємстві;
        \item з усіма вихованцями, учнями, студентами\ldots при оформленні до
          закладу освіти.
      \end{itemize}
    \item Первинний інструктаж.
      Проводиться безпосередньо на робочому місці до початку роботи з:
      \begin{itemize}
        \item новоприйнятим на підприємство;
        \item працівником, що переводиться з одного цеху до іншого;
        \item працівником, що буде виконувати нову для нього роботу;
        \item відрядженим працівником, що бере безпосередню участь у
          виробничому процесі цього підприємства.
      \end{itemize}
    \item Повторний інструктаж.
      Проводиться за працівниками на роботах з
      \begin{itemize}
        \item підвищеною небезпекою один раз на три місяці,
        \item для решти працівників --- один раз на шість місяців.
      \end{itemize}
    \item Позаплановий інструктаж.
      \begin{itemize}
        \item при введені в дію нових нормативних актів,
        \item при зміні технологічних процесів.
      \end{itemize}
    \item Цільовий інструктаж.
      Проводиться з працівниками
      \begin{itemize}
        \item при виконанні разових робіт, не передбачених трудовою угодою,
        \item при ліквідації аварії чи стихійного лиха,
        \item при проведенні робіт, на які оформлюються наряд допусків.
      \end{itemize}
  \end{itemize}
\end{document}
