% vim:spelllang=ru,en
\documentclass[a4paper,12pt,notitlepage,headsepline,pdftex]{scrartcl}

\usepackage{cmap} % чтобы работал поиск по PDF
\usepackage[T2A]{fontenc}
\usepackage[utf8]{inputenc}
\usepackage[english,russian]{babel}
\usepackage{concrete}
\usepackage{cite}
\usepackage{url}
\usepackage{fullpage}

\usepackage{textcase}
\usepackage[pdftex]{graphicx}

\usepackage{lscape}

\pdfcompresslevel=9 % сжимать PDF
\usepackage{pdflscape} % для возможности альбомного размещения некоторых страниц
\usepackage[pdftex]{hyperref}
% настройка ссылок в оглавлении для pdf формата
\hypersetup{unicode=true,
            pdftitle={PostgreSQL},
            pdfauthor={Погода Михаил},
            pdfcreator={pdflatex},
            pdfsubject={},
            pdfborder    = {0 0 0},
            bookmarksopen,
            bookmarksnumbered,
            bookmarksopenlevel = 2,
            pdfkeywords={},
            colorlinks=true, % установка цвета ссылок в оглавлении
            citecolor=black,
            filecolor=black,
            linkcolor=black,
            urlcolor=blue}

\usepackage{amsmath}
\usepackage{amssymb}
\usepackage{moreverb}
\usepackage{indentfirst}
\usepackage{misccorr}

\usepackage{xtab}
\usepackage{nccfoots}

\begin{document}
  \begin{table}
    \centering
    \begin{tabular}{|c|p{0.3\textwidth}|c|p{0.47\textwidth}|}
      \hline
      \bf\No &\bf Описание события &\bf Тип события &\bf Реакция на событие\\
      \hline\hline
      1 & Клерк оценивает популярность товара & N & Предоставить клерку
      выборку из таблиц \texttt{History of browsing}, \texttt{Books},
      \texttt{History of Purchasing}\\
      \hline
      2 & Клерк добавляет bundle & N & Предоставить клерку форму для
      ввода названия bundle и индивидуальных скидок для каждой книги в его
      составе, произвести соответствующие добавления в таблицы \texttt{Bundle}
      и \texttt{BundledBook}\\
      \hline
      3 & Клерк удаляет книгу/bundle & NN & Обратиться к администратору
      ИС\\
      \hline
      4 & Клерк изменяет описание книги/bundle & NN & Обратиться к
      администратору ИС\\
      \hline
      5 & Клерк составляет запрос на пополнения запаса книг & N & Предоставить
      форму для выбора необходимого количества, произвести соответствующие
      изменения в таблице \texttt{Request}\\
      \hline
      6 & Клерк оценивает задолженности клиентов & N & Предоставить клерку
      выборку из таблиц \texttt{Customer}, \texttt{RentedBook}\\
      \hline
    \end{tabular}
    \caption{ELM}
    \label{tab:elm}
  \end{table}
\end{document}
