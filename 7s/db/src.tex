\documentclass[a4paper,12pt,notitlepage,pdftex,headsepline]{scrartcl}

\usepackage{a4wide}
\usepackage{cmap} % чтобы работал поиск по PDF
\usepackage[T2A]{fontenc}
\usepackage[utf8]{inputenc}
\usepackage[russian]{babel}

\usepackage{textcase}
\usepackage[pdftex]{graphicx}

\usepackage{lscape}

\pdfcompresslevel=9 % сжимать PDF
\usepackage{pdflscape} % для возможности альбомного размещения некоторых страниц
\usepackage[pdftex]{hyperref}
% настройка ссылок в оглавлении для pdf формата
\hypersetup{unicode=true,
            pdftitle={Expert information systems},
            pdfauthor={Погода Михаил},
            pdfcreator={pdflatex},
            pdfsubject={},
            pdfborder    = {0 0 0},
            bookmarksopen,
            bookmarksnumbered,
            bookmarksopenlevel = 2,
            pdfkeywords={},
            colorlinks=true, % установка цвета ссылок в оглавлении
            citecolor=black,
            filecolor=black,
            linkcolor=black,
            urlcolor=blue}

\usepackage{amsmath}
\usepackage{amssymb}
\usepackage{moreverb}
\usepackage{indentfirst}
\usepackage{misccorr}
%for \includepdf
%\usepackage{pdfpages}

\begin{document}
\begin{titlepage}
  \begin{center}
    \large
    \MakeUppercase{Министерство образования и науки,}

    \MakeUppercase{молодёжи и спорта Украины}

    \MakeUppercase{Национальный технический университет Украины}

    \MakeUppercase{,,Киевский политехнический институт''}

    \addvspace{6pt}

    \normalsize
    Кафедра прикладной математики

    \vfill

    \textbf{Реферат}

    по дисциплине ,,Базы данных и информационные системы''

    на тему: \textit{,,Expert Information System''}
  \end{center}

  \vfill

  Выполнил\\
  студент группы КМ-92\\
  Погода~М.\,В.\\
  \vfill

  \begin{center}
    КИЕВ

    2012
  \end{center}
\end{titlepage}

\tableofcontents
\clearpage

\section{Предыстория}
  Экспертные системы появились благодаря стараниям исследователей в
  \textbf{Stanford Heuristic Programming Project}, а именно --- появления
  ,,праотца экспертных систем'' --- системы Дендрал и Мисин\footnote{Dendral
  and Mycin systems}.
  Большой вклад в эту технологию внесли Bruce Buchanan, Edward Shortliffe,
  Randall Davis, William vanMelle, Carli Scott и другие исследователи из
  Стэнфорда.
  Экспертные системы были одним из первых успешных программных продуктов,
  использующих искусственный интеллект.

  Исследования также активно велись во Франции, где велись работы по
  автоматизации рассуждений.
  Язык Prolog,  появившейся в 1972 году, был свидетельством громадного шага
  вперёд относительно таких экспертных систем, как Dendral или Mycin: он
  являлся программной оболочкой, готовой принять и запустить любую экспертную
  систему.
  Prolog работал используя логику первого порядка над множеством правил и
  фактов.
  Он, по сути, являлся средством для массового производства экспертных систем
  и был первым успешным декларативным языком программирования, позже став
  самым коммерчески-успешным языком программирования для искусственного
  интеллекта.
  Однако, Prolog не особо удобен и, к тому же, использует отличную от
  человеческой логику.

  В 1980х годах экспертные системы стали широко известны как инструмент для
  решения реальных задач.
  Университеты предлагали курсы по ЭС, а \( \frac23 \) компаний из
  \textbf{,,Fortune 1000''}\footnote{Список самых крупных компаний по версии
  американского журнала Fortune} использовали их в повседневной деятельности.
  Интерес к ним возрос после Японского проекта \textit{Fifth Generation
  Computer Systems}, в связи с чем возросли дотации на исследования экспертных систем в
  Европе.

  В связи с развитием информационных технологий, а также с увеличением объёмов
  этой самой информации, экспертные системы в той или иной степени
  интегрируются с информационными системами\footnote{Information systems}.
  Современные ЭС не обходятся без обширной базы знаний\footnote{Knowledge
  base}.

\section{Применение EIS}
  Экспертно-информационная система --- это набор программ или программное
  обеспечение, которое выполняет функции эксперта при решении какой-либо
  задачи в области его компетенции.
  EIS, как и эксперт-человек, в процессе своей работы оперирует со знаниями.
  Знания о предметной области, необходимые для работы EIS, определенным образом
  формализованы и представлены в памяти ЭВМ в виде базы знаний, которая может
  изменяться и дополняться в процессе развития системы.

  EIS выдают советы, проводят анализ, выполняют классификацию, дают
  консультации и ставят диагноз.
  Они ориентированы на решение задач, обычно требующих проведения экспертизы
  человеком-специалистом.
  В отличие от машинных программ, использующий процедурный анализ, EIS решают
  задачи в узкой предметной области (конкретной области экспертизы)на основе
  дедуктивных рассуждений.
  Такие системы часто оказываются способными найти решение задач, которые
  неструктурированны и плохо определены.
  Они справляются с отсутствием структурированности путем привлечения
  эвристик, т.\,е. правил, взятых ,,с потолка'', что может быть полезным в тех
  системах, когда недостаток необходимых знаний или времени исключает
  возможность проведения полного анализа.

  Главное достоинство EIS --- возможность накапливать знания, сохранять их
  длительное время, обновлять и тем самым обеспечивать относительную
  независимость конкретной организации  от наличия в ней квалифицированных
  специалистов.
  Накопление знаний позволяет повышать квалификацию специалистов, работающих
  на предприятии, используя наилучшие, проверенные решения.

\section{Структура ЭИС}
  \begin{itemize}
    \item Интерфейс пользователя.
    \item Пользователь.
    \item Интеллектуальный редактор базы знаний.
    \item Эксперт.
    \item Инженер по знаниям.
    \item Рабочая (оперативная) память.
    \item База знаний.
    \item Решатель (механизм вывода).
    \item Подсистема объяснений.
  \end{itemize}

  \textit{База знаний} состоит из правил анализа информации от пользователя по
  конкретной проблеме.
  ЭИС анализирует ситуацию и,в зависимости от направленности ЭИС, даёт
  рекомендации по разрешению проблемы.

  Как правило, база знаний экспертной системы содержит факты (статические
  сведения о предметной области) и правила --- набор инструкций, применяя
  которые к известным фактам можно получать новые факты.

  В рамках логической модели баз данных и базы знаний записываются на языке
  Пролог с помощью языка предикатов для описания фактов и правил логического
  вывода, выражающих правила определения понятий, для описания обобщенных и
  конкретных сведений, а также конкретных и обобщенных запросов к базам данных
  и базам знаний.

  Конкретные и обобщенные запросы к базам знаний на языке Пролог записываются
  с помощью языка предикатов, выражающих правила логического вывода и
  определения понятий над процедурами логического вывода, имеющихся в базе
  знаний, выражающих обобщенные и конкретные сведения и знания в выбранной
  предметной области деятельности и сфере знаний.

  Обычно факты в базе знаний описывают те явления, которые являются
  постоянными для данной предметной области.
  Характеристики, значения которых зависят от условий конкретной задачи, ЭИС
  получает от пользователя в процессе работы, и сохраняет их в рабочей памяти.
  Например, в медицинской ЭИС факт ,,У здорового человека 2 ноги'' хранится в
  базе знаний, а факт ,,У пациента одна нога'' --- в рабочей памяти.

  База знаний ЭИС создается при помощи трех групп людей:
  \begin{itemize}
    \item эксперты той проблемной области, к которой относятся задачи,решаемые
      ЭИС;
    \item инженеры по знаниям, являющиеся специалистами по разработке ЭИС;
    \item программисты, осуществляющие реализацию ЭИС.
  \end{itemize}

  ЭИС может функционировать в двух режимах:
  \begin{itemize}
    \item Режим ввода знаний.
      В этом режиме эксперт с помощью инженера по знаниям посредством
      редактора базы знаний вводит известные ему сведения о предметной области
      в базу знаний ЭИС.
    \item Режим консультации.
      Пользователь ведет диалог с ЭИС, сообщая ей сведения о текущей задаче и
      получая рекомендации ЭИС.
      Например, на основе сведений о физическом состоянии больного ЭИС ставит
      диагноз в виде перечня заболеваний, наиболее вероятных при данных
      симптомах.
  \end{itemize}
  \clearpage

\section{Классификации ЭИС}
  \subsection{Классификация ЭИС по решаемой задаче}
    \begin{itemize}
      \item Интерпретация данных.
      \item Диагностирование.
      \item Мониторинг.
      \item Проектирование.
      \item Прогнозирование.
      \item Сводное планирование.
      \item Обучение.
      \item Управление.
      \item Ремонт.
      \item Отладка.
    \end{itemize}
  \subsection{Классификация ЭИС по связи с реальным временем}
    \begin{itemize}
      \item Статические ЭИС --- ЭИС, решающие задачи в условиях не
        изменяющихся во времени исходных данных и знаний.
      \item Квазидинамические ЭИС, которые интерпретируют ситуацию, которая
        меняется с некоторым фиксированным интервалом времени.
      \item Динамические ЭИС --- это ЭИС, решающие задачи в условиях
        изменяющихся во времени исходных данных и знаний.
    \end{itemize}
  \clearpage

\section{Этапы разработки ЭИС}
  \begin{enumerate}
    \item Этап идентификации проблем --- определяются задачи, которые подлежат
      решению, выявляются цели разработки, определяются эксперты и типы
      пользователей.
    \item Этап извлечения знаний --- проводится содержательный анализ проблемной
      области, выявляются используемые понятия и их взаимосвязи, определяются
      методы решения задач.
    \item Этап структурирования знаний --- выбираются ИС и определяются
      способы представления всех видов знаний, формализуются основные понятия,
      определяются способы интерпретации знаний, моделируется работа системы,
      оценивается адекватность целям системы зафиксированных понятий, методов
      решений, средств представления и манипулирования знаниями.
    \item Этап формализации --- осуществляется наполнение экспертом базы
      знаний.
      В связи с тем, что основой ЭС являются знания, данный этап является
      наиболее важным и наиболее трудоемким этапом разработки ЭИС.
      Процесс приобретения знаний разделяют на
      \begin{enumerate}
        \item извлечение знаний из эксперта,
        \item организацию знаний, обеспечивающую эффективную работу системы,
        \item и представление знаний в виде, понятном ЭС.
      \end{enumerate}
      Процесс приобретения знаний осуществляется инженером по знаниям на
      основе анализа деятельности эксперта по решению реальных задач.
    \item Реализация ЭИС --- создается один или несколько прототипов ЭИС,
      решающие требуемые задачи.
    \item Этап тестирования --- производится оценка выбранного способа
      представления знаний в ЭИС в целом.
  \end{enumerate}
  \clearpage

\section{Наиболее известные ЭИС}
  \begin{itemize}
    \item WolframAlpha --- поисковая система, интеллектуальный
      ,,вычислительный движок знаний''.
    \item MYCIN --- наиболее известная диагностическая система, которая
      предназначена для диагностики и наблюдения за состоянием больного при
      менингите и бактериальных инфекциях.
    \item HASP/SIAP --- интерпретирующая система, которая определяет
      местоположение и типы судов в Тихом океане по данным акустических систем
      слежения.
    \item Akinator --- интернет-игра.
      Игрок должен загадать любого персонажа, а Акинатор должен его отгадать,
      задавая вопросы.
      База знаний автоматически пополняется, поэтому программа может отгадать
      практически любого известного персонажа.
  \end{itemize}
\end{document}
