\documentclass[a4paper,10pt,notitlepage,pdftex,headsepline]{scrartcl}

\usepackage{geometry}
\geometry{left=2cm}
\geometry{right=2cm}
\geometry{top=1cm}
\geometry{bottom=1cm}

\usepackage{cmap} % чтобы работал поиск по PDF
\usepackage[utf8]{inputenc}
\usepackage[russian]{babel}
\usepackage[T2A]{fontenc}

\usepackage{textcase}
\usepackage[pdftex]{graphicx}

\pdfcompresslevel=9 % сжимать PDF
\usepackage{pdflscape} % для возможности альбомного размещения некоторых страниц
\usepackage[pdftex]{hyperref}
% настройка ссылок в оглавлении для pdf формата
\hypersetup{
	unicode=true,
    pdftitle={},
    pdfauthor={Погода Михаил},
    pdfcreator={pdflatex},
    pdfsubject={},
    pdfborder    = {0 0 0},
    bookmarksopen,
    bookmarksnumbered,
    bookmarksopenlevel = 2,
    pdfkeywords={},
    colorlinks=true, % установка цвета ссылок в оглавлении
    citecolor=black, 
    filecolor=black,
    linkcolor=black,
    urlcolor=blue
}

\usepackage{amsmath}
\usepackage{amssymb}
\usepackage{moreverb}
\author{Michael Pogoda}
\title{plab35}
\date{\today}

\begin{document}
\begin{enumerate}
\item \textit{Як можна уявити світлову хвилю?
Основні характеристики монохроматичної хвилі.}

Світлова хвиля є складним явищем: в одних випадках він веде себе як електромагнітна хвиля, в інших --- як потік особливих частинок (фотонів).
Хвиля --- процес поширення коливань в просторі.

$A\cos\left(\omega t - kx + \alpha\right)$ --- рівняння світлової хвилі.

Основні характеристики монохромної хвилі:
\begin{itemize}
\item амплітуда;
\item фаза;
\item частота;
\item хвильове число;
\item напрямок поширення хвилі.
\end{itemize}
\item \textit{Яке світло називається природним, поляризованим? Чи може бути поляризованою повздовжня хвиля?}

В поляризованому світлі коливання якимось чином впорядковані.
В природному світлі коливання в кожний момент часу відбуваються в найрізноманітніших напрямках, хаотично.

Повздовжня хвиля не може бути поляризованою.
\item \textit{Які види поляризації світла ви знаєте?
Що таке площина коливань?}

Види поляризації світла:
\begin{itemize}
\item лінійна;
\item кругова;
\item еліптична.
\end{itemize}

Площина, в якій коливається світловий вектор (вектор напруженості електричного поля) називається площиною коливань.
\item \textit{Які ви знаєте поляризаційні пристрої? Що таке площина поляризатора?}

До першої з двох категорій, на які розділяють поляризаційні прилади, відносяться прості пристрої для здобуття і перетворення поляризованого світла:
\begin{itemize}
\item лінійні і циркулярні поляризатори;
\item фазові пластинки;
\item компенсатори оптичні
\item деполяризатори.
\end{itemize}
Друга категорія --- складніші конструкції і установки для кількісних поляризаційно-оптичних досліджень.
Як елементи в них входять поляризаційні прилади першої категорії, а також приймачі світла, монохроматори, допоміжні електронні пристрої і багато інших.

По історичним причинам площиною поляризації була названа не площина, в котрій коливається $Е$, а перпендикулярна до неї площина.
\item \textit{Яке світло називають частково-поляризованим?}

Частково-поляризоване світло --- це світло, в якому коливання одного напрямку домінують над коливаннями інших напрямків.
\item \textit{Ступінь поляризації світла? Який смисл мають $I_{max}$ і $I_{min}$?}

Якщо пропускати частково-поляризоване світло через поляризатор, то при повороті приладу навколо променя, інтенсивність пройденого світла буде змінюватись від $I_{max}$ до $I_{min}$.
До того ж перехід від одного до іншого буде здійснений при повороті на кут $\varphi = \frac{\pi}{2}$.

Ступінь поляризації:
$$P = \frac{I_{max} - I_{min}}{I_{max} + I_{min}}$$
\item \textit{Особливості проходження поляризованого світла крізь поляризатор. Закон Малюса.}

Коливання амплітуди $А$, що здійснюється у площині, яка утворює кут $\varphi$ з площиною поляризатора, можна розкласти на два коливання з амплітудами $A_\parallel = A\cos\varphi$ и $A_\perp = A\sin\varphi$.
Перше коливання пройде через прилад, друге --- буде затримане.
Інтенсивність пройденої хвилі $A_\parallel^2=A^2\cos^2\varphi$, тобто дорівнює
$I\cos^2\varphi$, де $I$ --- інтенсивність коливань з амплітудою $А$.
Отже, коливання, паралельні площині поляризатора, несе з собою частину інтенсивності, що дорівнює $\cos^2\varphi$.
В природному світлі всі значення $\varphi$ рівно можливі.
Тому доля світла, що пройшло через поляризатор, буде дорівнювати середньому значенню $\cos^2\varphi$, тобто $\frac12$.
При повороті поляризатора навколо напрямку природного променя інтенсивно пройденого світла залишиться тією ж, зміниться лише орієнтація площини коливань світла вихідного приладу.

Нехай на поляризатор падає плоско-поляризоване світло температури $A_0$ і інтенсивності $I_0$.
Крізь прилад пройде світло, яке складає коливання з амплітудою $A = A_0 \cos\varphi$, де $\varphi$ --- кут між площиною коливань падаючого світла і площиною поляризатора.
Відповідно інтенсивність пройденого світла: $I = I_0\cos^2\varphi$ --- закон Малюса.
\item \textit{Що таке $E_\parallel$ і $E_\perp$? Звідки випливають формули Френеля?}

Падаюча хвиля представлена у вигляді у вигляді суперпозиції двох хвиль $E_{\parallel 0}$ і $E_{\perp 0}$, електричні вектори котрих коливаються в площині падіння хвиль і перпендикулярно до неї.
Залежність зображеної та заломленої хвиль від кута падіння описуються формулами Френеля.
Так, наприклад, амплітуди зображених хвиль $E_\parallel$ і $E_\perp$ згідно цим формулам:
$$E_\parallel = E_{\parallel 0} \frac{\tg{\left(\Theta_1 - \Theta_2\right)}}{\tg{\left(\Theta_1 + \Theta+2\right)}}, E_\perp = E_{\perp 0}\frac{\sin{\left(\Theta_1 - \Theta_2\right)}}{\sin{\left(\Theta_1 - \Theta_2\right)}}$$
і мають різну залежність від кута падіння.
\item \textit{Формули Френеля для відбитих і заломлених хвиль.}

$n_1$ і $n_2$ --- абсолютні показники заломлення.
$\Theta_1$ і $\Theta_2$ --- кути падіння і заломлення.

З формули Френеля $\Theta_1 + \Theta_2 = \frac{\pi}{2}$, звідки амплітуда відбитої хвилі $E\parallel = 0$, а відбите світло містить лише компонент $E_\perp$, тобто повністю поляризований.
\item \textit{Закон Брюстера. Його пояснення з точки зору електронної теорії.}

Якщо кут падіння світла на границю розподілу двох діелектриків ненульовий, відбитий і заломлений промені опиняються частково поляризованими.
У відбитому промені переважають коливання, перпендикулярні до площини падіння, в заломленому промені --- коливання паралельні площині падіння.
Степінь поляризації залежить від кута падіння.
При куті падіння, задовольняючому умові $\tg\Theta_B = n_{12}$ (1), відбитий промінь повністю поляризований.
Степінь поляризації заломленого променя при куті падіння $\Theta_B$ досягає найбільшого значення, тим не менше цей промінь залишається поляризованим тільки частково.

Співвідношення (1) --- закон Брюстера.
$\Theta_B$ --- кут Брюстера або кут повної поляризації.
\item \textit{Що таке звичайна та незвичайна хвиля? Покажіть площини їх коливань.}

При проходженні світла через деякий кристал світловий промінь розділяється на 2 промені --- подвійне променеве заломлення.
Один з променів задовольняє звичайному закону заломлення і лежить в одній площині з падаючим променем і нормаллю.
Цей промінь \textbf{звичайний} і позначається $O$.
Для іншого променя, \textbf{незвичайного}, $\frac{\sin{i_1}}{\sin{i_2}}$ не залишається постійним при зміні кута падіння.
Незвичайний промінь не лежить в одній площині з падаючим променем і нормаллю до заломлюючої площини.
\item \textit{Сформулюйте принцип роботи оптичного квантового генератора.}

Основна ідея роботи полягає в інверсії електронної населеності шляхом ,,накачки'' робочого тіла енергією імпульсів.
Робоче тіло поміщається в оптичний резонатор, при циркуляції хвилі в котрому її енергія експоненціально зростає завдяки механізму вимушеного випромінювання.
\item \textit{Поясніть будову та принцип дії He-Ne лазера.}

He-Ne лазер використовується в якості джерела поляризованого світла.
Він складається з джерела живлення, газорозрядної трубки і дзеркал резонатора.
Довжина хвилі лазерного випромінювання $D$ дорівнює $0.63$ (мкм), потужність приблизно дорівнює $1$ МВт.
\end{enumerate}
\end{document}