\documentclass[a4paper,10pt,notitlepage,pdftex,headsepline]{scrartcl}

\usepackage{a4wide}
\usepackage{cmap} % чтобы работал поиск по PDF
\usepackage[utf8]{inputenc}
\usepackage[russian]{babel}
\usepackage[T2A]{fontenc}

\usepackage{textcase}
\usepackage[pdftex]{graphicx}

\pdfcompresslevel=9 % сжимать PDF
\usepackage{pdflscape} % для возможности альбомного размещения некоторых страниц
\usepackage[pdftex]{hyperref}
% настройка ссылок в оглавлении для pdf формата
\hypersetup{unicode=true,
            pdftitle={},
            pdfauthor={Погода Михаил},
            pdfcreator={pdflatex},
            pdfsubject={},
            pdfborder    = {0 0 0},
            bookmarksopen,
            bookmarksnumbered,
            bookmarksopenlevel = 2,
            pdfkeywords={},
            colorlinks=true, % установка цвета ссылок в оглавлении
            citecolor=black,
            filecolor=black,
            linkcolor=black,
            urlcolor=blue}

\usepackage{amsmath}
\usepackage{amssymb}
\usepackage{moreverb}
%for \includepdf
%\usepackage{pdfpages}

\author{Michael Pogoda}
\title{4m}
\date{\today}
\begin{document}
\section*{Метод касательных}
Идея связана с методом простой итерации.

$$x = \varphi \left( x\right)$$

$$\varphi\left(x\right) = x - \frac{f\left(x\right)}{f^\prime\left(x\right)}$$

Ограничения на $x$ и $f$ те же.

\begin{align}
 & x_{n+1} = x_n - \frac{f\left(x_n\right)}{f^\prime\left(x_n\right)} & \label{baseformula}
\end{align}

\subsection*{Докажем сходимость метода:}
Вычислим $\varphi^\prime\left(x\right)$:

$$\varphi^\prime\left(x\right) = 1- \frac{f^\prime\left(x\right)f^\prime\left(x\right) - f\left(x\right)f^{\prime\prime}\left(x\right)}{\left(\strut f^\prime\left(x\right)\right)^2} = \frac{f^{\prime\prime}\left(x\right) f \left( x \right)}{\left(\strut f^\prime \left( x \right) \right)^2}
\Rightarrow \varphi^\prime\left(\bar{x}\right) = 0$$

$$\left|\varphi^\prime\left(x\right)\right| \leqslant q < 1$$

Из сходимости метода итерации можно сделать вывод о сходимости данного метода.
\paragraph*{Теорема}
\textit{Если $f\left(a\right) f\left(b\right) < 0$, $f^\prime\left(x\right) \neq 0$, $f^{\prime\prime}\left(x\right) \neq 0$, и сохраняют определённые знаки на $\left[a; b\right]$, то по начальному приближению $x_0 \in \left[a;b\right]$, которое удовлетворяет $f^\prime\left(x_0\right) f^{\prime\prime}\left(x_0\right) > 0$, то можно вычислить методом касательных по функции (\ref{baseformula}) корень уравнения $f\left(x\right) = 0$ с любой точностью.}

Рекомендации:
\begin{enumerate}
\item Интервал $\left[a;b\right]$ выбирается так, чтобы $f^\prime\left(x\right)$ и $f^{\prime\prime}\left(x\right)$ сохраняли знак.
\item Начальное приближение $x_0$ можно взять любое, например какой-либо конец отрезка, но только так, чтобы $f^\prime\left(x\right) f^{\prime\prime}\left(x\right) > 0$.
\end{enumerate}
\subsection*{Оценка метода:}
Для метода касательных можно использовать формулу общей оценки для всех методов:
\begin{align}
 & \left|x_n - \bar{x}\right| = \frac{\left|f\left(x_n\right)\right|}{\left|f^\prime\left(\xi\right)\right|} \leqslant \frac{\left|f\left(x_n\right)\right|}{\min\limits_{\left[a; b\right]}\left| f^\prime\left(x\right)\right|} \leqslant \varepsilon & \label{formula2}
\end{align}

Оценка метода касательных по формуле Тейлора:

$$f\left(x_n\right) = \underbrace{f\left(x_{n-1}\right) + f^\prime\left(x_{n-1}\right)\left(x_n-x_{n-1}\right)}_{\text{равно нулю по (\ref{baseformula})}} + \frac{1}{2!}f^\prime\left(\xi_{n-1}\right)\left(x_n - x_{n-1}\right)^2,$$
где, $\xi_{n-1} \in \left[x_{n-1}; x_n\right]$.
$$\left|x_n - \bar{x}\right| \leqslant \frac{M_2}{2 m_1}\left(x_n - x_{n-1}\right)^2$$
$$M_2 = \max\limits_{\left[a;b\right]}\left|f^{\prime\prime}\left(x\right)\right|,          0 < m_1 < \min\limits_{\left[a;b\right]}\left|f^\prime\left(x\right)\right|$$

Формула оценки двух соседних приближений:
$$\left|\bar{x} - x_n\right| \leqslant \frac{M_2}{2 m_1} \left(\bar{x} - x_{n-1}\right)^2$$
Если $\frac{M_2}{2 m_1} \leqslant, \left|\bar{x} - x_{n-1}\right| < 10^{-m}$, то $\left|\bar{x} - x_n\right| < 10^{-2m}$.
На каждом шаге количество верных цифр уменьшается в 2 раза.

Метод быстро сходится, но много вычислений.
\subsection*{Случай близких корней}
Если существуют $x_1, x_2$ --- близкие корни, тогда смотрим на (\ref{baseformula}) $f\left(\bar{x_1}\right) = 0, f\left(\bar{x_2}\right) = 0$.

В данном случае касательная $\parallel 0x: \exists x_0\: f\left(x_0\right) = 0$.

По формуле Тейлора:
$$f( \bar{x_2} ) = f( x_0 ) + f^\prime( x_0 )( \bar{x_2} - x_0 ) + \frac{1}{2!}f^{\prime\prime}( \xi )( \bar{x_2} - x_0 )^2, \:\:\xi\in [x_0; \bar{x_2} ]$$

Предположив, что $\left|x_0- \bar{x_1}\right| = \left|x_0 - \bar{x_2}\right| = d$, мы можем из формулы Тейлора вывести значение для $d$:
$$d=\sqrt{\frac{-2f\left(x_0\right)}{f^{\prime\prime}\left(\xi\right)}}, \:\xi\in\left[x_0; \bar{x_2}\right]$$
$$x_{1,2} = x_0 \pm d$$
Если $\xi = x_0$, то можно найти $d$ и после этого построить начальное приближение для обоих корней.
\end{document}
