\documentclass[a4paper,10pt,notitlepage,pdftex,headsepline]{scrartcl}

\usepackage{geometry}
\geometry{left=2cm}
\geometry{right=2cm}
\geometry{top=1cm}
\geometry{bottom=1cm}

\usepackage{cmap} % чтобы работал поиск по PDF
\usepackage[utf8]{inputenc}
\usepackage[russian]{babel}
\usepackage[T2A]{fontenc}

\usepackage{textcase}
\usepackage[pdftex]{graphicx}

\pdfcompresslevel=9 % сжимать PDF
\usepackage{pdflscape} % для возможности альбомного размещения некоторых страниц
\usepackage[pdftex]{hyperref}
% настройка ссылок в оглавлении для pdf формата
\hypersetup{
	unicode=true,
    pdftitle={},
    pdfauthor={Погода Михаил},
    pdfcreator={pdflatex},
    pdfsubject={},
    pdfborder    = {0 0 0},
    bookmarksopen,
    bookmarksnumbered,
    bookmarksopenlevel = 2,
    pdfkeywords={},
    colorlinks=true, % установка цвета ссылок в оглавлении
    citecolor=black, 
    filecolor=black,
    linkcolor=black,
    urlcolor=blue
}

\usepackage{amsmath}
\usepackage{amssymb}
\usepackage{moreverb}
\author{Michael Pogoda}
\title{plab33}
\date{\today}

\begin{document}
\begin{enumerate}
\item \textit{Что такое дифракция света?}

Дифракция света --- явление, суть которого состоит в том, что свет способен огибать препятствия.
\item \textit{Существует ли принципиальное отличие между явлениями дифракции и интерференции?}

Нет. По принципу Гюйгенса-Френеля, дифракция является частичным случаем интерференции.
\item \textit{Сформулируйте принцип Гюйгенса-Френеля, приведите его аналитическую запись.}

Каждый элемент волнового фронта можно рассматривать как центр вторичного возбуждения, порождающего вторичные сферичные волны, а результирующее световое поле в каждой точке будет определятся интерференцией этих волн.
$$E = \int\limits_S\frac{a\,ds}re^{i(\omega t-ki)}$$
\item \textit{В отличие между дифракцией Фраунгофера и дифракцией Френеля?}

\textbf{Дифракция Фраунгофера} --- дифракционная картина, которая наблюдается на большом расстоянии от препятствия, которое огибает свет, а \textbf{дифракция Френеля} наблюдается на небольшом расстоянии от препятствия.
\item \textit{Получите условия минимумов и максимумов дифракционной картины.}

Условие максимума дифракционной картины:
$$\sin\left( \pi b \lambda^{-1}\sin\Theta\right) = \pm 1 \Leftrightarrow \pi b \lambda^{-1} \sin\Theta = \frac{2n + 1}{2}\pi, n\in \mathbb{Z}$$
$$b\sin\Theta = \frac{2n + 1}{2}\lambda \Rightarrow b\Theta = \frac{2n+1}{2}\lambda, n\in\mathbb{Z}$$
Условие минимума:
$$\sin\left(\pi b \lambda^{-1} \sin\Theta\right) = 0 \Rightarrow b \sin\Theta = n\lambda, n\in\mathbb{Z}\setminus\left\{0\right\}$$
$$b\Theta = n\lambda, n\in\mathbb{Z}\setminus\left\{0\right\}$$
\item \textit{В каких пределах должна находится ширина щели для наблюдения дифракции? Как изменится дифракционная картина, если изменять ширину щели от $b_{min}$ до $b_{max}$?}

Приблизительная оценка: $b - \lambda \leqslant b \leqslant 10^3 \lambda$, тогда для видимого света: $0.5 \leqslant b \leqslant 500$ (мкм).
При увеличении ширину дифракционной щели дифракционная картина будет сжиматься.
Её минимальная ширина определяется раздельной способностью глаза.
\item \textit{Запишите критерий разграничивания случаев дифракции Фраунгофера, Френеля и геометрической оптики. Дайте определение этих критериев на основании представлений про зоны Френеля.}

Рассматривают величину $f = \frac{\varphi^2}{z \lambda}$.
$\varphi$ --- угол направления на рассматриваемую точку; $z$ --- расстояние от щели$\backslash$ препятствия до экрана; $\lambda$ --- длина волны.

Дифракция Фраунгофера определяется условием $f \ll 1$, при этом при расчётах им пренебрегают.
Это означает, что на экране располагается линия части первой зоны Френеля.

Дифракция Френеля определяется условием $f \geqslant 1$, при этом на экране одна или даже несколько зон Френеля.
\item \textit{Что произойдёт с дифракционной картиной, если закрыть половину линзы, в фокальной плоскости которой расположен экран?}

Дифракционная картина не изменится, интенсивность света уменьшится.
\item \textit{Получите общее выражение и рассчитайте несколько значений отношений интенсивности побочных максимумов к главному.}

Для главного максимума: $I = I_0$

Для побочных максимумов: $\Theta_n = \frac{2n + 1}{2} \frac{\lambda}{z_0}, n\in \mathbb{Z}$
$$I_m = I_0 \left(
\frac{\sin\left(z b \lambda^{-1} \sin\left(\frac{2n + 1}{2} \frac{\lambda}{b}\right)\right)}
{\pi b \lambda^{-1} \sin\left(\frac{2n + 1}{2} \frac{\lambda}{b}\right)}\right)^2 = I_0
\left(\frac{\sin\left(\frac{2n + 1}{2}\pi\right)}
{\frac{2n + 1}{2}\pi}\right)^2 = I_0 \frac{1}{\left(\frac{2n + 1}{2}\pi\right)^2}$$
Тогда, $$\frac{I_n}{I} = \frac{1}{\left(\frac{2n + 1}{2}\pi\right)^2}$$
\item \textit{От чего зависит число видимых главных максимумов решётки?}

Число видимых главных максимумов решётки зависит от угла падения света на решётку, от размера решётки, количества штрихов и ширины щели.
\item \textit{Во сколько раз интенсивность главного максимума решётки больше чем интенсивность главного максимума одной её щели?}

Известно, что
$$I\left(\Theta\right) = I_0\left(\frac{\sin\alpha}{\alpha}\right)^2 \left(\frac{\sin\beta}{\beta}\right)^2$$
$$\alpha = \pi b \lambda^{-1} \sin\Theta, \beta = \pi d \lambda^{-1}\sin\Theta$$
$b$ --- ширина щели, $d$ --- период решётки.
При выполнении условий главного максимума $d\sin\Theta = m\lambda$.
Эта дробь на границе даёт отношение $\frac{1}{N^2}$.
Т.\,е. интенсивность в главном максимуме решётки в $N^2$ раз больше интенсивности главного максимума одной её щели.
\item \textit{Что такое интенсивность световой волны?}

Интенсивность --- величина, которая численно характеризует поток энергии, который переносится энергией в некотором направлении.
\item \textit{С помощью каких приборов и на основании каких эффектов измеряется интенсивность света в данной работе?
Откуда делается вывод про пропорциональность между интенсивностью света и показаниями прибора?}

Приборы для измерения интенсивности света: He--Ne лазер, дифракционная щель изменяемой ширины, в качестве экрана --- фотоприёмник.

Понятно, что используется принцип Гюйгенса-Френеля, на основании которого выведены расчётные формулы.
\end{enumerate}

\end{document}
