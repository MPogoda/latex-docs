\documentclass[a4paper,10pt,notitlepage,pdftex,headsepline]{scrartcl}

\usepackage{geometry}
\geometry{left=2cm}
\geometry{right=2cm}
\geometry{top=1cm}
\geometry{bottom=1cm}

\usepackage{cmap} % чтобы работал поиск по PDF
\usepackage[utf8]{inputenc}
\usepackage[russian]{babel}
\usepackage[T2A]{fontenc}

\usepackage{textcase}
\usepackage[pdftex]{graphicx}

\pdfcompresslevel=9 % сжимать PDF
\usepackage{pdflscape} % для возможности альбомного размещения некоторых страниц
\usepackage[pdftex]{hyperref}
% настройка ссылок в оглавлении для pdf формата
\hypersetup{
	unicode=true,
    pdftitle={},
    pdfauthor={Погода Михаил},
    pdfcreator={pdflatex},
    pdfsubject={},
    pdfborder    = {0 0 0},
    bookmarksopen,
    bookmarksnumbered,
    bookmarksopenlevel = 2,
    pdfkeywords={},
    colorlinks=true, % установка цвета ссылок в оглавлении
    citecolor=black, 
    filecolor=black,
    linkcolor=black,
    urlcolor=blue
}

\usepackage{amsmath}
\usepackage{amssymb}
\usepackage{moreverb}
\author{Michael Pogoda}
\title{plab31}
\date{\today}

\begin{document}
\begin{enumerate}
\item \textit{Що називається інтерференцією світла?}

Інтерференцією називається таке накладання хвиль, за якого результуюча інтенсивність не дорівнює сумі інтенсивностей хвиль, що приходять до точки накладання.
\item \textit{Які хвилі називаються когерентними? Чому світлові хвилі, що випромінюються незалежними джерелами, некогерентні?}

Когерентні хвилі --- це хвилі, що зберігають свої частотні, поляризаційні й фазові характеристики.
Умовою когерентності хвиль є незмінюваність у часі різниці між фазами коливань у них, що можливо лише тоді, коли хвилі мають однакову довжину (частоту).
\item \textit{Поясніть принцип отримання когерентних світлових хвиль та наведіть конкретні приклади (окрім біпризми Френеля).}

Для отримання двох когерентних між собою променів у оптиці використовують розділення початкового променя світла на два, а потім отримані хвилі сходяться в певній області простору, так званій області перекриття.
Для того, щоб виникла стійка інтерференційна картина, різниця ходу $\Delta$ цих хвиль до області перекриття не повинна бути більшою деякої характерної довжини, яка
називається довжиною когерентності.

Принцип Гюйгенса, метод Юнга.
\item \textit{Чи обов’язково буде спостерігатись інтерференція під час накладання когерентних хвиль у випадку: а) звукових хвиль; б) світлових хвиль?}

Інтерференція світла --- окремий випадок інтерференції хвиль.
Розрізняють стаціонарну інтерференцію хвиль, яка спостерігається при накладенні когерентних світлових хвиль, і нестаціонарну, при накладанні хвиль різної частоти. 

Інтерференціювати можуть лише хвилі з однаковою частотою.
\item \textit{Що називається оптичною та геометричною різницею ходу променів (хвиль)?}

Величина, що дорівнює різниці оптичних довжин шляхів, які проходяться хвилями, називається оптичною різницею ходу: $(n_2 r_2 - n_1 r_1)$.

Величина $(r_2 - r_1)$ називається геометричною різницею ходу.
\item \textit{Виведіть формулу (1.3). Запишіть вираз $\delta$ через довжину хвилі $\lambda'$ світлової хвилі в однорідному середовищі.}
$$A_1 \cos{\omega\left(t+\frac{s_1}{v_1}\right)}$$
$$A_2 \cos{\omega\left(t+\frac{s_2}{v_2}\right)}$$,
де $v_1 = \frac{c}{n_1}$, $v_2 = \frac{c}{n_2}$ --- фазові швидкості першої і другої хвиль.

Різниця фаз коливання:
$$\delta = \omega\left(\frac{s_2}{v_2} - \frac{s_1}{v_1}\right) = \frac{\omega}{c}\left(n_2 s_2 - n_1 s_1\right)$$
Виразив $\frac{\omega}{c}$ з рівності $2\pi\frac{\nu}{c}=\frac{2\pi}{\lambda_0}$, де $\lambda_0$ --- довжина хвилі в вакуумі, приведемо до вигляду
$$\delta = \frac{2\pi}{\lambda_0}\Delta$$, де $\Delta = n_2 s_2 - n_1 s_1 = L_2 - L-1$ --- оптична різниця ходу.
\item \textit{Виведіть умову (1.4).}
$$\Delta_{max} = k\lambda, \Delta_{min} = \left(k + \frac12\right)\lambda$$
З формули (1.3) видно, що якщо оптична різниця ходу $\Delta$ дорівнює цілому числу довжин хвиль у вакуумі: $\Delta = k\lambda, k\in\mathbb{Z}$, то різниця фаз $\delta$ є кратною $2 \pi$ і коливання, які викликані в точці $P$ двома хвилями, будуть проходити з однаковою фазою.
Отже, $\Delta = k\lambda, k\in\mathbb{Z}$ --- умова максимуму.

Якщо $\Delta$ дорівнює отриманому числу довжин хвиль в вакуумі:
$$\Delta = \left(k + \frac12\right)\lambda, k\in\mathbb{Z} \Rightarrow \delta = \left(2\pi k + \pi\right), k\in\mathbb{Z}$$,
так, що коливання в точці $P$ знаходяться в противофазі.
Таким чином, умова $\Delta = \left(k +\frac12\right)\lambda$ є умовою мінімуму.
\item \textit{Виведіть формули (1.5) і (1.6). Чому заломлюючі кути біпризми повинні бути дуже малими?}

$$s_2 - s_1 \approx 2l, n = 1$$
$\Delta$ дає оптичну різницю ходу $\Delta = \frac{x d}{l}$;
підставимо значення $\Delta$ в формулу (1.4) і отримаємо, що максимуми інтенсивності будуть спостерігатись при $$x_{max} = k\frac{l}{d}\lambda_0, k\in\mathbb{Z}$$
Для мінімуму $$x_{min} = \left(k+\frac12\right)\frac{l}{d}\lambda_0, k\in\mathbb{Z}$$

$\Delta x = \frac{l}{d}\lambda_0$ --- ширина інтерферентної полоси.
\end{enumerate}
\end{document}