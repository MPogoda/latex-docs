\documentclass[10pt,a4paper,titlepage]{article}

\usepackage[unicode,pdftex]{hyperref}
\hypersetup{pdfborder=3pt 3pt 2pt}

\usepackage{mathtext}
\usepackage{amsmath}
\usepackage{amssymb}

\usepackage{cmap}
\usepackage{longtable}

\usepackage{indentfirst}

\usepackage[T2A]{fontenc}
\DeclareSymbolFont{T2Aletters}{T2A}{ftm}{m}{n}

\usepackage[utf8]{inputenc}
\usepackage[russian]{babel}

\usepackage{geometry}
\geometry{top=2cm}
\geometry{left=1cm}
\geometry{right=1cm}

\author{Копычко Сергей Николаевич}
\title{Введение в компьютерные системы}
\date{\today}

\begin{document}

\maketitle

\newpage
\tableofcontents

\newpage
\section{Некоторые сведения из истории развития средств вычисления}

\textbf{1623 г.}, \textit{Шиккард (Германия)} --- 1-ая вычислительная машина.

\textbf{1641 г.}, \textit{Блез Паскаль} --- 1-ая счётная машина. Были реализованы операции сложения и вычитания.

\textbf{1673 г.}, \textit{Лейбниц} --- машина, исполняющая все арифметические операции (сложение, вычитание, умножение и деление).

\textbf{1821 г.}, \textit{Кельвин Томас} --- было налажено производство счётных машин ,,Томас'' по 100 штук в год.

\textbf{конец XVIII в.} --- машина Якобсена.

\textbf{1828 г.} --- счётные машины Слободского.

\textbf{1846 г.} --- Штоффель.

\textbf{1846 г.} --- счислитель Куммера.

\textbf{1868 г.} --- Буняковський.

\textbf{1878 г.} --- Чебышев.

\textbf{XIX в.} --- зубчатый арифмометр с переменным числом зубчаток.

\textbf{XIX в.}, \textit{Чарльз Беббидж} --- вычислительная машина, прообраз ПК --- ,,Чудачество Беббиджа''.

1-ая программа ,,Ада'' --- названа в честь внучки Байрона --- \textbf{Ады Лавейс}.

\paragraph{Блоки машины Беббиджа:}
\begin{enumerate}
\item Блок для хранения чисел

\end{enumerate}

\subsection{Периоды развития ЭВМ}

\subsubsection{Поколения ЭВМ с точки зрения инженера-электронщика (элементная база)}

\subsubsection{Поколения ЭВМ с точки зрения инженера-математика}

\newpage
\section{Этапы подготовки и решения задач на ЭЦВМ}

\newpage
\section{Алгоритм, его свойства и способы отображения}

\subsection{Типы вычислительных процессов}

\newpage
\section{Классификация ЭВМ по принципу действия}

\newpage
\section{Системы счисления}

\subsection{Позиционные системы счисления}

\subsection{Системы счисления, применяемые в ЭЦВМ}

\subsection{Арифметические операции в различных системах счисления}

\subsubsection{Правила сложения двух чисел в системе счисления с основанием $q$}

\subsubsection{Правила вычитания двух чисел в системе счисления с основанием $q$}

\subsection{Перевод чисел из одной системы счисления в другую}


\newpage
\section{Представление чисел в ЭЦВМ}

\newpage
\section{Форматы команд ЭЦВМ}

\subsection{Способы адресации}

\newpage
\section{Арифметические операции над нормализованными числами}

\subsection{Сложение (вычитание) двух нормализованных чисел:}

\begin{enumerate}
\item выравниваются порядки чисел (число, порядок которого меньше денормализуется так, чтобы его порядок стал равен порядку второго числа);
\item мантиссы обоих чисел складываются (вычитаются) по правилам сложения (вычитания) чисел с фиксированной запятой.
Сумма (разность) принимается за мантиссу суммы (разности), а порядок большего числа --- за порядок суммы (разности);
\item результат нормализуется.
\end{enumerate}

\subsection{Алгоритм умножения (деления) двух нормализованных чисел:}

\begin{enumerate}
\item умножаются (делятся) мантиссы.
Произведение (частное) мантисс принимается за мантиссу произведения (частного), а сумма (разность) порядков --- за порядок произведения (частного);
\item результат нормализуется.
\end{enumerate}

\subsection{Признаки переполнения разрядкой сетки:}

Обычно переполнение возникает при сложении чисел с одинаковыми знаками.
Признаком переполнения может служить противоположность знака слагаемых знаку полученной суммы, т.\,е., например, если знаки операндов --- единицы, а в знаке результата --- нуль.
Это можно фиксировать техническим устройством, и оно вырабатывает сигнал прерывания.
Из-за сложности такого устройства этот способ не применяют.
Намного проще обнаружить переполнение разрядной сетки, если использовать два разряда для кодирования знака числа.
При сложении чисел одного знака переполнение разрядной сетки приводит к несовпадению цифр в знаковых разрядах числа.
При этом комбинация $01$ указывает на то, что сумма положительная, а $10$ --- на то, что сумма отрицательная.
Сравнивает цифры знаковых разрядов специальное устройство, и при их несовпадении вырабатывает соответствующий сигнал управления.

Операция вычитания в ЭЦВМ реализуется достаточно сложно, т.\,к. при этом часто возникает необходимость заёма единиц в соседнем старшем разряде.
Если же уменьшаемое меньше вычитаемого, этот процесс будет продолжатся до самого старшего разряда, а затем вычитание нужно делать заново, поменяв знак разности и вычитаемое с уменьшаемым местами.
Техническая реализация операции сложения в ЭВМ значительно проще, чем операция вычитания.
По этой причине в ЭЦВМ применяют только схемы сложения с использованием специальных кодов.

\newpage
\section{Способы кодирования чисел}

Используются следующие коды чисел:

\begin{itemize}
\item прямой --- $+7 = 00000111$; $-7 = 11000111$; $0 = 11000000 = 00000000$. Старшие разряды --- знаковые, остальные --- цифровые.
\item обратный --- $+7 = 00000111$; $-7 = 11111000$; $0 = 00000000 = 11111111$ --- инвертированный прямой код.
\item дополнительный --- $+7 = 00000111$; $-7 = 11111001$; $0 = 00000000$ --- необходимо рассмотреть исходный код справа налево (т.\,е. с младших разрядов) и перенести неизменными в дополнительный код все нули и первую единичку.
Остальные биты прямого кода отрицательного числа, расположенные между этой первой единичкой и знаковыми разрядами, необходимо записать в дополнительный код инвертированными.
Аналогично можно получить прямой код отрицательного код по его дополнительному коду.
\end{itemize}

На практике чаще всего используются коды с одним или двумя знаковыми разрядами.
Коды с двумя знаковыми разрядами называются модифицированными.
Плюс кодируется 00, минус кодируется 11.

Прямой код используется в ЭЦВМ для хранения положительных и отрицательных чисел в памяти компьютера, а также для записи положительных чисел при выполнении арифметических операций.

Обратный и дополнительный коды (в том числе и модифицированные) позволяют операцию вычитания в ЭЦВМ заменить операцией сложения, что даёт возможность к сведению всех арифметических операций к выполнению операций сложения.
Модифицированные коды чисел используются для выявления переполнения разрядной сетки ЭЦВМ.

Числа, представленный в этой форме, в ЭЦВМ также записываются в виде указанных кодов, при этом порядок и мантисса рассматриваются как самостоятельные двоичные числа с фиксированной запятой.
Допускается представление мантисс и порядков чисел в одном и том же или разном кодах.

\newpage
\section{Выполнение арифметических операций в ЭЦВМ с фиксированной запятой и с плавающей запятой}

Операцию вычитания в ЭЦВМ в непосредственном виде не производят.
Её заменяют выполнением операции сложения с использованием обратного или дополнительного кода.

\subsection{Сложение чисел с фиксированной запятой}

Для этой цели должна быть выполнена следующая последовательность действий:

\begin{enumerate}
\item преобразование прямого кода исходных чисел в обратный или дополнительный;
\item поразрядное сложение кодов с учётом возможных переносов из разряда в разряд (справа налево);
\item преобразование результата в прямой код.
\end{enumerate}

При поразрядном сложении обратных кодов знаковые разряды складываются как и разряды цифровые (разряды мантисс).
Если образуется в результате сложения единица переполнения (переноса) из знакового разряда, то её прибавляют к младшему разряду суммы (эта операция носит название \textit{циклический перенос}).

При поразрядном сложении дополнительных кодов знаковые разряды складываются как разряды мантисс.
Образующаяся из знакового разряда единица переноса (переполнения) не учитывается (теряется).

\subsection{Сложение чисел с плавающей запятой}

Для выполнения операции сложения необходимо:

\begin{enumerate}
\item выровнять порядки слагаемых, для чего меньший порядок увеличить до большего за счёт денормализации мантиссы числа с меньшим порядком; порядком суммы является общий порядок слагаемых;
\item складывание мантиссы с использованием обратного или дополнительного кода (обычного или модифицированного) по правилам сложения в машинах с фиксированной запятой.
При этом могут встретиться три случая:
\begin{enumerate}
\item сложение происходит без переполнения разрядной сетки и нарушения нормализации;
\item сложение происходит без переполнения разрядной сетки, но результат после перевода в прямой код оказывается ненормализованным (в старшем разряде мантиссы стоит 0) --- нарушение нормализации вправо.
Нормализацию результата осуществляют сдвигом мантиссы влево и соответствующим уменьшением порядка суммы;
\item при сложении мантисс происходит переполнение разрядной сетки --- нарушение нормализации влево.
В этом случае ЭВМ автоматически выполняет сдвиг результата на один разряд вправо, увеличивает на единицу порядок и заносит в правый знаковый разряд цифру, стоящую в левом знаковом разряде.
\end{enumerate}
\end{enumerate}

Время выполнения операции сложения в ЭВМ с плавающей запятой примерно в 2 раза больше времени выполнения в ЭВМ с фиксированной запятой.

\subsection{Умножение чисел в ЭЦВМ}

Умножение двоичных чисел аналогично умножению десятичных чисел: сначала получают частичные произведения, затем их суммируют с учётом значения соответствующего разряда множителя.
В соответствии с таблицей двоичного умножения каждое частичное произведение равно нулю, если в соответствующем разряде множителя стоит 0, или равно множителю, сдвинутому на соответствующее число разрядов, если в соответствующем разряде множителя стоит 1.
Таким образом операция умножения в двоичной системе сводится к операциям сдвига и сложения.
Умножение производится начиная с младшего или старшего разряда множителя, что и определяет направление сдвига.

Сдвиг числа в СС с основанием $q$ эквивалентен умножению этого числа на $q^m$.
Если $m > 0$, то сдвиг осуществляется на $m$ разрядов влево, если $m < 0$, то на $m$ разрядов вправо.
При сдвиге чисел вправо и влево необходимо сохранять содержимое знаковых разрядов (арифметический сдвиг).
Сдвиг всего кода числа, включая знаковые разряды относится к логическим операциям.
При сдвиге положительного числа влево пропадают цифры разрядов, расположенные перед знаковым разрядом, а в появляющиеся справа разряды заносятся нули.

Сдвиг отрицательных чисел, представленных в дополнительном или обратном кодах, выполняется по особым правилам.
При сдвиге отрицательных чисел вправо, во все освобождающиеся слева разряды должны быть занесены цифры знакового разряда.
При сдвиге влево в дополнительных кодах чисел справа заполняются нулями, а для обратных кодах --- единицами.

В зависимости от организации умножения, результат может иметь код одинарной или двойной длины.
Как правило, код двойной длины приходится округлять до одинарной, но в ряде случаев для повышения точности используются результаты двойной длины и предусматривается вывод как старших, так и младших разрядов.
Числа в ЭВМ умножаются в прямом коде (по правилам двоичной арифметики).
Знак произведения определяется алгебраической суммой знаков сомножителей, причём образующуюся единицу переполнения не учитывают.

\subsubsection{Умножение чисел в ЭВМ с фиксированной запятой}

Если после выполнения операции умножения необходимо произвести округление результата до нужного количества разрядов, то пользуются следующим правилом: если в старшем отбрасываемом разряде 0 --- то младший из оставшихся разрядов не изменяется, если $1$ --- то прибавляется к младшему из оставшихся разрядов.

\subsubsection{Умножение чисел в ЭВМ с плавающей запятой}

Протекает аналогично, добавляется операция определения порядка произведения алгебраическим сложением порядков сомножителей и производится нормализация результата с соответствующим изменением порядка произведения.
Скорость выполнения операции умножения определяется числом выполняемых операций сложения и сдвига, в среднем время $\sim n\cdot t$ (сложения).
По этой причине умножение принято называть длинной операцией, по сравнением с операциями сложения и сдвига.
Удельный вес операции умножения --- 20--40\%, а время выполнения --- 70--80\%.

Для ускорения операции умножения применяются различные методы.
Наиболее эффективный основан на преобразовании группы единиц множителей по формуле $2^k-1$. Если умножение $a \cdot 1111\dots{}1$ требует $k$ сложений, то $a \cdot (1000 \dots 0-1)$ требует только два сложения, а число сдвигов остаётся неизменным.
Максимальное число сложений будет при множителе с чередующимися единицами и нулями.

\subsection{Деление чисел}

Одна из самых трудных операций, заменяли на умножение на обратное число.
Операция деления в ЭВМ реализуется одним из двух алгоритмов: с восстановлением остатка и без восстановления остатка.

\newpage
\section{Логические основы построения ЭВМ}

\subsection{Понятие логической функции}

В ЭВМ информация подвергается не только арифметической, но и логической обработке.
ЭВМ выполняет преобразования над двоичными числами в результате чего образуется двоичное число, являющееся результатом выполнения соответствующей логической операции.
В основе работы логических схем и устройств ЭВМ используется математическая логика, изучающая применение математических методов для решения различных логических задач.
Наибольшее применение этой науки получила алгебра логики, часто называемая исчислением высказываний.
Под высказыванием понимается любое утверждение, которое в зависимости от смысла бывает истинным или ложным.
В алгебре логике интересуются не содержанием высказываний, а лишь их истинностью или ложностью; другие признаки высказываний в алгебре логики не рассматриваются.
Если высказывание истинно, то говорят, что его значение равно 1; если ложно --- то 0.

Таким образом, значение высказывания можно рассматривать как дискретную величину, принимающую два дискретных значения --- 0 или 1.
Это приводит к полному соответствию между логическими высказываниями и цифрами двоичной системы исчисления.
С точки зрения логики высказывания делятся на:

\begin{itemize}
\item постоянно истинные высказывания;
\item постоянно ложные высказывания;
\item высказывания, которые могут истинными или ложными в зависимости от определённых условий, т.\,е. принимать значения 1 или 0 попеременно.
\end{itemize}

По содержанию высказывания бывают простыми и сложными(составными).
Сложные высказывания образуются из простых с помощью союзов \textbf{НЕ, ИЛИ, И}.
Простое высказывание при вхождении в состав сложного является логическим аргументом, а сложное высказывание --- логической функцией, зависящей от истинности или ложности аргументов.
Простые высказывания обозначают строчными буквами, сложные --- прописными буквами.

Всякое устройство ЭВМ можно рассматривать как функциональный преобразователь, входными аргументами которого являются исходные двоичные числа, а выходной функцией от них --- новое двоичное число, которое образовалось в результате выполнения данной операции.
При этом как входные аргументы, так и выходные функции могут принимать лишь одно из двух возможных значений: 0 или 1.
В каждом случае количество входных аргументов может быть различным:
\begin{itemize}
\item простейший --- переменная $x$, принимающая значение или 0, или 1;
\item общий случай --- таких переменных может быть $n$: $x_1, x_2, \dots, x_n$.
\end{itemize}

Логические двоичные функции получили название булевых по имени английского математика \texttt{XIX} века \textbf{George Bool}.

Совокупность значений входных аргументов называется \textit{набором} и обозначается $x_1, x_2, \dots, x_n$, где $x_i$ равно или 0, или 1.
Функция $f(x_1, x_2, \dots, x_n)$, определяемая на наборах двоичных аргументов $x_1, x_2, \dots, x_n$, и принимающая в качестве своих возможных значений 0 или 1, называется \textit{логической (булевой, переключательной)} функцией.
Для задания булевой функции достаточно построить таблицу её значений, отвечающей всевозможным различным наборам её аргументов.
Таблица носит название \textit{таблица истинности}.

\subsection{Некоторые свойства переключательных функцией:}
\begin{itemize}
\item любая булева функция от $n$ переменных определяется на $2^n$ наборах.
Набор $n$ аргументов --- двоичное $n$-разрядное число, но количество различных $n$-разрядных двоичных чисел равно $2^n$.
Условимся упорядочивать аргументы булевой функции по возрастанию индексов.
Тогда каждому набору можно приписать номер, равный двоичному числу, соответствующему данному набору;
\item число различных переключательных функций от $n$ аргументов конечно и равно $2^{2^n}$.
Булева функция от $n$ аргументов определена на $2^n$ наборах, на которых она принимает значения $0$ или $1$.
Поэтому каждой булевой функции соответствие $2^n$ разрядной двоичное число, а поскольку этих чисел $2^{2^n}$, то такое же количество и различных булевых функций.
В дальнейшем будем приписывать каждой булевой функции номер, равный двоичному числу, образованному значениями булевой функции.
\end{itemize}

Задача синтеза логических схем ЭВМ эквивалентна математической задаче представления (с использования принципа суперпозиции) сложных булевых функций через простые булевы функции, описывающих работу логических элементов.

\textbf{Суперпозиция} булевой функции состоит в замене аргументов булевой функции другими булевыми функциями и в перестановке (или переименованием) аргументов булевых функций.
При её осуществлении необходимо учитывать и порядок выполнения (т.\,е. старшинство) булевых операций.

Старшинство операций: \textbf{НЕ, И, ИЛИ}.

Преобразования над аргументами, в результате которых получены значения булевых функций получили название \textit{операции алгебры логики}.

\subsection{Булевы функции одного аргумента:}

\begin{itemize}
\item константа 0 --- $f_0(x) = 0$; // генератор нулей
\item переменная $x$ --- $f_1(x) = x$;
\item отрицание $x$ --- $f_2(x) = \overline{\mathstrut x}$; // инверсия переменной $x$; элемент \textbf{НЕ}, инвертор
\item константа 1 --- $f_3(x) = 1$. // генератор единиц.
\end{itemize}

\subsection{Булевы функции двух аргументов:}

\begin{itemize}
\item логическое сложение двух или нескольких простых высказываний --- функциональная зависимость, в результате которой сложное высказывание $Р$ будет истинное, если хотя бы одно из составляющих его простых высказываний истинно, и ложно, когда одновременно ложны все составляющие его простые высказывания.
Формула логической связи $P=X+Y$ --- $P$ есть $X$ или $Y$ --- дизъюнкция.
$f_7$ --- собирательная схема;
\item логическое умножение двух и более высказываний заключается в том, что сложное высказывание будет истинно в том и только том случае, когда все составляющие его простые высказывания будут одновременно истинны. $P=XY$ --- $P$ есть $X$ и $Y$.
$f_1$ --- конъюнктор, схема совпадения.
\end{itemize}

\subsection{Формы логических функций и их использования для синтеза логических схем}

Табличное представление логических функций усложняется с увеличением числа аргументов.
Так для трёх аргументов мы имеем $2^{2^3} = 256$ логических функций.

Более удобным является аналитическое представление логических функций в виде формул.
Из-за неоднозначности представления любой переключательной функции через исходные необходимо найти такую форму её представления, которая соответствует наиболее простой электрической схеме.
Наиболее рациональным является представление логических функций в нормальный формах: \textit{совершенной дизъюнктивной} и \textit{совершенной конъюнктивной}.

\textbf{Элементарная конъюнкция (минтерн)} --- логическая функция, принимающая единичное значение на одном из всех возможных наборов и нулевые значения --- на всех остальных наборах.

\textbf{Элементарная дизъюнкция (макстерн)} --- логическая функция, принимающая нулевое значение на одном из всех возможных наборов входных аргументов и единичное значение --- на всех остальных наборах.

Алгебраически минтерн --- конъюнкция аргументов и их отрицаний, а макстерн --- дизъюнкция аргументов и их отрицаний.
Количество минтернов и макстернов совпадает с числом наборов различных аргументов, т.\,е. для $n$ аргументов их будет $2^n$.

\textbf{Ранг} функции --- число аргументов, составляющих элементарную конъюнкцию или дизъюнкцию.

Форма представления логических функций посредством инверсии, конъюнкции и дизъюнкции называется нормальной.

Различают конъюнктивную и дизъюнктивную нормальную форму.

\begin{itemize}
\item \textbf{Конъюнктивная (КНФ)} --- логическое произведение элементарных дизъюнкций.
\item \textbf{Дизъюнктивная (ДНФ)} --- логическая сумма элементарных конъюнкций.
\end{itemize}

Выражение логической функции, составленное из минтернов одинакового ранга, связанных дизъюнкцией, называют \textit{совершенная дизъюнктивная нормальная форма (СДНФ)}.

Логическую функцию, содержащую макстерны одного ранга, связанные конъюнкцией, называют \textit{совершенной конъюнктивной нормальной формой (СКНФ)}.

Если логическая функция содержит конъюнкции (дизъюнкции) разных рангов, то следует повысить её ранг для образования СДНФ (СКНФ).

При анализе или синтезе логических схем ЭВМ обычно используют описание их работы в виде соответствующей таблицы, из которой легко могут быть получены логические формулы в виде СДНФ (СКНФ).

\subsubsection{Правила образования СДНФ функции по таблице истинности}

\begin{enumerate}
\item По каждому набору переменных, на которых функция принимает значение 1, составить элементарные конъюнкции (минтерны).
\item В минтерн записать неинвертированными переменные, заданные 1 в таблице истинности, а инвертированными --- те переменные, которые заданы в таблице истинности 0.
\item Соединить элементарные конъюнкцию знаком дизъюнкции.
\end{enumerate}

\subsubsection{Правила образования СКНФ функции по таблице истинности}
\begin{enumerate}
\item По каждому набору переменных, на которых функция принимает значение $0$, составить макстерны.
\item В макстерн записать неинвертированными переменные, заданные $0$ в соответствующем наборе, а инвертированными --- те переменные, которые заданы 1.
\item Соединить макстерны знаком конъюнкции.
\end{enumerate}

Полученные формулы в СКНФ (СДНФ) можно использовать для синтеза функциональных схем логических устройств.

\subsection{Понятие о функциональных системах и булевых функция}

Одна из задач синтеза схем заключается в выборе типов элементов, из которых должны собираться логические схемы.
Основное требование к набору логических элементов --- построить с помощью имеющегося набора любую сколь угодно сложную схему.
Но ввиду того, что законы функционирования элементов описываются булевыми функциями, то сформулированное требование сводится к определению набора таких булевых функций, с помощью которых можно получить любую сколь угодно сложную функцию, что соответствует операции суперпозиции, заключающейся в замене одних аргументов функции другими.
Упрощение логических выражений можно достичь выражением сложных булевых функций через другие.
Система булевых функций называется \textit{функционально-полной}, если с помощью функций, входящих в эту систему, применяя операции суперпозиции, можно получить любую сколь угодно сложную булеву функцию.

\subsubsection{Минимизация булевых функций}

Преобразования булевой функцией с целью упрощения её аналитического выражения называется её \textit{минимизацией}.
Рассмотренные СДНФ (СКНФ) используются для первоначального представления заданной булевой функции через функции основной системы.
Однако эти формы неудобны для построения логических схем ЭЦВМ, т.\,к. схемы, реализующие эти формы, оказываются часто сложными.
Для минимизации логических функций на практике используются следующие методы:

\begin{itemize}
\item метод непосредственных преобразований;
\item метод Квайна;
\item метод Корно-Вейча;
\item метод Блэйка;
\item и другие.
\end{itemize}

\paragraph{Метод непосредственных преобразований}

Алгоритм:

\begin{enumerate}
\item Строится СДНФ булевой функции по таблице истинности.
\item С использованием различных законов и соотношений алгебры логики СДНФ преобразуется (и упрощается) в минимальную дизъюнктивную нормальную форму (МДНФ), содержащую минимальное число булевых элементарных конънкций.
\end{enumerate}

\textit{Проведём синтез схем для функции $F(x,y,z)$}
\begin{itemize}
\item на логических элементах \textbf{НЕ, И, ИЛИ};
\item на элементах Пирса, \textbf{НЕ, ИЛИ}.
\end{itemize}

\[F(x,y,z)=\bar x\bar y\bar z+\bar xy\bar z+x\bar yz+xyz=\bar x\bar z\left(y+\bar y\right)+xz\left(y+\bar y\right)=\bar x\bar z+ xz\]

\[F(x,y,z)=\overline{\mathstrut \left( {x + z} \right)} + \overline{\mathstrut \left( \bar x+\bar z\right) }\]


\subsection{Понятие о синтезе логических схем и цифровых автоматов}

\subsubsection{Этапы синтеза цифровых устройств}

При синтезе логических устройств ЭВМ, выполняющих необходимое преобразование информации, можно выделить следующие этапы:
\begin{enumerate}
\item на основании анализа функций, которые должны выполняться данным устройством, формируется логические условия его функционирования в виде соответствующей таблицы;
\item по этой таблице составляется СДНФ булевой функции;
\item производится минимизация логической функции;
\item по упрощённой логической формуле строится функциональная схема устройства, причём минимальному числу и однородности логических элементов отдаётся предпочтение.
\end{enumerate}

\subsection{Комбинационные схемы и конечные автоматы}

По существу логическое устройство ЭВМ выполняет необходимые преобразования поступающей на вход цифровой информации.
Их возможно классифицировать по типам:
\begin{itemize}
\item комбинационные схемы, или автоматы без памяти, или примитивные автоматы;
\item конечные, или полные автоматы, или автоматы с памятью.
\end{itemize}

\subsubsection{Комбинационные схемы}

\textbf{Комбинационные схемы (КС)} --- устройство, в котором совокупность выходных сигналов в дискретный момент времени $t_i$ однозначно определяется набором входных сигналов, поступивших на вход устройства в тот же момент времени $t_i$.
КС можно представить в виде $m$-$k$-полюсника, имеющего $i$ входов и $j$ выходов.
Входной алфавит КС задаётся набором символов $m_1, m_2, \dots, m_i$, а выходной алфавит КС задаётся набором символов $k_1, k_2, \dots, k_j$.
КС характеризуется:
\begin{itemize}
\item числом входных сигналов
\item числом выходных сигналов
\item логической формулой или таблицей истинности
\end{itemize}

При изменении набора входных сигналов $M$ меняется набор выходных сигналов $K$.
Таким образом выходные сигналы КС полностью определятся входными сигналами и не зависят от внутреннего состояния КС.
Поэтому КС также называются \textbf{автоматами без памяти}, или \textbf{примитивными автоматами}.
Конечно КС в таком представлении является идеализацией техничного устройства.
В реальных схемах из-за различной продолжительности переходных процессов в цепях и элементах схем, время появления выходных сигналов не совпадает со временем поступления входных сигналов, т.\,е. выходные сигналы появляются с некоторым запаздыванием по отношению ко времени поступления входного слова.
С целью управления работы устройств --- разрешения подачи на вход КС нового набора входных сигналов лишь после окончания переходных процессов в КС --- используется \textbf{тактированный способ} работы устройств, когда новый такт преобразования информации начинается только после окончания предыдущего.

\subsubsection{Конечный автомат}

\textbf{Конечный автомат} --- устройство, выходное слово в котором в дискретный момент времени $t_i$ определяется не только по входному слову в тот же момент времени $t_i$, а и по его внутреннему состоянию, обусловленному его предшествующими этапами работы.

Автомат с памятью задаётся тремя наборами переменных:
\begin{itemize}
\item $M$ --- входной набор;
\item $K$ --- выходной набор;
\item $Q$ --- внутренний алфавит.
\end{itemize}

Функционирование конечного автомата однозначно определено если установлены связи во времени между тремя его алфавитами.
Обычно значения алфавитов сохраняются неизменными на протяжении временных интервалов, называемых \textbf{тактами}.

Закон функционирования конечного автомата можно задать такими способами:
\begin{itemize}
\item табличным;
\item аналитическим;
\item с помощью графов;
\item матричным;
\item и другими.
\end{itemize}

%\begin{comment}Набор входных переменных в дискретный момент $t$.
%Внутреннее состояние автомата $Q_t$ в дискретный момент времени $t$ (до воздействия входных переменных).
%Внутреннее состояние автомата $Q_{t+1}$ в дискретный момент времени $t+1$ (после воздействия переменных).
%Значения выходных переменных $K_{t+1}$ (после воздействия входных переменных $M_t$) в дискретный момент времени $t+1$.
%\end{comment}

Аналитический способ реализуется заданием формул или уравнений для функций переходов и выходов, получаемых из табличного представления образование СКНФ или СДНФ.

\textbf{Функции переходов} устанавливают аналитическую связь между внутренним состоянием автомата $Q_{t+1}$ в момент времени $t+1$ в зависимости от предыдущего состояния $Q_t$ и набора входных переменных $M_t$.

\textbf{Функции выходов} отображают связь между выходным алфавитом $K_{t+1}$ в зависимости от внутреннего состояния $Q_t$ и входного $M_t$ алфавитов.

Таким образом, аналитическое поведение автомата описывается так

\[Q_{t+1}=F_1(Q_t,M_t)\]

\[K_{t+1}=F_2(Q_t,M_t)\]

Конечный автомат, выходной сигнал которого зависит от состояния, в котором находится автомат и от входного сигнала, называется \textbf{автоматом Мели}.

Существуют автоматы другого типа, у которых выходной сигнал зависит исключительно от состояния автомата и не зависит от входного сигнала --- \textbf{автоматы Мура}:

\[Q_{t+1}=F_1(Q_t,M_t)\]

\[K_{t+1}=F_2(Q_t)\]

Графическое представление закона функционирования автоматов осуществляется с помощью графов.
При этом, вершинами графа (узловыми окружностями) отображают внутреннее состояние автомата, а переход из одного состояния в другое --- дугами (ветвями) графа, на которых указывают входные переменные в момент изменения его внутреннего состояния.
Выходные переменные для автоматов Мура наносятся внутри вершин графа, а для автомата Мели --- на дугах.

Рассмотренные способы используют при анализе работы автоматов и синтезе их логической структуры.

\newpage
\section{Типовые элементы ЭВМ}

\subsection{Классификация и общая характеристика элементов ЭВМ}
\textbf{Элемент ЭВМ} --- функциональное устройство, выполняющее одну из задач реализации булевых функций, с запоминанием информации, преобразования, формирования и усиления сигналов.

Несмотря на то, что количество элементов в ЭВМ большое, число их разновидностей относительно невелико, что значительно упрощает разработку и повышает технологичность изготовления ЭВМ.

\subsubsection{Классификация элементов по назначения}

\begin{itemize}
\item Логические (однофункциональные элементы) --- устройство, реализующее одну логическую функцию \textbf{НЕ, И, ИЛИ} (инвертор, конънктор и дизънктор).
Данная система элементов составляет функционально-полную систему элементов.
\item Функциональные --- устройства, реализующие несколько булевых функций (\textbf{И-НЕ} (\textit{Шеффера}), \textbf{ИЛИ-НЕ} (\textit{Пирса}), \dots ).
Сюда же относятся запоминающие устройства (триггеры, интегральные элементы памяти, элементы памяти на ферритовых сердечниках).
\item Вспомогательные --- осуществляют формирование, усиление, генерацию сигналов, а также их преобразования по мощности, амплитуде и длительности, не изменяя их логических значений (усилители, формирователи, генераторы, преобразователи).
\end{itemize}

\subsubsection{Элементы ЭВМ для работы с сигналами, задающий 0 или 1}

По характеру сигналов, используемых для физического представления двоичных нуля и единицы, все элементы делятся на:

\begin{itemize}
\item потенциальные (потенциал --- сигнал неограниченной длительности) --- заключается в том, что ноль и единица передаются различными уровнями потенциалов или определёнными значениями тока.
Сигналы 0 и 1 могут быть заданы двояко. В \textbf{системе низких потенциалов (СНП)}, или в \textbf{отрицательной логике} единица задаётся более низким уровнем сигнала, чем 0.
В СВП --- наоборот.
\item Импульсные --- состоит в том, что 1 задаётся импульсом положительной полярности, а 0 --- импульсом отрицательной полярности.
Применяется когда 1 воспроизводится наличием импульса, а 0 --- его отсутствием.
Используется комбинированный способ (потенциально-импульсно) --- когда в одной схеме обрабатывается два вида сигналов.
\item Динамические --- единица передаётся электрическими сигналами синусоидальной или пилообразной формой, а 0 --- их отсутствием.
\end{itemize}

В интегральной схемотехнике используется потенциальный способ.

\paragraph{По параметрам}

По параметрам элементы можно классифицировать в зависимости от

\begin{itemize}
\item реализации логических функций;
\item Нагрузочная способность элементы $D$ характеризуется числом $n_r$ аналогичный элементов $D^\prime$ которых можно подключить к его выходам без снижения амплитуды выходных информационных сигналов передачи 0 и 1 (чтобы обеспечивалось срабатывание элементов $D^\prime$.
Оценивается коэффициентом разветвления по выходу $n_r$.
Чем выше $n_r$, тем меньше элементов требуется для построения ЭВМ. $n_r \in [4;10]$, у специальных устройств $n_r \in [20;50]$.
Максимальное число $n$ логических элементов $D$, подключаемых ко входу элемента без изменения его выходного сигнала, называется коэффициентом объединения по вхожу.
Увеличение $n$ упрощает реализацию многих логических устройств.
На практике $n \in [2;6]$.
Для увеличения $n$ используют специальные устройства, которые называются \textit{расширители}, которые позволяют увеличить значение $n$ до 10.
\item Быстродействие или время задержки характеризует элементы по длительности переходных процессов при передаче 0 или 1 и обуславливается формирование фронта или спада импульса и задержки из-за инерционности срабатывания.
Для ИС $\overline{\mathstrut t} = 10^{-8}$ -- $10^{-9}$ с.
\item Предельная рабочая частота элемента --- диапазон рабочих частот сигналов, передаваемых элементом без искажения, чтобы за время одного такта успели завершиться переходные процессы.
\item Помехоустойчивость элемента в зависимости от режима его работы может быть
\begin{itemize}
\item статической --- устойчивая работа элемента при длительном воздействии потенциала помехи; определяется уровнем напряжения, которое можно подать на вход элемента относительно уровня $0$ или $1$ без вызова ложного срабатывания элемента;
\item динамической --- рассматривается при работе элементов в импульсном режиме и зависит от параметров импульса помехи (формы и амплитуды), скорости переключения элементов.
\end{itemize}
\item Потребляемая мощность элемента.
Различают два режима работы элемента --- режим переключения и статическое состояние.
В зависимости от режима работы потребляемая мощность различна и возрастает при повышении частоты переключения.
\end{itemize}

К общим техническим параметрам относят надёжность, стоимость, конструктивно-механические особенности и другие.

\subsection{Интегральная схемотехника}

Требования увеличения быстродействия и уменьшения потребляемой мощности привели к созданию ряда биполярных цифровых интегральных схем.
Современные технологии позволяют изготавливать микросхемы, содержащие десятки тысяч логических элементов.

ИС содержит отдельные логические элементы.
СИС (средние ИС) содержат 1 или несколько функциональных узлов (сумматоры, регистры \dots).
В состав БИС (больших), имеющих степень интеграции 2--4, входит одно или несколько функциональных устройств (АЛУ, УУ, ОЗУ).
В состав CБИС (сверхБИС), имеющих $k > 4$, --- входят микропроцессоры и построенные на их основе сложные цифровые устройства, в том числе и микроЭВМ.

\subsubsection{Классификация ИС}

\paragraph{По технологическому исполнению:}
\begin{itemize}
\item Плёночные ИС --- все элементы в виде плёнок на изоляционной подложке.
\item Твёрдосхемные ИС (наиболее распространённые) --- в поверхностном слое поверхности кремния площадью 1.5 мм$^2$ методами планарной технологии формируются области, выполняющие функции диодов, транзисторов и пассивных компонентов.
\item Гибридные ИС --- сочетают активные компоненты (бескорпусные диоды и транзисторы) в сочетании с пассивной частью в виде многослойной схемы, выполненной вакуумным напылением плёнок на подложку.
\end{itemize}

\paragraph{По типу базовых элементов:}
\begin{itemize}
\item Элементы резисторно-транзисторной логики (РТЛ).
\item Элементы диодно-транзисторной логики (ДТЛ).
\item Элементы транзисторно-транзисторной логики (ТТЛ).
\item и другие.
\end{itemize}

\subsubsection{Требования, предъявляемые к выбору элементной базы ЭЦВМ и цифровых устройств:}
\begin{itemize}
\item Набор элементов должен быть однородным, так, чтобы число типов элементов было минимальным.
\item Выбранные элементы должны составить полную функциональную систему элементов.
\item Все элементы должны быть совместимы по входным и выходным параметрам без дополнительных согласующих устройств.
\item Элементы должны быть надёжными в заданном рабочем интервале.
\item Число источников питания должно быть минимальным
\end{itemize}

\subsubsection{Логические элементы ЭВМ}

\paragraph{Элементы РТЛ}

Работа логических схем основывается на работе транзисторного ключа или обычного усилительного ключа или обычного усилительного каскада.
В отличии от усилительного каскада, на вход ключа подаётся значительный по амплитуде сигнал, задающий транзистору определённое состояние в ключевом режиме.
Простейшие схемы ключа с резистивными связями является транзисторный каскад с общим эмиттером.

Изменяя ток базы, можно управлять током коллектора.

Транзистор может находиться в состоянии насыщения, активном (или усиления) и отсечки (не проводит).

Для реализации двузначных переменных используется два состояния транзисторного ключа --- насыщения (транзистор открыт) и отсечки (транзистор закрыт).
Логический вход для задания логического аргумента обозначен $X$, а функциональный выход - $P$. $R_{б}$, $R_{к}$ и $R_{см}$ --- базовый, коллекторный и резистор смещения.
$+E_{см}$ и $-E_{к}$ --- напряжение смещения и коллекторное напряжение (напряжение питания).
Напряжение питания для $npn$-транзисторов - $+E_{к}$.
Пусть в исходном состоянии на вход ключа поступает отрицательное напряжение, представляющее собой логический 0 (интервал 0--$t_1$ на временной диаграмме).
Проходя по цепи $R_{б}$--$R_{см}$ ток понижает потенциал базы транзистора до такого уровня, при котором транзистор полностью открывается.
Так как в открытом транзисторе сопротивление цепи эмиттер-коллектор достаточно мало по сравнению с $R_{к}$, то практически всё напряжение источника питания --- $E_{к}$ падает на резисторе $R_{к}$ и на выходе ключа устанавливается потенциал, близкий к нулевому, соответствующий логической 1 в СВП.
Если на вход ключа поступает логическая 1 в виде высокого (близкого к нулевому) уровня напряжения (интервал $t_1$--$t_2$), то благодаря источнику $+E_{см}$ на базе транзистора устанавливается положительный потенциал, транзистор закрывается, и на выходе схемы устанавливается близкий к $-E_{к}$ (низкий) потенциал, соответствующий логическому 0 в СВП.
Таким образом работа транзисторного ключа соответствует реализации логической функции \textit{НЕ}.

Для увеличения быстродействия схемы $R_{б}$ шунтируют ёмкостью $C_{б}$, которая в момент переключения транзистора ускоряет установление переходных процессов в цепи базы при смене уровня напряжения.

Для реализации функциональных элементов (более сложных булевых функций) применяются схемы с последовательным и параллельным соединением, рассмотренных транзисторных ключей.

Рассмотрим схемы на последовательно соединенных транзисторах.

В СВП схема, построенная на последовательно соединённых транзисторах $pnp$ типа реализует в положительной логике \textit{функцию Пирса}.
В отрицательной логике данная схема реализует \textit{функцию Шеффера}.

Для получения схемы \textit{И-ИЛИ} на выход ТТЛ элемента необходимо подключить инвертор.
Рассмотренные схемы носят название транзисторных логических схем с непосредственными связями (\textit{НСТЛ}).

Достоинства схем НСТЛ --- высокое быстродействие (задержка фронта --- 5--10 нс) и высокая рабочая частота (рабочая частота --- 10--20 МГц).

Недостаток --- низкая помехоустойчивость и невысокая нагрузочная способность ($k \leqslant 4$).
Для компенсации этих недостатков в схемах НСТЛ вводят резистивные или резистивно-ёмкостные связи (РТЛ и РЕТЛ).

\subparagraph{Резистор во входных цепях}
\begin{itemize}
\item[+] снижают рабочие токи и потребляемую мощность элементов;
\item[+] увеличивают нагрузочную способность;
\item[+] повышают помехоустойчивость;
\item[-] снижает быстродействие (задержка фронта --- 30--50 нс);
\item[-] повышает требовательность к разбросу параметров резисторов;
\item[-] требует больше площади в ИС.
\end{itemize}

\subparagraph{Конденсаторы во входных цепях параллельно резисторам}
\begin{itemize}
\item[+] Увеличивают быстродействие (задержка фронта --- 10--15 нс) при сохранении всех других параметров;
\item[-] технологические трудности в изготовлении ИС.
\end{itemize}


\paragraph{Элементы ДТЛ}

Диодные схемы могут работать как в импульсном, так и в потенциальном режиме.
Для реализации логических элементов \textit{ИЛИ, И} можно использовать одну из диодных схем (либо с общим анодом, либо с общим катодом), изменяя систему задания двоичной переменной.

\subparagraph{Диодная схема с общим анодом}

Диодные схемы реализуют только функции \textit{ИЛИ, И} и поэтому функциональные возможности их ограничены и неполные.
Они имеют невысокий коэффициент передачи, поэтому нужны дополнительные усилители.
Их применяют в сочетании с транзисторами, образуя при этом ДТЛ.

\subsubsection{Запоминающие элементы}

Элементы памяти, предназначенные для записи, хранения и считывания двоичной информации.
Такие элементы имеют два устойчивых состояния, одно из которых принимается за логический нуль, а другое --- за логическую единицу.
В качестве элементов памяти наибольшее распространение получили триггеры, магнитные сердечники с прямоугольной петлёй гистерезиса (ППГ) и полупроводниковые элементы памяти.

\textbf{Триггер} --- элементарный автомат, имеющий два устойчивых состояния.
Функция памяти триггера заключается в том, что он остаётся в одном из устойчивых состояний после прекращения действия входного сигнала, приведшего его в это состояние.
Триггеры используются для кратковременного хранения информации при выполнении арифметических и логических операций.
Простейшая схема триггера на дискретных компонентах:(((Картинко))).

В схемном отношении триггер (\textbf{Тг}) - два усилителя постоянного тока, с взаимнообратными положительными связями, благодаря которым один из транзисторов открыт, а другой --- закрыт.
Пусть в исходном состоянии транзистор $T_1$ открыт, а транзистор $T_2$ закрыт, т.\,е. потенциал точки $Q$ высокий, близкий к нулю, а потенциал точки $\overline{\mathstrut Q}$ низкий, близкий к $E_{к}$.
Если на вход $S$ подать сигнал высокого уровня, то транзистор $T_1$ закроется и с выхода $Q$ будет сниматься потенциал низкого уровня, примерно равный $-E_{к}$.
Этот потенциал откроет транзистор $T_2$ и с выхода $\overline{\mathstrut Q}$ будет сниматься потенциал высокого уровня (более положительный, чем $-E_{к}$), который будет поддерживать транзистор $T_1$ в закрытом состоянии.
Если же в этом состоянии триггера на вход $S$ повторно подать сигнал высокого уровня, то состоянии схемы не изменится.
При подаче сигнала высокого уровня на вход $R$ схемы транзистор $T_1$ откроется и на выходе $Q$ будет высокий уровень напряжения, а на выходе $\overline{\mathstrut Q}$ --- низкий.
Если повторно подать на вход $R$ сигнал высокого уровня, то состояние схемы не изменится.

Эти два состояния будут устойчивыми и после прекращения действия управляющих сигналов $S$ и $R$.

При одновременной подаче сигналов на $R$ и $S$ триггер переходит в неустойчивое состояние.

Принято считать, что триггер хранит ,,1'', если на его выходе $Q$ находится потенциал низкого уровня ($T_1$ закрыт, $T_2$ открыт), ,,0'' --- высокого уровня ($T_1$ открыт, $T_2$ закрыт).

Вход $S$\footnote{\textbf{S}et}, при котором триггер устойчив в единичном состоянии --- \textit{единичный вход}, $R$\footnote{\textbf{R}eset} --- при котором в нулевом --- \textit{нулевой вход}.

Для передачи единичного сигнала из триггера необходимо изменить его состояние из ,,1'' на ,,0'', и получить на выходе $Q$ изменение потенциального сигнала от низкого к высокому.

Триггеры в современной техники строятся на базе РТЛ, ТТЛ, \dots.

Различают модификации триггеров в зависимости от их предназначения и схемного состава.
Управлять можно не только с помощью управляющих входов $R, S$, но и другими сигналами (тактирующими или синхросигналами).
По набору входов определяется тип триггера

В ряде триггерных схем сигнал на выходе зависит не только от управляющих входных воздействий, но и от внутреннего состояния триггера.

Логическое проектирование триггеров вытекает из рассмотренных нами алгоритмов синтеза логических схем и включает такие этапы:
\begin{enumerate}
\item логическое описание законов функционирования в виде таблиц переходов;
\item составление по этим таблицам логических функций или уравнений;
\item упрощение логического уравнения;
\item выбор элементной базы;
\item построение структурной схемы триггера.
\end{enumerate}

\begin{longtable}{|c|c|c|c|c|l|}
\caption{Таблица переходов $R$-$S$-триггера}\\
\hline
$Q_t$	&	$S_t$	&	$R_t$	&	$Q_{t+1}$	&	$\overline{\mathstrut Q_{t+1}}$	&	Комментарий\\
\hline
0	&	0	&	0	&	0	&	1	&	Хранение нуля\\
0	&	0	&	1	&	0	&	1	&	Подтверждение нуля\\
0	&	1	&	0	&	1	&	0	&	Установка единицы\\
0	&	1	&	1	&	---	&	---	&	Неопределенное состояние\\
1	&	0	&	0	&	1	&	0	&	Хранение единицы\\
1	&	0	&	1	&	0	&	1	&	Установка нуля\\
1	&	1	&	0	&	1	&	0	&	Подтверждение единицы\\
1	&	1	&	1	&	---	&	---	&	Неопределённое состояние\\
\hline
\end{longtable}

Рассмотренный асинхронный $R$-$S$ триггер может быть выполнен на двух логических элементах \textbf{И-НЕ} или на двух логических элементах \textbf{ИЛИ-НЕ}.

\paragraph{Синхронные триггеры}

В этих триггерах запись информации производится только при поступлении тактирующего (разрешающего) сигнала.
Для управления синхронного однотактного $R$-$S$ триггера вводится дополнительный командный вход $C$, разрешающий воздействие сигналов $R$ или $S$ при наличии тактирующего сигнала (синхроимпульса) $C$.

Элементы $D_1$ и $D_3$ образуют схему входной логики для синхронного управления $R$-$S$ триггером, построенном на элементах $D_2$ и $D_4$.
Так как входная информация поступает через дополнительные элементы $I$, она может быть записана в триггер $T$ только при поступлении на синхронизирующий вход $C$ тактирующего сигнала $T$.
При отсутствии синхронизации триггер может быть установлен в состоянии ,,1'' или ,,0'' подачей сигналов, соответствующих логическому нулю на дополнительные, несинхронизируемые входы $\overline{\mathstrut R_{д}}$ и $\overline{\mathstrut S_{д}}$.

Из временной диаграммы видно, что триггер предварительно установленный в состояние ,,1'', сигналом ,,0'' по входу $\overline{\mathstrut S_{д}}$ может изменить своё состояние на нулевое только при поступлении сигналов ,,1'' на входы $R$ и $C$ одновременно.
Затем при одновременном воздействии сигналов $S$ и $T$ триггер переходит в единичное состояние.

Наряду с однотактными применяют также двухтактные $R$-$S$ триггеры, обозначаемые \textit{TT}.

\paragraph{Счётный триггер}

Этот триггер изменяет своё состояние от каждого поступающего на его вход сигнала и называется $T$-триггер.
Он имеет один управляющий вход $T$ --- общий для двух комбинационных схем, два выхода $Q$ и $\overline{\mathstrut Q}$ и дополнительный вход $R$ для установки нуля в некоторых разновидностях триггера.

При поступлении на вход $T$, равного единице, происходит запись в триггер состояния, противоположному ранее хранимому.
Таким образом триггер производит счёт импульсов по модулю 2, и поэтому сигнал ,,1'' на его выходе появляется в два раза реже, чем на входе $T$, поэтому счётный триггер может использоваться как делитель частоты.
Логика работы Т-триггера задаётся таблицей

\begin{longtable}{|c|c|c|c|}
\caption{Таблица переходов счётного $T$-триггера}\\
\hline
$T$	&	$Q_t$	&	$Q_{t+1}$	&	$\overline{\mathstrut Q_{t+1}}$\\
\hline
0	&	0	&	0	&	1\\
0	&	1	&	1	&	0\\
1	&	0	&	1	&	0\\
1	&	1	&	0	&	1\\
\hline
\end{longtable}

\[Q_{t+1}=\bar{T}Q_t+T\overline{\mathstrut Q_t}\]

Существуют другие типы триггеров, которые отличаются способом функционального построения:
\begin{itemize}
\item \textbf{$D$-триггер}\footnote{\textbf{D}elay} --- реализует функцию временной задержки входного сигнала.
\item \textbf{$JK$-триггер} --- универсальный триггер.
В зависимости от соединения его входов может работать как синхронный $R$-$S$-триггер, $T$-триггер или $D$-триггер.
Сигналы $G$ и $K$ во время действия сигнала $C$ должны быть неизменными.
\item \textbf{$DV$-триггер} --- универсальный триггер.
\end{itemize}

\newpage
\section{Типовые функциональные схемы}
\subsection{Регистр}
\textbf{Регистр} --- функциональное устройство, предназначенное для приёма, кратковременного хранения и считывания двоичной информации.
Регистр состоит из отдельных триггеров.

\textbf{Регистр} --- линейка из $n$ триггеров, предназначенная для приёма, кратковременного хранения и считывания $n$-разрядного двоичного числа.

По назначения регистры делятся на:
\begin{itemize}
\item накопительные --- используют для ввода, хранения и вывода двоичной информации.

Числа в регистр поступают параллельно в двоичном коде или в \textit{параллельном парафазном} двоичном коде.

Парафазный параллельный код отличается тем, что помимо входных сигналов на все входы поступают и их инверсные значения.

Вывод информации производится в прямом или обратном (инверсном) кодах.
\item Сдвигающие --- предназначены для ввода, хранения, сдвига и вывода двоичных чисел.

Сдвиг информации может быть организован вправо, влево или реже в двух направлениях на один или больше разрядов.

Ввод чисел производится в последовательном или параллельном коде.

Последовательный код применяется тогда, когда не имеет существенного значения быстродействие устройства.
\item Преобразующие.

Работают с информацией в параллельном и последовательном кодах.
Выполняют кроме операций записи, хранения, сдвига, чтения целый ряд логических операций: дизъюнкцию, конъюнкцию, сложение и другие, преобразуют параллельный код в последовательный код и наоборот.
\end{itemize}

Кроме триггеров в состав регистров входят комбинационные логические схемы.

\subsubsection{Характеристики регистров}
\begin{itemize}
\item Тип триггеров;
\item набор операций;
\item последовательность ввода и сдвига;
\item период следования тактирующих сигналов.
\end{itemize}

Рассмотрит принцип работы регистров на базе функциональных схем без учёта элементной базы из серии микросхем.

\subsubsection{Накапливающий регистр параллельного действия на $R$-$S$-триггерах}

\subsubsection{Сдвигающий регистры или регистры последовательного действия}

Осуществляют сдвиг двоичной информации, поступающей в двоичном коде разряд за разрядом.

Сдвиг чисел используется при выполнении операций умножения и деления.
\paragraph{Схема четырёхразрядного сдвигающего регистра на двухтактных триггерах $D$-типа}


Широкое применение получили преобразователи

\subsection{Счётчики}

\textbf{Счётчик} --- функциональный узел, предназначенный для:
\begin{itemize}
\item подсчёта числа импульсов, поступающих на его вход;
\item временного хранения каждого состояния;
\item преобразования непрерывного сигнала, заданного последовательностью импульсов в параллельный двоичный код или в набор управляющих сигналов;
\item деление частоты входного сигнала.
\end{itemize}

Основой счётчиков являются $T$-триггеры.

Число разрядов счётчика определяет количество его устойчивых состояний, которое называется коэффициентом пересчёта $K_{сч}$.

В зависимости от величины $K_{сч}$ счётчики бывают двоичные (бинарные) и с произвольным коэффициентом пересчёта (соответственно без обратных связей и с обратными связями).

В бинарных счётчиках $K_{сч} = 2^n$, где $n$ --- число разрядов счётчика.

В счётчиках с произвольным коэффициентом пересчёта $K_{сч}\not=2^n$.
Такой счётчик приходит в исходное состояние при поступлении на его вход $K_{сч} \not = 2^n$ счётных импульсов

\paragraph{По назначению счётчики бывают:}
\begin{itemize}
\item \textit{Суммирующие (прямого счёта или сложения)}; работает по принципу суммирования импульсов, поступающих на его вход (((четырёхразрядный суммирующий счётчик на $T$-триггерах, в качестве которых применяются универсальные $JK$-триггеры)))
\item \textit{Вычитающие (обратного счёта или вычитания)}.
Для построения счётчика обратного счёта необходимо в двоичном счётчике перенос от разряда к разряду брать не с единичных, а с нулевых выходов триггеров.
(((Четырёхразрядный вычитающий счётчик на $T$-триггерах, в качестве которых применяются универсальные $JK$-триггеры)))
При такой коммутации перенос от разряда к разряду образуется при переходе соответствующего триггера в состояние хранения ,,1'' (а не ,,0'', как это было в суммирующем счётчике).

Уст,,0'' --- подачей сигнала очищает счётчик.
Первый импульс устанавливает все триггеры в единичное состояние (\texttt{1111}): \textbf{Тг$_1$}, переключаясь из ,,0'' в ,,1'', образует на инверсном выходе $\overline{\mathstrut Q_0}$ изменение потенциала от более высокого к более низкому, т.\,е. происходит считывание ,,1'' c $\overline{\mathstrut Q_0}$, поступающей на входы $T$ следующих разрядов счётчика и устанавливающий таким же ,,1'' сигналом с $\overline{\mathstrut Q_i}$ выходов в единичное состояние все триггеры.
Второй импульс изменит состояние \textbf{Тг$_1$} на ,,0'', и при этом с его выхода передачи сигнала на следующие разряды не будет, а состояние счётчика будет \texttt{1110}.
Третий импульс изменит состояние \textbf{Тг$_1$} на единичное, а \textbf{Тг$_2$} перейдёт в состояние ,,0''; установится состояние счётчика \texttt{1101} и т.\,д.

\begin{longtable}{|c|c|c|c|c|}
\caption{Таблица переходов, отражающая все состояния счётчика}\\
\hline
\textnumero~импульса		&	\textbf{Тг$_1$}	&	\textbf{Тг$_2$}	&	\textbf{Тг$_3$}	&	\textbf{Тг$_4$}\\
\hline
1	&	1	&	1	&	1	&	1\\
2	&	1	&	1	&	1	&	0\\
3	&	1	&	1	&	0	&	1\\
4	&	1	&	1	&	0	&	0\\
5	&	1	&	0	&	1	&	1\\
6	&	1	&	0	&	1	&	0\\
7	&	1	&	0	&	0	&	1\\
8	&	1	&	0	&	0	&	0\\
9	&	0	&	1	&	1	&	1\\
10	&	0	&	1	&	1	&	0\\
11	&	0	&	1	&	0	&	1\\
12	&	0	&	1	&	0	&	0\\
13	&	0	&	0	&	1	&	1\\
14	&	0	&	0	&	1	&	0\\
15	&	0	&	0	&	0	&	1\\
16	&	0	&	0	&	0	&	0\\
\hline
\end{longtable}
Отдельно счётчики обратного счёта используются редко.
Операция вычитания организуется вместе со сложением в схемах реверсивных счётчиков.
\item \textit{Реверсивные (работающие в двух направлениях)}.
В этом типе счётчиков предусмотрены дополнительные логические схемы, управляющие переключением счётчика на суммирование или на вычитание.
\item \textit{Делители частоты}.
Рассмотренные схемы счётчиков применяются в качестве делителей частоты входного сигнала: если на входе счётчика сигнал появляется с частотой $\nu$, то на выходе \textbf{Тг$_1$} частота появления сигнала будет равна $\nu/2$, на выходе \textbf{Тг$_2$} --- $\nu/4$, на выходе \textbf{Тг$_3$} --- $\nu/8$, на выходе \textbf{Тг$_4$} --- $\nu/16$.
\end{itemize}

\subsubsection{Десятичные счётчики}
В двоичных счётчиках коэффициент пересчёта, т.\,е. количество различных устойчивых состояний, равен $2^n$ ($n$ --- число разрядов).
В зависимости от $n$ такой счётчик может отсчитать 2, 4, 8, 16,\dots импульсов и выработать на своём выходе перенос.

В ряде случаев необходимо, чтобы $K_{сч}$ был отличным от $2^n$, например широко распространены счётчики с $K_{сч} = 10$.
Такой счётчик после каждого десятого импульса должен возвращаться в исходное состояние, формируя на своём выходе импульс переноса.
Разрядность счётчика с произвольным коэффициентом пересчёта определяется из условия: $2^{n-1}<K_{сч}<2^n$.

Очевидно, что для счётчика с $K_{сч} = 10$ $n=4$.
Но этот счётчик будет иметь $2^4=16$ устойчивых состояний.
Следовательно, в счётчике с $K_{сч} =10$ $N=6$ состояний являются избыточными, и их надо исключить.
Обычно это делается введением обратных связей с выхода счётчика на единичные входы триггеров тех разрядов, которые в двоичном представлении числа $N$ имеют единицы.
Например: в нашем случае $N=6_{10}=0110_2$, т.\,е. сигнал обратной связи необходимо подать на $S$-выходы второго и третьего разрядов.
(((Картинко)))

Десятичный счётчик, или делитель частоты с модулем пересчёта $K_{сч}=10$ работает следующим образом: входные импульсы поступают на входы $C$ триггеров.
Счётчик отсчитывает последовательно входные импульсы от исходного кода \texttt{0110} до кода \texttt{1111}, после чего очередной (десятый) импульс появится на выходе счётчика в качестве импульса переноса и установит счётчик но цепям обратной связи в исходное состояние \texttt{0110}.
\begin{longtable}{|c|c|c|c|c|}
\caption{Таблица переходов десятичного счётчика}\\
\hline
\textnumero~сигнала	&	\textbf{Тг$_4$}	&	\textbf{Тг$_3$}	&	\textbf{Тг$_2$}	&	\textbf{Тг$_1$}\\
\hline
1	&	0	&	1	&	1	&	1\\
2	&	1	&	0	&	0	&	0\\
3	&	1	&	0	&	0	&	1\\
4	&	1	&	0	&	1	&	0\\
5	&	1	&	0	&	1	&	1\\
6	&	1	&	1	&	0	&	0\\
7	&	1	&	1	&	0	&	1\\
8	&	1	&	1	&	1	&	0\\
9	&	1	&	1	&	1	&	1\\
10	&	0	&	1	&	1	&	0\\
\hline
\end{longtable}
\subsection{Дешифраторы}
\textbf{Дешифратор} --- комбинационное устройство, преобразующее поступающей на его входы кодовое число в управляющий сигнал только на одном из выходов.
С помощью дешифраторов двоичный код числа преобразуется в сигнал, управляющий выбором соответствующего блока, схемы или устройства ЭВМ, например, выборка необходимых ячеек памяти, расшифровка кодов операций с выдачей управляющих сигналов в те элементы, узлы и устройства ЭВМ, которые связаны с выполнением данной операции и т.\,д.

Схема декодирования (дешифрации) представляет собой матрицу логических элементов и содержит в общем случае $n$ входов и $2^n$ выходов.
Каждой комбинации входных сигналов соответствует появление сигналов на одном из выходов.
Функционирование дешифратора, имеющего $n=3$ входов и $P=2^3=8$ выходов можно описать таблицей состояний, из которой соответствует, что каждому набору входных переменных $x_1, x_2, x_3$ соответствует единичный сигнал только на одном выходе $P_i$ дешифратора.
\begin{longtable}{|c|c|c|c|c|c|c|c|c|c|c|}
\caption{Таблица состояний дешифратора, имеющего $n=3$ входов}\\
\hline
\multicolumn{3}{|c|}{Входы}	&	\multicolumn{8}{|c|}{Выходы}\\
\hline
$x_1$	&	$x_2$	&	$x_3$	&	$P_0$	&	$P_1$	&	$P_2$	&	$P_3$	&	$P_4$	&	$P_5$	&	$P_6$	&	$P_7$\\
\hline
0	&	0	&	0	&	1	&	0	&	0	&	0	&	0	&	0	&	0	&	0\\
0	&	0	&	1	&	0	&	1	&	0	&	0	&	0	&	0	&	0	&	0\\
0	&	1	&	0	&	0	&	0	&	1	&	0	&	0	&	0	&	0	&	0\\
0	&	1	&	1	&	0	&	0	&	0	&	1	&	0	&	0	&	0	&	0\\
1	&	0	&	0	&	0	&	0	&	0	&	0	&	1	&	0	&	0	&	0\\
1	&	0	&	1	&	0	&	0	&	0	&	0	&	0	&	1	&	0	&	0\\
1	&	1	&	0	&	0	&	0	&	0	&	0	&	0	&	0	&	1	&	0\\
1	&	1	&	1	&	0	&	0	&	0	&	0	&	0	&	0	&	0	&	1\\
\hline
\end{longtable}

Из этой таблицы можно записать логические функции, реализуемые каждым выходом дешифратора:

\begin{tabular}{ll}
$P_0=\bar x_1 \bar x_2 \bar x_3 $	&	$P_4=\bar x_1 \bar x_2 x_3$\\
$P_1=x_1\bar x_2 \bar x_3 $	&	$P_5=x_1\bar x_2 x_3$\\
$P_2=\bar x_1 x_2\bar x_3$	&	$P_6=\bar x_1 x_2 x_3$\\
$P_3=x_1 x_2\bar x_3$	&	$P_7=x_1 x_2 x_3$\\
\end{tabular}

Функциональная схема дешифратора, построенном на этих уравнениях, содержит восемь элементов \textbf{И} и три инвертора для получения инверсных значений входных переменных $\bar x_1, \bar x_2,\bar x_3$.
Схемы \textbf{И} дешифраторов могут быть реализованы различными способами.
Ранее наиболее распространёнными были диодные дешифраторы. В настоящее время они могут быть реализованы на основных логических схемах в интегральном исполнении.
Существуют также отдельные микросхемы в виде дешифраторов.

(((Картинко---функциональная схема)))

Принципиальная схема диодного дешифратора на три входа имеет следующий вид: (((Картинко)))

Входной код поступает и хранится в трёхразрядном регистре на триггерах.
С этих триггеров можно получить значения входных переменных $x_1, x_2, x_3$ в прямом и обратном кодах.
С каждой выходной шиной дешифратора связаны 3 диода, поэтому сигнал появится только на той шине, из которой будут заперты все 3 диода.
Если же будет открыт хотя бы один диод, то потенциал этой шины будет низким, т.\,к. потечет ток через открытый диод, и напряжение источника питания $+E_{пит}$ будет практически полностью падать на резисторе $R$ данной выходной шины.
Фактически для каждой функции $P$ реализованы конънкция переменных $x_1, x_2, x_3$ и их инверсий.

Пусть, например, на вход дешифратора поступила комбинация сигналов \texttt{101}, т.\,е. $x_1=1; x_2=0;x_3=1$.
При этом высокие потенциалы будут сниматься с единичных выходов $Q_1$ и $Q_3$ крайних триггеров и с нулевого выхода среднего триггера.
Следовательно, все 3 диода, связанные с шиной $P_5$ будут заперты, и с неё будет сниматься высокий управляющий сигнал.
Можно убедиться, что при этом на остальных шинах будет отсутствовать управляющий сигнал, т.\,к. для любой из этих шин будет открыт, по крайней мере, один диод.

Рассмотренная схема относится к \textit{линейным дешифраторам}.
На входе схемы --- парафазный входной сигнал, на выходе --- сигнал в однофазном коде.

Дешифраторы относятся к избирательным комбинационным схемам и работают совместно с регистрами, счётчиками и другими устройствами.

С увеличением числа входов усложняется построение дешифраторов за счёт множества соединений.
Поэтому линейные дешифраторы содержат не более, чем на 3--5 входов.
Обычные ИС позволяют в одном корпусе иметь дешифратор на 3--4 входа, в БИС --- на 5--6 входов.

Ограничение числа входов и выходов связано с тем, что у ИС и БИС коэффициенты объединения и разветвления имеют пределы.

На функциональных схемах дешифраторы изображаются следующим образом:(((КАРТИНКО)))

Входы дешифраторов обозначены десятичными числами, являющимися весами двоичных разрядов, а выходы --- десятичными числами, соответствующими кодовым комбинациям на входе ($P=2^{n-1}; m=2^n-1$).
Если число выходов $m$ дешифратора равно $2^n$, то дешифратор называют \textit{полным}, а если используется только часть выходов, то --- \textit{неполным}.

Линейные дешифраторы быстродействующие, но при большом числе входов их применение становится неэкономичным с точки зрения аппаратурных затрат.
Более экономичными, в особенности при большом числе входов, являются \textit{многоступенчатые} и \textit{пирамидальные} дешифраторы.
\subsection{Сумматоры}
\textbf{Сумматор} --- основной узел АЛУ ЭВМ, предназначенный для суммирования чисел поразрядным сложением.

В зависимости от элементной базы сумматоры могут быть \textit{комбинационными} и \textit{накапливающими}.
\begin{itemize}
\item \textbf{Сумматоры комбинационного типа} строятся на комбинационных логических элементах \textbf{И, ИЛИ, НЕ}.
В таких сумматорах слагаемые подаются одновременно, при этом на выходе образуется их сумма.
Комбинационные сумматоры не обладают запоминающей способностью, поэтому после прекращения действия входных сигналов исчезает на выходе результат суммирования.
Поэтому на выходе такого сумматора имеется  регистр, куда каждый раз производится запись результата суммирования.
\item \textbf{Накапливающие сумматоры} реализуются на триггерах, на счётные входы которых подаются разряды слагаемых.
Суммируемые числа поступают на вход сумматора по очереди, одно за другим, а результат суммирования запоминается в сумматоре.
\end{itemize}
\subsubsection{Сумматоры комбинационного типа}
В таких сумматорах процесс суммирования производится поразрядно, причём в каждом разряде необходимо выполнить суммирование двух разрядов слагаемых, а также переноса из соседнего младшего разряда.
Т.\,е., в общем случае в каждом разряде требуется выполнить суммирование трёх двоичных цифр.
Однако часто такое суммирование разбивается на две аналогичные элементарные операции: суммирование двух разрядов слагаемых и суммирование полученного результата с переносом из соседнего разряда.
Каждая из этих операций выполняется схемой, называемой \textit{полусумматором} --- устройством для суммирования двух одноразрядных двоичных чисел.

На выходах полусумматора со входами $x_i$ и $y_i$ образуется сумма $S_i$ и перенос $P_{i+1}$ в следующий разряд.
Логика работы полусумматора задаётся таблицей истинности:

\begin{tabular}{|c|c||c|c|}
\hline
$x_i$	&	$y_i$	&	$S_i$	&	$P_{i+1}$\\
\hline
0	&	0	&	0	&	0\\
0	&	1	&	1	&	0\\
1	&	0	&	1	&	0\\
1	&	1	&	0	&	1\\
\hline
\end{tabular}

Уравнение для $S_i$ и $P_{i+1}$: $S_i=\bar{x}_iy_i+x_i\bar{y}_i$; $P_{i+1}=x_iy_i$

(((Функциональная схема полусумматора:)))

Преобразуя выражение для суммы $S_i$ к виду:\[S_i=\bar{x}_iy_i+x_i\bar{y}_i=\bar{x}_iy_i+x_i\bar{y}_i+y_i\bar{y}_i+x_i\bar{x}_i=\bar{x}_i\left(\mathstrut x_i+y_i\right)+\bar{y}_i\left(\mathstrut x_i+y_i\right)=\left(\mathstrut x_i+y_i\right)\left(\mathstrut\bar{x}_i+\bar{y}_i\right)=\overline{\mathstrut x_iy_i}\left(\mathstrut x_i+y_i\right)\], получим схему: (((СХЕМКО)))

Условное обозначение одноразрядного полусумматора:(((())))

\textit{Сумматоры} отличаются от полусумматоров тем, что можно сложить три одноразрядных двоичных числа, что позволяет учитывать единицу переноса от предыдущего разряда.
Логика работы сумматора описывается таблицей состояний, в которой $x_i, y_i$ --- суммируемые двоичные цифры в $i$-том разряде, $P_{i-1}$ --- перенос из младшего разряда, $P_{i+1}$ --- перенос в старший разряд.

\begin{tabular}{|c|c|c||c|c|}
\hline
$P_{i-1}$	&	$x_i$	&	$y_i$	&	$S_i$	&	$P_{i+1}$\\
\hline
0	&	0	&	0	&	0	&	0\\
0	&	0	&	1	&	1	&	0\\
0	&	1	&	0	&	1	&	0\\
0	&	1	&	1	&	0	&	1\\
1	&	0	&	0	&	1	&	0\\
1	&	0	&	1	&	0	&	1\\
1	&	1	&	0	&	0	&	1\\
1	&	1	&	1	&	1	&	1\\
\hline
\end{tabular}
или следующие уравнения:\[S_i=\bar{P}_{i-1}\bar{x}_iy_i+\bar{P}_{i-1}x_i\bar{y}_i+P_{i-1}\bar{x}_i\bar{y}_i+P_{i-1}x_iy_i;\]
\[P_{i+1}=x_iy_i+x_iP_{i-1}+y_iP_{i-1}\]
((((Функциональная схема))))
(((((Условное обозначение на функциональных схемах:)))))


Кроме того, полный одноразрядный сумматор на три входа может быть построен на основе двух полусумматоров и логического элемента \textbf{ИЛИ}:(((((КАРТИНКО)))))

В зависимости от характера ввода-вывода чисел и организации переносов многоразрядные сумматоры, в качестве которых используются одноразрядные, бывают последовательного и параллельного принципов действия.

В \textit{последовательном сумматоре} сложение чисел осуществляется поразрядно, начиная с младшего разряда, с помощью сумматора на 3 входа.
Образующийся в данном разряде перенос $P_{i+1}$ задерживается на время $t_{з}$ и поступает на вход $P_{i-1}$ сумматора в момент прихода следующего разряда слагаемых.
Т.\,е., производится последовательное сложение чисел разряд за разрядом.
Достоинством последовательного сумматора является простота аппаратной реализации, а недостатком --- большое время суммирования.

Ускорение операции сложения может быть достигнуто в \textit{параллельном сумматоре}.
Суммируемые коды подаются на входы сумматора одновременно по всем разрядам.
Для этого в ((((((((((((((((((())))))))))))))))))) на три входа, вырабатывающий на своих выходах сигналы суммы $S_i$ в данном разряде и переноса $P_{i+1}$ в старший разряд.
В процессе распространения переноса будет устанавливаться окончательное значения суммы в каждом разряде.
Следовательно, в течении этого времени на входах сумматора должны присутствовать сигналы, соответствующие суммируемым кодам.
Максимальное время суммирования будет для таких кодов, при которых перенос возникающий в первом разряде, распространяется по всем разрядам.
Сокращение времени суммирования в этом случае достигается сквозной передачей сигналов переноса из разряда в разряд.
\subsubsection{Накапливающие сумматоры}
Их основой являются триггеры со счётным входом. ((((((((((((((Схема одноразрядного накапливающего сумматора:))))))))))))))

Включает в себя счётный триггер и необходимые логические элементы для формирования переноса.
Слагаемые $x_i,y_i$ последовательно поступают через логический элемент \textbf{ИЛИ} на счётный вход триггера через временной интервал, достаточный для завершения переходных процессов в триггере.
Затем через $\tau_1$ аналогичным образом поступает перенос $P_{i-1}$ из младшего разряда.
В результате суммирования $x_i,y_i,P_{i-1}$ на единичном выходе триггера образуется сумма $S_i$ данного разряда.
Для формирования переноса $P_{i+1}$ в старший разряд используется дополнительная схема на логических элементах.

\textit{Многоразрядный сумматор накапливающего типа} представляет цепочку таких одноразрядных схем, у которых цепи переносов соединены последовательно через элементы задержки.
После суммирования ((((((((((((())))))))))))).

\subsection{Мультиплексоры}
В тех случаях, когда требуется последовательно опросить логические состояния многих устройств и передать их на один выход, применяют устройство, называемое \textit{мультиплексорам}\footnote{от англ. \textbf{multiplex} --- многократный}.
Ниже приведена схема мультиплексора с 2-мя информационными входами ($x_0,x_1$) и управляющим (адресным) входом $a$; рядом --- эквивалентная схема мультиплексора ((((((((((((Туча рисунков))))))))))))

При $a=1$ на выход передаётся значение $x_1$, а при $a=0$ --- значение $x_0$.
$y=\bar ax_0+ax_1$

((((((((((((((((((((Схема и условные обозначения мультиплексора на 4 входа($x_0,x_1,x_2,x_3$))))))))))))))))))))

Она имеет 2 адресных входа: $a_0$ и $a_1$. Из схемы следует, что $y=\bar a_1\bar a_0x_0+\bar a_1a_0x_1+a_1\bar a_0x_2+a_1a_0x_3$.

Например, если $a_1=1,a_0=0$, то $y=0\cdot1\cdot x_0+0\cdot0\cdot x_1+1\cdot1\cdot x_2+1\cdot0\cdot x_3=x_2$.
Действительно, при $a_1=1,a_0=0$ получаем $10_2=2_{10}$. В данном случае \texttt{2} --- номер опрашиваемого входа.
Нетрудно проверить, что например при $A=11_2=3_{10}$ $y=x_3$.

Мультиплексоры выпускают в виде микросхем, например \texttt{К155КП2} (четырёхканальный мультиплексор $4\times1$), или \texttt{К155КП1} (16-канальный мультиплексор $16\times1$).

\end{document}
