% vim:spelllang=uk
\documentclass[a4paper,10pt,notitlepage,pdftex,headsepline]{scrartcl}

\usepackage{cmap} % чтобы работал поиск по PDF
\usepackage[utf8]{inputenc}
\usepackage[ukrainian]{babel}
\usepackage[T2A]{fontenc}
\usepackage{concrete}
\usepackage{fullpage}

\pdfcompresslevel=9 % сжимать PDF
\usepackage{pdflscape} % для возможности альбомного размещения некоторых страниц
\usepackage[pdftex]{hyperref}
% настройка ссылок в оглавлении для pdf формата
\hypersetup{unicode=true,
            pdftitle={Українська мова за професійним спрямуванням},
            pdfauthor={Михайло Погода},
            pdfcreator={pdflatex},
            pdfsubject={},
            pdfborder    = {0 0 0},
            bookmarksopen,
            bookmarksnumbered,
            bookmarksopenlevel = 2,
            pdfkeywords={},
            colorlinks=true, % установка цвета ссылок в оглавлении
            citecolor=black,
            filecolor=black,
            linkcolor=black,
            urlcolor=blue}

\author{Михайло Погода}
\title{Українська мова за професійним спрямуванням}
\date{\today}

\begin{document}
  \maketitle

\section{Наукова комунікація як складова іспиту}
  Підручники:
  \begin{itemize}
      \item Зоряна Мацюк, Ніна Станкевич ,,Українська мова професійного
          спілкування'', КИЇВ 2006
      \item Світлана Шевчук, Ірина Клеменко ,,Українська мова за професійним
          спрямуванням'', КИЇВ 2013
  \end{itemize}

  Виступ, який ґрунтується не на одному джерелі.

  \subsection{Українська термінологія в професійному спілкуванні}
      \subsubsection{Історичні і сучасні проблеми українського термінознавства}
      \subsubsection{Терміни, професіоналізми та їхні ознаки}
      \subsubsection{Загальнонаукова, міжгалузева та вузькоспеціальна
      термінологія}
      \subsubsection{Кодифікація та стандартизація термінів. Алгоритм укладання
      термінологічного стандарту}
      \subsubsection{Типи термінологічних словників}
      \subsubsection{Словники в професійному мовленні}
      \subsubsection{Структурні особливості словникових статей}
      \subsubsection{Українські електронні термінологічні словники}

  \subsection{Науковий стиль і його засоби у професійному мовленні}
    \subsubsection{Розвиток наукового стилю української літературної мови}
    \subsubsection{Особливості наукового тексту та професійного наукового
    викладу думок}
    \subsubsection{Науковий етикет}
  \subsection{Жанри наукових досліджень}
    \subsubsection{Оформлення результатів наукової діяльності}
    \subsubsection{План, тези, конспект як важливий засіб організації
    розумової діяльності}
    \subsubsection{Анотування та реферування наукових текстів}
    \subsubsection{Основні правила бібліографічного опису, оформлення запитів}


    Особливості відмінювання кількісних числівників і узгодження числівників з
    іншими частинами мови.

    Стосунки, взаємини, відносини, ставлення, відношення.
    \subsubsection{Реферат як жанр академічного письма. Складові реферату}
    \subsubsection{Стаття як самостійний науковий твір. Вимогу ВАК України до
    наукової статті}
    \subsubsection{Основні вимоги до виконання й оформлення курсової та
    бакалаврської робіт}
    \subsubsection{Рецензія. Відгук}
    \subsection{Проблема ё}
    \subsubsection{Форми та види перекладів: буквальний переклад,
    реферативний переклад, адекватний, анотаційний, технічний, літературний}
    \subsubsection{Калькування елементів близькоспоріднених мов}
    \subsubsection{Вибір синонімів підчас перекладу}
    \subsubsection{Переклад термінів}
    \subsubsection{Особливості редагування наукового тексту}
    \subsubsection{Специфіка комп’ютерного перекладу
\end{document}
