% vim:spelllang=ru,en
\documentclass[a4paper,12pt,notitlepage,headsepline,pdftex]{scrartcl}

\usepackage{cmap} % чтобы работал поиск по PDF
\usepackage[T2A]{fontenc}
\usepackage[utf8]{inputenc}
\usepackage[english,russian]{babel}
\usepackage{concrete}
\usepackage{cite}
\usepackage{url}

\usepackage{textcase}
\usepackage[pdftex]{graphicx}

\usepackage{lscape}

\pdfcompresslevel=9 % сжимать PDF
\usepackage{pdflscape} % для возможности альбомного размещения некоторых страниц
\usepackage[pdftex]{hyperref}
% настройка ссылок в оглавлении для pdf формата
\hypersetup{unicode=true,
            pdftitle={PostgreSQL},
            pdfauthor={Погода Михаил},
            pdfcreator={pdflatex},
            pdfsubject={},
            pdfborder    = {0 0 0},
            bookmarksopen,
            bookmarksnumbered,
            bookmarksopenlevel = 2,
            pdfkeywords={},
            colorlinks=true, % установка цвета ссылок в оглавлении
            citecolor=black,
            filecolor=black,
            linkcolor=black,
            urlcolor=blue}

\usepackage{amsmath}
\usepackage{amssymb}
\usepackage{moreverb}
\usepackage{indentfirst}
\usepackage{misccorr}

\usepackage{xtab}
\usepackage{nccfoots}

\begin{document}
\begin{titlepage}
  \begin{center}
    \large
    \MakeUppercase{Министерство образования и науки,}

    \MakeUppercase{молодёжи и спорта Украины}

    \mbox{\MakeUppercase{Национальный технический университет Украины}}

    \MakeUppercase{,,Киевский политехнический институт''}

    \addvspace{6pt}

    \normalsize
    Кафедра прикладной математики

    \vfill

    \textbf{Реферат}

    по дисциплине ,,Микропроцессорные системы''

    на тему: \textit{,,Intel Xeon E5''}
  \end{center}

  \vfill

  \noindent
  Выполнил\\
  студент группы КМ-92\\
  Погода~М.\,В.\\
  \vfill

  \begin{center}
    КИЕВ

    2013
  \end{center}
\end{titlepage}

\section{Intel Xeon}
Intel Xeon --- линейка серверных микропроцессоров производства Intel.
Название остаётся неизменным для нескольких поколений процессоров.
Название ранних моделей состояло из соответствующего названия из ряда
настольных процессоров и слова Xeon, современные модели имеют в названии
только Xeon.
В общих чертах серверная линейка процессоров отличается от настольной
увеличенным кэшем и поддержкой больших многопроцессорных систем.
Также Pentium II Xeon в отличие от ,,десктопного'' Pentium II имел кэш второго
уровня, работающий на полной частоте ядра, а не на половине.

Intel базирует процессоры Xeon E5 на таком же кристалле, что и CPU серии Core
i7-3000.
Максимум для линейки Core i7 --- это шесть ядер и общий кэш L3 на 15 Мбайт.
Но на этом же чипе располагаются восемь ядер и 20 Мбайт кэша.


Модульный принцип этого дизайна реализован через концепцию кольцевой шины
(если точнее, первыми CPU с кольцевой шиной были процессоры Xeon 7500).
У вас есть ядра, контроль над PCI Express, QPI (QuickPath Interconnect) и
четырёхканальный контроллер памяти, и всё это связано с кольцевой шиной.
Поскольку каждое ядро привязано к 2,5 Мбайт кэша L3, манипулировать
характеристиками кристалла относительно просто, и при этом создавать больше
количество моделей, производительность которых варьируется вполне
предсказуемо.

Для таких процессоров, как Core i7-3960X, Intel просто обрезает два ядра и,
соответственно, по 2,5 Мбайт кэша.
Но кэш можно настраивать ещё тоньше.
У некоторых моделей Xeon E5 представлено по 2 Мбайт на ядро, это говорит о
том, что его можно настраивать блоками по 512 кбайт.

  Новая модель Xeon отличается не только увеличенным количеством ядер, но и
  изменениями микроархитектуры (Sandy Bridge Efficient Perfomance).
\section{Sandy Bridge}
Sandy Bridge --- микроархитектура центральных процессоров, разработанная
фирмой Intel. Основана на 32-нм технологическом процессе, содержит встроенный
видеоускоритель.
Анонсирована 3 января 2011 года.

Новая микроархитектура несёт поддержку новых SIMD (инструкций для работы с
векторными вычислениями Advanced Vector Extensions, AVX), которые дополнят
расширения SSE (новый набор, оставаясь обратно совместимым с SSE, увеличивает
разрядность рeгистров в два раза --- до 256 бит, а также даёт в распоряжение
программистов дополнительные трёх- и четырёхоперандные команды).
При этом Intel обещает, что использование AVX будет способно поднять скорость
работы некоторых алгоритмов на величину, достигающую 90\,\%.

Также это первая архитектура, в которую Intel встроил технологию Quick Sync,
предназначенную для ускорения кодирования и декодирования видеоконтента.
Реализована в виде специализированных аппаратных модулей в составе
графического ядра.

Поддерживаются технологии Advanced Encryption Standard (AES) и Virtualization
Machine Extensions (VMX).

\subsection{Процессорные ядра}
  \begin{enumerate}
    \item Добавлен кэш инструкций нулевого уровня (L0), содержащий до 1500
      декодированных микроопераций для экономии энергии и улучшения пропускной
      способности инструкций.
      Блок предварительной выборки может отключать декодер инструкций, если он
      обнаруживает уже декодированную инструкцию в кэше.
    \item Оптимизирована и улучшена точность предсказателя переходов.
    \item Advanced Vector Extensions (Intel AVX).
      AVX --- это расширение набора инструкций SSE, в два раза - до 256 бит
      увеличивающее размер регистра для обработки векторных инструкций (пока
      только для работы с float point, но с планируемым расширением
      функциональности) и обеспечивающее поддержку инструкций с тремя
      операндами.
    \item Технология Intel Turbo Boost 2.05.
      Эта технология позволяет увеличить тактовую частоту в случае, когда
      задействованы не все вычислительные ядра.
      В предыдущем поколении процессоров Xeon 5690 одно активное ядро могло
      увеличить свою частоту на 266 МГц, а в процессоре E5-2690 --- до 900 МГц.
    \item Технология Intel Integrated I/O (Intel IIO).
      Intel IIO позволяет переместить контроллер подсистемы ввода-вывода с
      отдельной микросхемы на системной плате непосредственно в процессор, тем
      самым сокращающая задержки в сети и устройствах хранения данных в
      виртуальных системах, и поддерживающую новейшую спецификацию PCI
      Express* 3.0 для работы с приложениями, требующими высокой пропускной
      способности.
    \item Технология Intel Data Direct I/O (Intel DDIO).
      Эта технология позволяет периферийным устройствам, например,
      контроллерам Ethernet, направлять трафик подсистемы ввода-вывода
      непосредственно в кэш-память процессора.
      В результате снижается объем данных, передаваемых в системную память,
      энергопотребление и задержка подсистемы ввода-вывода.
    \item Набор инструкций Intel Advanced Encryption Standard --- New
      Instructions (Intel\copyright AES-NI).
      Представляет собой обновленную технологию аппаратного ускорения
      шифрования по стандарту Advanced Encryption Standard (AES).
  \end{enumerate}

\subsection{Технологии удаленного управления и DRM}
  В Sandy Bridge имеется блок DRM (ТСЗАП) под названием ,,Intel Insider''.
  Компания Intel заявляет что он является ,,дополнительным уровнем защиты
  контента,,.

  В процессорах Sandy Bridge с функцией vPro имеется возможность удаленного
  управления, например удаленного блокирования ПК или стирания информации с
  НЖМД.
  Заявлено, что подобные функции полезны в случае кражи ПК.
  Команды могут быть переданы при помощи 3G, Ethernet, или другого подключения
  к сети Интернет.

\subsection{Структура}
  Структуру чипа Sandy Bridge можно условно разделить на следующие основные
  элементы:
  \begin{itemize}
    \item процессорные ядра;
    \item графическое ядро;
    \item кеш-память L3;
    \item ,,Системный агент''.
  \end{itemize}

  Все перечисленные элементы объединены с помощью 256-битной межкомпонентной
  кольцевой шины, выполненной на основе новой версии технологии QPI.

  Шина состоит из четырёх 32-байтных колец:
  \begin{itemize}
    \item шины данных (англ. Data Ring);
    \item шины запросов (англ. Request Ring);
    \item шины мониторинга состояния (англ. Snoop Ring);
    \item шины подтверждения (англ. Acknowledge Ring).
  \end{itemize}

  \emph{Основные преимущества кольцевой топологии шины}:
  \begin{itemize}
    \item высокая масштабируемость (до 20 ядер на кристалл);
    \item снижение задержки кэша 3-го уровня, и перевод его на частоту
      процессора;
    \item использование графическим ядром кэша 3-го уровня.
  \end{itemize}

  Производительность кольцевой шины достигает 96 Гбайт в секунду на соединение
  при тактовой частоте 3 ГГц, что фактически в четыре раза превышает
  показатели процессоров Intel предыдущего поколения.

\section{Технические характеристики}
  Существует 37 различных моделей Xeon E5.
  Флагманом линейки процессоров является Xeon E5-2687W ---
  процессор исключительно для рабочих станций мощностью 150 Вт, использующий
  все восемь физических ядер, кэш на 20 Мбайт, два соединения QPI 8 GT/s, 40
  линий PCIe третьего поколения на кристалле и четырёхканальный контроллер
  памяти поддерживающий DDR3-1600 (предыдущее поколение Intel Xeon
  поддерживает трехканальный DDR3-1333).
  Объем поддерживаемой памяти составляет 768 ГБайт.
  Эта высокоинтегрированная система на кристалле (SOC) с техпроцессом 32 нм,
  состоит из 2,27 миллиарда транзисторов, упакованных в громоздкий кристалл
  площадью 434 мм$^{2}$.

  Есть ещё одно заметное различие между односокетными процессорами Core
  i7/Xeon E5-1600 и мультисокетными платформами Intel --- это наличие QPI.
  Когда компания Intel заменила процессоры на базе Gulftown архитектурой Sandy
  Bridge-E, она одновременно переключилась с платформ из трёх частей (CPU,
  северный мост и южный мост) на двухчиповую раскладку (CPU и PCH), при этом
  убирая хаб ввода/вывода, отвечающий за линии PCI Express.
  Соединение между процессором и северным мостом, которое обеспечивал QPI,
  было разорвано.
  С интеграцией PCIe прямо в процессор, южный мост можно привязать к CPU через
  интерфейс Direct Media, похожий на PCI Express.
  Таким образом, QPI полностью отключён в чипах Sandy Bridge-E.

  Однако мультисокетовые системы по-прежнему нуждаются в QPI для
  межпроцессорной передачи.
  Процессоры Sandy Bridge-EP используют два соединения QPI.
  В конфигурации 2S они оба используются для транспортировки данных в обе
  стороны между сокетами.
  При работе с четырьмя процессорами они создают что-то вроде круга, передавая
  данные влево и вправо.
  Intel пытается представить скорость передачи данных QPI как дифференцирующую
  функцию, но тогда как максимум Xeon 5600 составляет 6.4 GT/s, что даёт 25.6
  Гбит/с на соединение, более современный Xeon E5 имеет соединения по 8 GT/s и
  пропускная способность поднимается до 32 Гбит/с на соединение.
  Очевидно, что в рабочих станциях 2S, 64 Гбит/с агрегированной пропускной
  способности более чем достаточно.
  Таким образом, ,,узкие места'' вызванные шиной убраны.

  Кроме количества ядер, кэша L3 и QPI, Sandy Bridge-EP архитектурно
  соответствует Sandy Bridge-E.
  Поддержка AVX, AES-NI, Turbo Boost второго поколения, Hyper-Threading ---
  всё это присутствует в новой линейке.

  Стоит отметить ещё одну особенность архитектуры Sandy Bridge-EP: её
  четырёхканальный контроллер памяти поддерживает зеркальное резервирование,
  аппаратную коррекцию ошибок и функцию lockstep.
  Все три функции были доступны на процессорах Xeon 5500/5600, но
  трёхканальная архитектура контроллера памяти требовало компромиссов.
  Сейчас вы можете дублировать два канала и в каждом восстановиться после
  сбоя.



  Архитектура Sandy Bridge наделала много шума в сфере настольных процессоров.

  Новая линейка E5 охватывает больше сегментов рынка.
  Теперь есть восьмиядерные процессоры для двухсокетных систем начального
  уровня, представленные в семействе Xeon E5-2400.
  Линейка Xeon E5-1600 для односокетных рабочих станций предлагает
  функциональность уровня Core i7-3000, но с дополнительными функциями RAS,
  которые важны для некоторых пользователей.
  Процессоры Xeon E5-2600 охватывают широкий спектр серверов и рабочих станций
  2S.
  А линейка процессоров Xeon E5-4600 реализует идею массовых четырёхсокетных
  конфигураций, максимизируя производительность на ватт в PCH-окружениях.

\end{document}
