\documentclass[a5paper,10pt,titlepage,pdftex,headsepline]{scrartcl}

\usepackage{cmap} % чтобі работал поиск по PDF
\usepackage[utf8]{inputenc}
\usepackage[ukrainian]{babel}
\usepackage[T2A]{fontenc}

\usepackage{textcase}

\pdfcompresslevel=9 % сжимать PDF
\usepackage{pdflscape} % для возможности альбомного размещения некоторіх страниц
\usepackage[pdftex]{hyperref}
% настройка ссілок в оглавлении для pdf формата
\hypersetup{unicode=true,
            pdftitle={Конспект з історії української культури},
            pdfauthor={Погода Михайло},
            pdfcreator={pdflatex},
            pdfsubject={},
            pdfborder    = {0 0 0},
            bookmarksopen,
            bookmarksnumbered,
            bookmarksopenlevel = 2,
            pdfkeywords={},
            colorlinks=true, % установка цвета ссілок в оглавлении
            citecolor=black,
            filecolor=black,
            linkcolor=black,
            urlcolor=blue}

\author{Михайло Погода}
\title{Конспект з історії української культури}
\date{\today}
\begin{document}
\maketitle
\newpage
\tableofcontents
\newpage

\section{Культурологія в системі гуманітарних знань}
\subsection{Від побутових уявлень до наукового поняття ,,культура''}
Що не натура, то культура.

Культура (лат. culture --- ,,обробіток'', ,,обробляти'') --- сукупність матеріальних і духовних, нематеріальних цінностей, створених протягом його історії.

У класичній давнині слово ,,культура'' вперше було зафіксовано в праці Марка Порція Катона ,,De agri cultura''(III ст.~до н.\,е.)

Цицерон казав, що культура --- обробка душі  людини, предметом для обробки є філософія.

Цицерон:
\begin{quote}
  Культура --- вдосконалення душі за допомогою риторики.
\end{quote}

Вміння співвідносити особисті і громадські інтереси --- показник громадськості.

\begin{itemize}
\item Античність: знаю, тому що вірю. Блаженні вбогі духом, бо ім належить Царство Небесне.
\item Відродження: вірю, тому що знаю.
\item Просвітництво: XVIII ст. В цю добу до культури почали відносити не лише матеріальні, а й духовні цінності.
\item XIX ст. Починають розвиватись природничі науки: фізика, хімія, біологія.
  Формується етнографія.
  Етнографи: суспільство розвивається за законами природи.

Морган виділив такі етапи:
\begin{enumerate}
\item дикість;
\item варварство;
\item цивілізація.
\end{enumerate}

Голова етнограф. школи --- Едвард Тайлор першим дав визначення поняття ,,культура'' --- культура або цивілізація в широкому етнографічному значенні складається в цілому зі знання, вірувань, мистецтва, моралі, законів, звичаїв та деяких інших особливостей, засвоєних людиною, як членом суспільства.

\item Середина 20-х: батько культури --- антрополог Леслі Уайт.
  Він вважав, що поняття ,,культура'' з'ясовує:
\begin{itemize}
\item сутність взаємодії людини з навколишнім світом;
\item поведінка людини --- реакція;
\item основи людського існування --- операції з символами.
\end{itemize}
\end{itemize}

Уайт:
\begin{quote}
  Сутність культури полягає не в праці, або свідомості, а в умінні передавати досвід за допомогою символів\dots
\end{quote}

Екстрасоматична традиція --- культура.

У системі культури виділяють три підсистеми:
\begin{itemize}
\item технологічна підсистема, що включає матеріальні пристрої для спілкування культури;
\item соціальна підсистема, що визначає способи і типи поведінки в суспільстві;
\item ідеологічна підсистема, що складається з ідей, образів, вірувань.
\end{itemize}

\subsection{Про предмети и методи культурології. Основні напрями в сучасній культурології. Монізм и плюралізм}
Основні напрями в сучасній культурі:
\begin{enumerate}
  \item \textit{Позитивістський}: культура розвивається за певними законами;
  \item \textit{Філософія культури} (школа Віндельбанда, Ріккерта, Дильтея): треба розрізняти науки про природу і дух (уособлення культури)

Якщо природні явища визначаються законами Всесвіту, то люди в своїй поведінці  керуються певними безумовними нормами та принципами:
\begin{itemize}
\item істина;
\item добро;
\item краса;
\item святість.
\end{itemize}

Дильтей першим проголосив, що явище ,,культури'' слід пізнавати лише занурюючись у них, як у життєву цілісність.

Дильтей:
\begin{quote}
  Природу ми пояснюємо, а культуру --- розуміємо.
\end{quote}

  \item \textit{Циклічна концепція} (Данилевський, Шпенглер, Тойнді).
\end{enumerate}

Монізм --- визнання існування однієї культури, в яку входять всі культури.

Плюралізм --- визнання кожної культури, як окремої к-ри, зі своїм окрем шляхом розвитку.

Данилевський першим ввів поняття ,,культурно історичний тип''.
За Динилевським кожен етап розвитку к-ри проходить як у рослини: народження, зростання, розквіт, засихання.

історично-культурні типи за Данилевським:
\begin{itemize}
  \item єгипетський,
  \item Китайський,
  \item Асиро-Вавілонський,
  \item Давньосилітський,
  \item іранський,
  \item єврейський,
  \item Грецький,
  \item Римський,
  \item Аравійський,
  \item європейський.
\end{itemize}

Данилевський сформував 5 законів циклічного типу:
\begin{enumerate}
\item мова;
\item незалежність;
\item окрема цивілізація;
\item різноманітні етнограф. типи у цивілізації;
\item етапи розвитку, як у рослини.
\end{enumerate}

Шпенглер визначив 3 стадії розвитку культури:
\begin{enumerate}
\item визрівання --- міфологічне ставлення до світу;
\item розквіт --- піднесення релігійного руху;
\item засихання --- цивілізація; агресивне ставлення людини до природи, урбанізація, пригнічення духовного життя, життя стає формальним и штучним, посилюється вплив бюрократії, культура набуває механичних рис.
\end{enumerate}

Шпенглер виділів 10 типів культури:
\begin{enumerate}
  \item Єгипетська;
  \item Індійська;
  \item Вавілонська;
  \item Китайська;
  \item Греко-римська;
  \item Магічна (Візантійсько-Арабсько);
  \item Фаустівська (Західно-європейська);
  \item Майя;
  \item Російсько-Сибірська;
  \item Американська.
\end{enumerate}

\subsection{Інформаційно-семіотична концепція культури}
Культура представляється як інформаційна система, в якій існує суспільство.

\textit{Семіотика} --- наука про знаки й знакові системи.

\textbf{Культура} --- сукупність знаків і кодів, у яких втілюється інформація.

\begin{itemize}
  \item \textit{Культура} --- світ артефактів.
  \item \textit{Культура} --- світ смислів.

    Артефакти мають суб’єктивну визначеність, в них втілені смисл і значення: людина вкладає в них те, що не існує без людини.

    Існує три види смислів:
    \begin{itemize}
      \item \textbf{знання} --- інформація про об’єкти дійсності;
      \item \textbf{цінності} --- характеристики об’єктів з точки зору їх необхідності потребам людини;
      \item \textbf{регулятиви} --- правила, або вимоги, відповідно до яких людина будує свою діяльність
    \end{itemize}
  \item \textit{Культура} --- світ знаків (як носії смислів артефакти стають знаками).
\end{itemize}

\textit{Явище культури} --- це система знаків, у яких зашифрована ця інформація.

Розуміти якесь явище культури --- сприймати його невидимий суб’єктивний смисл.

\subsubsection{Культура й цивілізація}
На побутовому рівні сприймалися як синоніми.

Цивілізацією називають культуру матеріальну. Культура --- духовний світ

Цивілізація --- певний рівень розвитку культури.

Ознаки цивілізації (по Моргану):
\begin{enumerate}
  \item наявність монументальних будівель;
  \item міста;
  \item писемність.
\end{enumerate}

\subsection{Соціологія культури}
\textit{Культурологія} --- комплексна гуманітарна наука, або комплекс наук, що охоплює всю сукупність знань про культуру.

Склад:
\begin{itemize}
  \item філософія культури;
  \item теорія культури;
  \item історія культури;
  \item культурна антропологія;
  \item прикладна культурологія;
  \item історія культурологічних вчень;
  \item соціологія культури.
\end{itemize}

\textit{Соціологія культури} вивчає культуру з погляду соціології.
Тобто культура вивчається як частина соціальної системи, соціальних відносин, як певний соціальний інститут.
Досліджує соціологічні функції культури у суспільному житті.

Функції культури:
\begin{itemize}
  \item виховна;
  \item аксіологічна;
  \item комунікативна;
  \item адаптивна;
  \item інтегративна.
\end{itemize}

Соціологія розглядає культуру, як спосіб регуляції людей.

Суспільні групи, які відрізняються економічними та політичними інтересами, соціальним статусом, психологічними й культурними особливостями, називають \textbf{субкультурою}.

Різновиди субкультур:
\begin{itemize}
  \item Елітарна;
  \item Середнього класу;
  \item Соціальних низів;
  \item Маргінальна;
  \item Вікова;
  \item Сексуальни меншини.
\end{itemize}
\section{Сутність культури}
\subsection{Культура --- друга ,,природа''. Співвіднесеність культури та природи}
Протиставлення культури та природи в культурології є фундаментальним.

Культура вибудовується над природою, використовуючи закони природи.

Різниця між волею та свободою полягає в тому, що людина не просто задовольняє свої бажання, а й контролює їх, надаючи своїм потребам культури.

Спікоза: \begin{quote}
	Свобода --- це усвідомлення необхідність.
\end{quote}

Культурне середовище формує:
\begin{itemize}
	\item Потреби людини;
	\item Здатності;
	\item Вміння;
	\item Фізіологію людини.
\end{itemize}
\subsection{Культура як неорганічне тіло людини}
Те, що обслуговує органічні потреби --- \textit{матеріальна культура}.
\textit{Духовна культура} формує в людині загальні здатності
\subsection{Відмінність між матеріальною та духовною культурами}
Форми духовної культури:
\begin{itemize}
	\item міф,
	\item релігія,
	\item мораль,
	\item мистецтво,
	\item філософія,
	\item наука
\end{itemize}
\section{Етнічний і національний вимір культури}
\subsection{Етнос і своєрідність етничної культури}
З XIX століття поняття \textit{етнос} з’являється разом із наукою етнографією --- вивчає народи, які живуть общинним життям і не мають писемності.

Етнос --- найдавніша форма соціального об’єднання людей, де на першому плані кровна спорідненість.
Для етносу об’єднуючою силою також виступає \textit{етнічна культура}.
Вона є способом культурної ідентифікації (поділу людей на ,,своїх'' і ,,чужих'').
На етапі варварства етнічна культура сприяє самоідентифікації.

\textit{Етнічна самосвідомість} --- це усвідомлення людьми своєї приналежності до цієї, а не якої-небудь іншої спільноти за такими ознаками: індивідуально-психологічні зв’язки, загальна система цінностей, об’єктивно-матеріальні умови життєдіяльності.

Міф є формою колективною самосвідомості.
\subsection{Нація та форми національної самосвідомості}
Передумови для формування національної самосвідомості:
\begin{enumerate}
	\item Державна влада;
	\item Писемність.
\end{enumerate}

Національна культура створюється інтелігенцією.
Вона є, як надбудова над етнічною.

\textit{Нація} --- об’єднання людей національною економікою, національною державою, а також національною культурою.
Особливу роль відіграє феномен національної самосвідомості для національної ідентифікації та національного самоідентифікації.

В національній культурі можна виділити 2 рівні:
\begin{enumerate}
	\item Вона виражена в національному характері, національній психології, які формуються стихійно, під впливом загально-національного життя.
	\item Вона представлена літературною мовою, високим мистецтвом і філософією, які артикулюються свідомими зусиллями національної інтелігенції.
\end{enumerate}
\subsection{Способи оволодіння національною культурою}
Національна культура є основою національної приналежності людини, яка у наш час перетворюється у предмет особистого вибору.
Свідомо обирається те, що потребує свідомих власних зусиль для освоєння духовної спадщини.
Кожна нація, на відміну від етносу, створює спеціальні заклади культури.
\section{Етногенез українців}
\subsection{Індоєвропейська мовна спільнота}
\textit{Етногенез} --- тривалий процес утворення народності на базі різних етнічних компонентів.

Вивчення етногенезу опираються на комплексне використання наук:
\begin{itemize}
	\item історія;
	\item мовознавство;
	\item етнографія;
	\item антропологія;
	\item археологія.
\end{itemize}

Українці належать до групи слов’янських народів.
Слов’яни --- мовні та культурні нащадки ідноєвропейців.
\subsection{Слов’яни. Доба формування концепції витоків}
Результатом зусиль лінгвістів, наприкінці XIX століття була зроблена класифікація індоєвропейських мов (сучасні й мертві мови).
Праслов’яни відокремились в кінці першого тисячоліття до нашої ери.
До першого століття нашої ери відбувалося становлення, еволюція праслов’янської мови.

V століття нашої ери --- велике переселення народів.
Концепції первісної території слов’ян:
\begin{itemize}
	\item Вісло-Дніпровська теорія;
	\item Дунайська;
	\item Вісло-Одерська;
	\item Азіатська;
	\item Норманська.
\end{itemize}

Письменники давніх часів називають 3 племені, з яких виділились слов’яни:
\begin{itemize}
	\item склавіни;
	\item венеди;
	\item анти.
\end{itemize}
\subsection{Концепції етногенезу українців}
\begin{itemize}
	\item Києво-Руська;
	\item Ганнослов’янська;
	\item Пізносередньовічна;
	\item Трипільсько-азійсько.
\end{itemize}

XV століття --- сформувалися українські слов’яни (за 3 концепцією).

Необхідною умовою встановлення віку будь-якого суспільно-історичного явища, методами історії та археології є неперервність культурно-історичного розвитку.
\section{Початки українознавств. Періодизація українознавств}
\begin{enumerate}
	\item XV століття - 60 роки XIX століття
	\begin{enumerate}
		\item XV--середина XVII століття;
		\item Остання чверть XVII століття -- кінець XVIII століття;
		\item до 60х років XIX століття.
	\end{enumerate}
	\item 60-ті роки XIX століття -- перша половина XX століття.
\end{enumerate}
Відомості про український народ, як окремий етнос, належать до XV століття.
Гойом де Боллак --- ,,Опис України''.

XVII століття, військовий інженер, був на службі у польського короля, будував фортеці.

Період хмельниччини, цінні відомості з українознавства --- короткий київський літопис.
Найцінніша літописна пам’ятка українознавства --- Густинський літопис --- історія України в контексті світової історії.

Характерна поява історіографії.
Хроніка Феодосія Сафоновича, Інокентій Гізень --- ,,Санопсис'' (1674 рок).

Центри українознавства --- Москва та Петербург.
,,Описание свадебных украинских обрядов'', \dots
Доводиться, що політично-національно та культурно історія України є самобутньою і має свої власні кордони.

Формування наукового українознавства в XIX столітті:
 в Харківському університеті було створено гурток любителів української народності ,,Запорожжская старина''.
 
 1876 --- ,,Світогляд українського народу'' --- перший систематичний огляд українців.
 Починається діяльність українських етнографів, з’являються видання з етнографії: десятитомна ,,Історія України-Русі''\dots
\section{Міфологічний простір слов’янського язичництва, релігія, міфологія, культи}
\subsection{Структура міфологічного простору}
\textit{Міфологічний простір} --- звичайний простір, координати якого є магічними, міфологічними.

\textbf{Символіка (горизонтальна)}: схід і полудень --- добрі значення, захід і північ --- погані.
Схід --- висхідний --- ,,початок''.
Напрям руху проти сонця --- погано.
,,Ліво'' й ,,Право'' --- великі значення.

\textbf{Символіка (вертикальна)}: три частини дерева.

\subsection{Мікрокосм людини}
\textit{Мікрокосм} --- внутрішній світ.
У стародавніх слов’ян були складніші уявлення про співвідношення душі та тіла, ніж у християн.
Язичники вважали, що людина має кілька душ:
\begin{enumerate}
	\item та, що лишається на тому світі після смерті та з’являється до нащадків у родительські дні.
	\item та, що дає людині вітальну, життєву силу.
	\item та, що робить людину людиною, членом суспільства.
\end{enumerate}
До мікрокосма входить родина. Кожен з них має окремий символ й функції.
\subsection{Міфологія}
\textit{Міфи} --- історії космосу, впорядкування світу. Система норм життя, що відобразилась у повір’ях та забубонах.
У календарній інтерпретації міфи найяскравіше виражені полярності, пов’язані із сонцестоянням і рівноденням.
\subsection{Язичницький культ}
\textit{Звичаї} --- норми поведінки, які успадковуємо за традицією.

\textit{Обряди} --- звичаї із символічним смислом.
Обряди, що становлять цілісну систему називають \textbf{культом}.

Культ є органічною частиною релігії.
Культові дії супроводжували все життя людини --- від народження до смерті.
Збереглися 4 обряди: народження, христильні, весільні й поховальні.
\section{Формування географічної карти України}
Лісисту частину населяли землеробські племена (трипільська, черняхівська та зуребенецька культури).
Степова частина України --- кочові племена (кімерійці, скіфи, сармати).
Північна причорноморська частина --- міста-поліси (Ольвія, Тіра, Пантікапей,\dots).
Середнє подніпров’я --- безпосередні предки слов’ян.

До XV століття основна частина українських земель під владою іноземців.
За часів Бориса Хмельницького частина української території здобула незалежність.

Колонізація причорноморських степів, зруйнування Кримського ханства.
Утворились: слов’яно-сербські, німецькі, грецькі колонії.

До початку XX століття українці бід владою Австрії та Росії.
У XX столітті в результаті першою світовою війни,
розпад Австро-Угорської імперії й боротьба за незалежність народів, які входили до її складу.
\section{Культура княжої доби}
\subsection{Київська Русь --- одна з наймогутніших країн середньовіччя}
Феномен культурного злету Київської Русі пояснюється тісними контактами з Візантією, Хазарією, країнами Європи.
\subsection{Дохристиянські вірування}
\subsection{Значення прийняття християнства для духовного життя Київської Русі}
Із прийняттям християнства у Київській Русі поширилась нова складка організована релігія.
Церква утверджувала нові норми моралі.
Церкви та монастирі стали осередками писемності та освіти.
Розвивались мистецтва та релігія.
Базою духовної культури є освіта та писемність.
\subsection{Освіта: писемність, школи, книги}
Школи при монастирях, дворах.
Примусове навчання бояр.
Знання грецької мови --- вищий рівень освіти.
Школи для дівчат (школа Анни Ярословни була чи не першою школою для дівчат у Європі).
\subsection{Наукові знання}
Книги вироблялись з пергаменту (теляча шкура).
,,Зборнік Ярослава''. Никон --- найдавніший літописець. Нестор. Сильвестр.

В Київській Русі створюються багато монастирів.
Одним з перших й найбільших був Києво-Печерський монастир 1051 року.

Монах Агапіт --- перший лікар Києво-Печерського монастиря.
\subsection{Мистецтво: архітектура, малярство, іконопис, прикладне мистецтво, музика}
Будівництво на Русі --- дерев’яне.
З прийняттям християнства розпочалося кам’яне будівництво.
Софіївський собор зберігся з часів Київської Русі.
\section{Історія розвитку та становлення української мови}
\subsection{Значення та функції}
Кожна мова відображає спосіб мислення певної нації й кожної конкретної людини.

Функції мови:
\begin{itemize}
	\item інформативна;
	\item пізнавальна;
	\item комунікативна;
	\item естетична.
\end{itemize}
\subsection{Походження української мови та формування літературної мови}
\begin{enumerate}
	\item Українська, білоруська та російська мови сформувалися в XV--XVI століттях з давньоруської мови.
	\item Єдиної давньоруської мови не існувало і українська мова сягає у добу спільно-слов’янської мови (V століття).
\end{enumerate}

Найдавніші пам’ятки письменства спільно-слов’янською мовою:
,,Остромирове Євангеліє'' (1056 рік),
,,Ізборник Святослава'' (1076 рік).

Кримський зробив висновки, що\begin{quote}мова Наддніпрянщини та Червоної Русі має всі сучасні малоруські особливості.\end{quote}
\subsection{Основні етапи розвитку літературної мови}
\begin{enumerate}
	\item Літературно-писемна мова Київської Русі --- старослов’янська мова (X -- XIV століття).
	\item Стара слов’яно-руська літературно-писемна мова (XIV -- перша половина XVII століття).
	\item Стара слов’яно-руська літературно-писемна мова формування української нації (XVII -- XVIII століття).
	\item Нова українська літературна мова (XVIII -- початок XX століття).
\end{enumerate}

Сучасна українська літературна мова сформувалася на основі полтавсько-київських говорів.
\subsection{Словники і граматики}
Перша граматика церковно-слов’янської мови була надрукована у Вільно у 1568 році.

1596 рік --- Лаврентій Зизаній ,,Граматика'' та ,,Лексис''.

1619 рік --- Мелетій Смотрицький ,,Граматика''.

1627 рік --- Памва Беринди ,,Лексикон''.

1907 рік --- словник Бориса Грінченка.

1918 рік --- І.\,Онієнко ,,Український проект правопису''.

1919 рік --- ,,Головніші правила українського правопису''/
\section{Освіта й наука в Україні}
\subsection{Братства та братські школи}
Перші братства виникли у Львові, як громадські організації в оборону православ’ю.

Пізніше виникли братства у Києві.
До київського братства записався Сагайдачний.
Саме братства створювали школи на православній основі.
\subsection{Острозька академія}
Величезну роль в організації українського православного шкільництва відіграв князь К.\,Острозький.
В різних містах Волині він створив школи, де боролися з окатоличенням молоді.
У своїй резиденції він заснував перший вищий заклад України --- Острозький колегіум.
Викладалися 7 вільних наук: трівіум (граматика, риторика, діалектика), квадрівіум (арифметика, геометрія, музика й астрономія), мови (слов’яно-руська, грецька та латинська)
\subsection{Києво-Могилянська академія}
У 1615 році було подаровано маєток під школу київському братстві.

У 1631 році Петро Могила заснував Лаврську школу.

У 1632 році ці школи об’єднали в Києво-Могилянський колегіум.

У 1701 році за наказом Петра Першого стає Києво-Могилянською академією.

В ній вивчались як античні філософи, так і філософи того часу.
Були створені бібліотека та друкарня.
З академії вийшло багато громадських, політичних, культурних осіб.
\subsection{Університети і вищі школи Україні}
Друга половина вісімнадцятого століття --- спроба створити світський університет.

Феофан Прокопович за 100 днів пішки дійшов до Риму, там поступив в католицький університет и прийняв католицизм, повернувся до Києва и прийняв назад християнство, після чого співпрацював з Петром Першим.
\end{document}
