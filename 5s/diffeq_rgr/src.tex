\documentclass[a4paper,10pt,notitlepage,pdftex,headsepline]{scrartcl}

\usepackage{a4wide}
\usepackage{cmap} % чтобы работал поиск по PDF
\usepackage[utf8]{inputenc}
\usepackage[russian]{babel}
\usepackage[T2A]{fontenc}

\usepackage{textcase}
\usepackage[pdftex]{graphicx}

\usepackage{lscape}

\pdfcompresslevel=9 % сжимать PDF
\usepackage{pdflscape} % для возможности альбомного размещения некоторых страниц
\usepackage[pdftex]{hyperref}
% настройка ссылок в оглавлении для pdf формата
\hypersetup{unicode=true,
            pdftitle={РГР по Дифференциальным уравнениям},
            pdfauthor={Погода Михаил},
            pdfcreator={pdflatex},
            pdfsubject={},
            pdfborder    = {0 0 0},
            bookmarksopen,
            bookmarksnumbered,
            bookmarksopenlevel = 2,
            pdfkeywords={},
            colorlinks=true, % установка цвета ссылок в оглавлении
            citecolor=black,
            filecolor=black,
            linkcolor=black,
            urlcolor=blue}

\usepackage{amsmath}
\usepackage{amssymb}
\usepackage{moreverb}
%for \includepdf
%\usepackage{pdfpages}

\author{Михаил Погода}
\title{Расчётно-графическая работа по дифференциальным уравнениям}
\date{\today}

\def\rang{\mathop{\rm rang}}
\begin{document}
\maketitle
\begin{center}
\bf
Вариант \textnumero~23
\end{center}
\section{Линейное уравнение с постоянными коэффициентами}
\subsection*{Задание}
Проанализировать корректность постановки задачи Коши для данного дифференциального уравнения и решить его
\[
y^{\prime\prime} - 4 y^\prime + 4 y = 2 e^{2x}
\]
\[
y(0) = -1,\qquad y^\prime(0) = 0
\]
\subsection*{Решение}
Так как функция $2 e^{2 x}$ непрерывна, а коэффициенты при $y, y^\prime, y^{\prime\prime}$ --- постоянные, то задача Коши поставлена корректно.

\subsubsection*{Поиск решение однородного уравнения}
Характеристическое уравнение:
\[
\lambda^2 - 4 \lambda + 4
\]
\[
\lambda = 2, k = 2
\]
Фундаментальная система решений:
 ФСР = $\left\{ e^{2 x}, x e^{2 x} \right\}$.
 
\[
y_\text{о.\,р.} = C_1 e^{2 x} + C_2 x e^{2 x}
\]
\subsubsection*{Поиск частичного решения}
\[
y_\text{ч.\,р.} = C_3 x^2 e^{2 x}
\]
\[
y_\text{ч.\,р.}^{\prime\prime} - 4 y_\text{ч.\,р.}^\prime + 4 y_\text{ч.\,р.} = 2 e^{2 x}
\]
Подставив и упростив, получим
\[
C_3 = 1
\]

Таким образом, общее решение исходного уравнения имеет вид
\[
y = C_1 e^{2 x} + C_2 x e^{2 x} + x^2 e^{2 x}
\]
\subsection*{Поиск констант из начальных условий}
Найдём константы $C_1$ и $C_2$ из начальных условий:
\[
y(0) = C_1 = -1
\]
\[
y^\prime = 2 C_1 e^{2 x} + C_2 e^{2 x} + 2 C_2 x e^{2 x} + 2 e^{2 x} (x + x^2)
\]
\[
y^\prime(0) = -2 + C_2 = 0
\]
\subsection*{Ответ}
\[
y = e^{2 x} (x^2 + 2 x - 1)
\]
\newpage
\section{Оператор эволюции}
\subsection*{Задание:}
Определить оператор эволюции системы линейных дифференциальных уравнений с матрицей $A$.
Записать решение через найденный оператор эволюции.
На каком множестве определён этот оператор?
\[
A = \left(\begin{matrix}
3 & 0 & -1\\
3 & -2& 1\\
6 & -1& -1
\end{matrix}\right)
\]
\subsection*{Решение}
Характеристическое уравнение:
\[
\left|
\begin{matrix}
3-\lambda & 0 & -1\\
3 & -2-\lambda & 1\\
6 & -1 & -1-\lambda
\end{matrix}
\right| = 0
\]
или
\[\lambda^3 = 0\]
\[\lambda = 0,\qquad k = 3\]

\[e^{A t} = e^{(N + S) t}\]
\[S = P S_0 P^{-1}\]
\[S_0 = \left(\begin{matrix}
\lambda & 0 & 0\\
0 & \lambda & 0\\
0 & 0 & \lambda
\end{matrix}\right) = \Theta\]
\[S = \Theta\]
\[N = A - S = A\]
\[N^2 = \left(\begin{matrix}
3 & 0 & -1\\
3 & -2& 1\\
6 & -1& -1
\end{matrix}\right)\times\left(\begin{matrix}
3 & 0 & -1\\
3 & -2& 1\\
6 & -1& -1
\end{matrix}\right)=\left(\begin{matrix}
3 & 1 & -2\\
9 & 3 & -6\\
9 & 3 & -6
\end{matrix}\right)\]
\[N^3=\left(\begin{matrix}
3 & 1 & -2\\
9 & 3 & -6\\
9 & 3 & -6
\end{matrix}\right)\times\left(\begin{matrix}
3 & 0 & -1\\
3 & -2& 1\\
6 & -1& -1
\end{matrix}\right) = \Theta\]
$N$ --- нильпотентная матрица третьего порядка.
\[e^{N t} = I + Nt + \frac{N^2 t^2}{2} =\left(\begin{matrix}
1 & 0 & 0\\
0 & 1 & 0\\
0 & 0 & 1
\end{matrix}\right) + \left(\begin{matrix}
3 & 0 & -1\\
3 & -2& 1\\
6 & -1& -1
\end{matrix}\right) t + \left(\begin{matrix}
3 & 1 & -2\\
9 & 3 & -6\\
9 & 3 & -6
\end{matrix}\right) \frac{t^2}{2}\]
\[\displaystyle e^{N t} = \left(\begin{matrix}
\strut\frac{3}{2} t^2 + 3 + 1 & \frac{1}{2} t^2           & -t^2 - t\\
\frac{9}{2} t^2 + 3 t   & \frac{3}{2} t^2 - 2 t + 1 & t - 3 t^2\\
\frac{9}{2} t^2 + 6 t   & \frac{3}{2} t^2 - t       & 1 - t - 3 t^2
\end{matrix}\right)\]
\subsection*{Ответ}
\[e^{A t} = e^{(S + N) t} = e^{S t} e^{N t} = e^{N t} = \left(\begin{matrix}
\strut\frac{3}{2} t^2 + 3 + 1 & \frac{1}{2} t^2           & -t^2 - t\\
\frac{9}{2} t^2 + 3 t   & \frac{3}{2} t^2 - 2 t + 1 & t - 3 t^2\\
\frac{9}{2} t^2 + 6 t   & \frac{3}{2} t^2 - t       & 1 - t - 3 t^2
\end{matrix}\right)\]
Данный оператор действует в пространстве $\mathbb{R}^3$.
\newpage
\section{Фазовый портрет нелинейной системы}
\subsection*{Задание}
Проанализировать количество и тип стационарных точек нелинейной системы на плоскости.
Исследовать на простоту и определить стойкость по Ляпунову полученных недвижимых точек.
Построить фазовый портрет системы.
\[
\begin{cases}
\dot{x} = x(y - 2x)\\
\dot{y} = x^2 - 8 y + y^2 - x\\
\end{cases}
\]
\newpage
\section{Действительная каноническая форма оператора}
\subsection*{Задание}
Найти действительную каноническую форму оператора
\[
\left(
\begin{matrix}
1 & 0 & -4 & 2 & -2\\
2 & -2 & -4 & 6 & -1\\
0 & 0 & -1 & 1 & -1\\
1 & -1 & -2 & 3 & 0\\
1 & -1 & -2 & 2 & 0
\end{matrix}
\right)
\]
\subsection*{Решение}
Характеристическое уравнение:
\[
(\lambda - 1)^3 (\lambda + 1)^2 = 0
\]
\[
\lambda_1 = 1, \qquad k = 3
\]
\[
\lambda_2 = -1, \qquad k = 2
\]
\subsubsection*{Определение количества блоков для $\lambda_1$}
\[\delta^{\lambda_1}_1 = 5 - \rang{(A - I)} = 1\]
\[\delta^{\lambda_1}_2 = 5 - \rang{(A - I)^2} = 2\]
\[\delta^{\lambda_1}_3 = 5 - \rang{(A - I)^3} = 3\]
\[v^{\lambda_1}_1 = 2 \delta^{\lambda_1}_1 - \delta^{\lambda_1}_2 = 0\]
\[v^{\lambda_1}_2 = -\delta^{\lambda_1}_1 + 2 \delta^{\lambda_1}_2 - \delta^{\lambda_1}_3 = 0\]
\[v^{\lambda_1}_3 = \delta^{\lambda_1}_3 - \delta^{\lambda_1}_2 = 1\]
\subsubsection*{Определение количества блоков для $\lambda_2$}
\[\delta^{\lambda_2}_1 = 5 - \rang{(A + I)} = 1\]
\[\delta^{\lambda_2}_2 = 5 - \rang{(A + I)^2} = 2\]
\[v^{\lambda_2}_1 = 2 \delta^{\lambda_2}_1 - \delta^{\lambda_2}_2 = 0\]
\[v^{\lambda_2}_2 = \delta^{\lambda_2}_3 - \delta^{\lambda_2}_2 = 1\]
\subsection*{Ответ:}
Действительная каноническая форма:
\[\left(\begin{matrix}
1 & 0 & 0 & 0 & 0\\
1 & 1 & 0 & 0 & 0\\
0 & 1 & 1 & 0 & 0\\
0 & 0 & 0 & -1 & 0\\
0 & 0 & 0 & 1 & -1
\end{matrix}\right)
\]
\newpage
\section{Операционное исчисление}
\subsection*{Задание}
Решить, используя преобразования Лапласа
\[
\begin{cases}
\dot{x} - \dot{y} = -\sin{t}\\
\dot{x} + \dot{y} = \cos{t}\\
x(0) = \frac{1}{2}\\
y(0) = -\frac{1}{2}
\end{cases}
\]
\subsection*{Решение}
\[
\begin{aligned}
x(t) & \risingdotseq X(p)\\
\dot{x}(t) &\risingdotseq p X(p) - x(0) = p X(p) - \frac{1}{2}\\
y(t) &\risingdotseq Y(p)\\
\dot{y}(t) &\risingdotseq p Y(p) - y(0) = p Y(p) + \frac{1}{2}\\
\sin{t} &\risingdotseq \frac{1}{p^2 + 1}\\
\cos{t} &\risingdotseq \frac{p}{p^2 + 1}
\end{aligned}
\]
\[
\begin{cases}
p X(p) - p Y(p) - 1 = -\frac{1}{p^2 + 1}\\
p X(p) + p Y(p) = \frac{p}{p^2 + 1}
\end{cases}
\]
Сложив уравнения, получим
\[2 p X(p) = \frac{p - 1}{p^2 + 1} + 1\]
\[X(p) = \frac{1}{2}\left(\frac{1}{p^2 + 1} + \frac{p}{p^2 + 1}\right)\]
Подставив полученное значение во второе уравнение системы:
\[\frac{p}{2}\frac{1 + p}{p^2 + 1} + p Y(p) = \frac{p}{p^2 + 1}\]
\[Y(p) = \frac{1}{2}\left(\frac{1}{p^2 + 1} - \frac{p}{p^2 + 1}\right)\]
Вернёмся от изображений к оригиналам:
\[\begin{aligned}
X(p) &\fallingdotseq \frac{1}{2}\left(\sin{t} + \cos{t}\right)\\
Y(p) &\fallingdotseq \frac{1}{2}\left(\sin{t} + \cos{t}\right)
\end{aligned}\]
Проверим выполнение начальных условий:
\[ x(0) = \frac{1}{2}\left(\sin{0} + \cos{0}\right) = \frac{1}{2}\left(0 + 1\right) = \frac{1}{2}\]
\[ y(0) = \frac{1}{2}\left(\sin{0} - \cos{0}\right) = \frac{1}{2}\left(0 - 1\right) = -\frac{1}{2}\]
\subsection*{Ответ}
\[\begin{aligned}
x(t) &= \frac{1}{2}\left(\sin{t} + \cos{t}\right)\\
y(t) &= \frac{1}{2}\left(\sin{t} - \cos{t}\right)
\end{aligned}\]
\newpage
\section{Квазилинейные уравнения}
\subsection*{Задание}
Найти решение, удовлетворяющее начальным условиям
\[\begin{cases}
x\frac{\partial u}{\partial x} + y\frac{\partial u}{\partial y} = u - x y\\
u(x, 2) = 1 + x^2
\end{cases}\]
\subsection*{Решение}
\[\frac{dx}{x} = \frac{dy}{y} = \frac{du}{u - xy}\]

\[\frac{dx}{x} = \frac{dy}{y}\]
\[\ln{x} = \ln{y} + \tilde{C_1} \Rightarrow \ln{\frac{x}{y}} = \tilde{C_1}\]
\[\frac{x}{y} = e^{\tilde{C_1}}\]
\[\frac{x}{y} = C_1\]

\[\frac{du}{u-xy} = \frac{dx}{x}\]
\[x du = (u - xy) dx\]
\[x du = (u - \frac{x^2}{C_1}) dx\]
\[\frac{du}{dx}=\frac{u}{x}-\frac{x}{C_1}\]
\[u = C_2 x - \frac{x^2}{C_1}\]
\[u = C_2 x - x y\]
\[C_2 = \frac{u + x y}{x}\]

\[\Phi\left(\frac{x}{y}, \frac{u + xy}{x}\right) = 0\]

Найдём из начальных условий соотношение между $C_1$ и $C_2$:
\[\frac{x}{2} = C_1 \Rightarrow x = 2 C_1\]
\[C_2 = \frac{1+4 C_1^2 + 4 C_1}{2 C_1} = \frac{\left(1 + 2 C_1\right)^2}{2 C_1}\]
\subsection*{Ответ}
\[\cfrac{u + xy}{x} = \cfrac{\left(1+2 \cfrac{x}{y}\right)^2}{2\cfrac{x}{y}}\]
или в явном виде
\[u = \cfrac{y}{2}\left(1 + 2 \cfrac{x}{y}\right)^2 - x y\]
\end{document}



















