\documentclass[a4paper,10pt,notitlepage,pdftex]{scrartcl}

\usepackage{cmap} % чтобы работал поиск по PDF
\usepackage[utf8]{inputenc}
\usepackage[english,russian,ukrainian]{babel}
\usepackage[T2A]{fontenc}

\usepackage{textcase}
\usepackage[pdftex]{graphicx}

\pdfcompresslevel=9 % сжимать PDF
\usepackage{pdflscape} % для возможности альбомного размещения некоторых страниц
\usepackage[pdftex]{hyperref}
% настройка ссылок в оглавлении для pdf формата
\hypersetup{
	unicode=true,
    pdftitle={},
    pdfauthor={Кулик Ольга},
    pdfcreator={pdflatex},
    pdfsubject={},
    pdfborder    = {0 0 0},
    bookmarksopen,
    bookmarksnumbered,
    bookmarksopenlevel = 2,
    pdfkeywords={},
    colorlinks=true, % установка цвета ссылок в оглавлении
    citecolor=black, 
    filecolor=black,
    linkcolor=black,
    urlcolor=blue
}

\usepackage{amsmath}
\usepackage{amssymb}
\usepackage{moreverb}

\usepackage{a4wide}
\usepackage{geometry}
\geometry{left=1cm}
\geometry{top=1.5cm}
\geometry{right=1cm}
\author{Кулик Ольга}
\title{ЧМ}
\date{\today}

\usepackage{moreverb}

\begin{document}
\thispagestyle{empty}
{
\LARGE

\vspace*{7cm}

\begin{center}
Домашняя контрольная работа \textnumero~3\\
Тема: ,,Задача Коши''

Вариант \textnumero~11\\
\end{center}

\vspace*{4cm}
\begin{flushright}
Студента\\
ФПМ, КМ-92\\
Погоды Михаила
\end{flushright}
}

\newpage

\section*{Задание \textnumero~1\\}

Рассмотреть задачу Коши:

$$u^{\prime\prime} = \frac{1}{2}u + x$$

$$ u(0) = 0$$

Построить ее численное решение на отрезку [0;2] с шагом h=0.25.

Сравнить результаты между собой и с аналитическим решением:
$$u(x)=-2(x+2)+4e^{\frac{x}{2}}$$

\subsection*{По схеме Эйлера}
$$y_{i+1} = y_i + h\cdot f(x_i,y_i);$$
$$h = x_{i+1} - x_i;$$
$$y_0 = u_0 $$
\subsubsection*{Решение:}
\begin{tabular}{|c|c|c|c|c|c|}
\hline
$i$	& $x_i$		& $y_i$		& $f_i$		& $u(x_i)$	& $\Delta_i$\\
\hline
$0$	& $0$		& $0$		& $0$		& $0$		& $0$\\
$1$	& $0.25$	& $0$		& $0.2500$	& $0.0326$	& $0.0326$\\
$2$	& $0.50$	& $0.6250$	& $0.5313$	& $0.1361$	& $0.0736$\\
$3$	& $0.75$	& $0.1953$	& $0.8477$	& $0.3200$	& $0.1247$\\
$4$	& $1.00$	& $0.4072$	& $1.2036$	& $0.5949$	& $0.1877$\\
$5$	& $1.25$	& $0.7081$	& $1.6041$	& $0.9730$	& $0.2649$\\
$6$	& $1.50$	& $1.1091$	& $2.0546$	& $1.4680$	& $0.3589$\\
$7$	& $1.75$	& $1.6227$	& $2.5614$	& $2.0955$	& $0.4727$\\
$8$	& $2.00$	& $2.2631$	& $3.1316$	& $2.8731$	& $0.6100$\\
\hline
\end{tabular}

\subsection*{По схеме Эйлера-Коши}

$$y^*_{i+1} = y_i + h\cdot f(x_i,y_i)$$

$$y_{i+1} = y_i + \frac{h}{2}\left(f(x_i,y_i)+f(x_{i+1},y^*_{i+1})\right)$$

\subsubsection*{Решение:}
\begin{tabular}{|c|c|c|c|c|c|c|c|}
\hline
$i$	& $x_i$		& $f_i$		& $y^*_i$	& $f_i(x_i,y^*_i)$	& $y$		& $u(x_i)$	& $\Delta_i$\\
\hline
$0$	& $0$		& $0$		& $0$		& $0.25$			& $0$		& $0$		& $0$\\
$1$	& $0.25$	& $0.2656$	& $0.0977$	& $0.5488$			& $0.0313$	& $0.0326$	& $0.0013$\\
$2$	& $0.50$	& $0.5665$	& $0.2747$	& $0.8873$			& $0.1331$	& $0.1361$	& $0.0030$\\
$3$	& $0.75$	& $0.9074$	& $0.5416$	& $1.2708$			& $0.3148$	& $0.3200$	& $0.0052$\\
$4$	& $1.00$	& $1.2935$	& $0.9105$	& $1.7052$			& $0.5870$	& $0.5949$	& $0.0078$\\
$5$	& $1.25$	& $1.7309$	& $1.3947$	& $2.1973$			& $0.9619$	& $0.9730$	& $0.011$\\
$6$	& $1.50$	& $2.2265$	& $2.0096$	& $2.7548$			& $1.4529$	& $1.4680$	& $0.0150$\\
$7$	& $1.75$	& $2.7878$	& $2.7726$	& $3.3863$			& $2.0756$	& $2.0955$	& $0.0199$\\
$8$	& $2.00$	& $3.4237$	& $3.7033$	& $4.1016$			& $2.8474$	& $2.8731$	& $0.0258$\\
\hline
\end{tabular}

\subsection*{По схеме Рунге-Кутта четвертого порядка}

$$y_{i+1} = y_i + \frac{1}{6}(k_1 + 2 k_2 + 2 k_3 + k_4)$$
$$k_1 = h\cdot f(x_i, y_i);$$
$$k_2 = h\cdot f(x_i + \frac{h}{2}, y_i + \frac{k_1}{2});$$
$$k_3 = h\cdot f(x_i + \frac{h}{2}, y_i + \frac{k_2}{2});$$
$$k_4 = h\cdot f(x_i + h, y_i + k_3);$$

\subsubsection*{Решение:}
\begin{tabular}{|c|c|c|c|c|c|c|c|}
\hline
$i$	& $x_i$		& $y_i$			& $u(x_i)$		& $\Delta_i$\\
\hline
$0$	& $0$		& $0$			& $0$			& $0$\\
$1$	& $0.25$	& $0.0325928$	& $0.0325938$	& $1.0388\cdot 10^{-6}$\\
$2$	& $0.50$	& $0.1360993$	& $0.1361017$	& $2.3543\cdot 10^{-6}$\\
$3$	& $0.75$	& $0.3199617$	& $0.3199657$	& $4.0017\cdot 10^{-6}$\\
$4$	& $1.00$	& $0.5948790$	& $0.5948851$	& $6.0460\cdot 10^{-6}$\\
$5$	& $1.25$	& $0.9729753$	& $0.9729838$	& $8.5637\cdot 10^{-6}$\\
$6$	& $1.50$	& $1.4679884$	& $1.4680001$	& $1.1645\cdot 10^{-5}$\\
$7$	& $1.75$	& $2.0954858$	& $2.0955012$	& $1.5394\cdot 10^{-5}$\\
$8$	& $2.00$	& $2.8731074$	& $2.8731273$	& $1.9936\cdot 10^{-5}$\\
\hline
\end{tabular}

\subsection*{Вывод:}
При возростании $i$ погрешность увеличивается, так как интегральные кривые расходятся. 

Метод Эйлера является методом первого порядка, т.\,е. имеет первый порядок точности $O(h)$. 

Метод Эйлера-Коши имеет второй порядок точности, т.\,е. его применение уменьшает в среднем значение погрешностей до величины второго порядка малости $O(h^2)$. 

Метод Рунге-Кутта четвертого порядка имеет четвёртый порядок точности, то есть суммарная ошибка на конечном интервале интегрирования имеет порядок $O(h^4)$.

\subsection*{Приложение (код на java):}
\subsubsection*{euler\_cauchy\_method.m}
\listinginput{1}{euler_cauchy_method.m}

\subsubsection*{euler\_method.m}
\listinginput{1}{euler_method.m}

\subsubsection*{func.m}
\listinginput{1}{func.m}

\subsection*{real\_answer.m}
\listinginput{1}{real_answer.m}

\subsection*{runge\_kutta\_method.m}
\listinginput{1}{runge_kutta_method.m}

\subsection*{task.m}
\listinginput{1}{task.m}
\end{document}