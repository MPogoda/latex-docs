\documentclass[a4paper,12pt,notitlepage,pdftex,headsepline]{scrartcl}

\usepackage{a4wide}
\usepackage{cmap} % чтобы работал поиск по PDF
\usepackage[utf8]{inputenc}
\usepackage[russian]{babel}
\usepackage[T2A]{fontenc}

\usepackage{textcase}
\usepackage[pdftex]{graphicx}

\usepackage{lscape}

\pdfcompresslevel=9 % сжимать PDF
\usepackage{pdflscape} % для возможности альбомного размещения некоторых страниц
\usepackage[pdftex]{hyperref}
% настройка ссылок в оглавлении для pdf формата
\hypersetup{unicode=true,
            pdftitle={ДКР по ОБЖ},
            pdfauthor={Погода Михаил},
            pdfcreator={pdflatex},
            pdfsubject={},
            pdfborder    = {0 0 0},
            bookmarksopen,
            bookmarksnumbered,
            bookmarksopenlevel = 2,
            pdfkeywords={},
            colorlinks=true, % установка цвета ссылок в оглавлении
            citecolor=black,
            filecolor=black,
            linkcolor=black,
            urlcolor=blue}

\usepackage{amsmath}
\usepackage{amssymb}
\usepackage{moreverb}

\author{Михаил Погода}
\title{ДКР по ОБЖ}
\date{\today}

\begin{document}
  \thispagestyle{empty}
  \begin{center}
    \large
    \MakeUppercase{Министерство образования и науки,}

    \MakeUppercase{молодёжи и спорта Украины}

    \MakeUppercase{Национальный технический университет Украины}

    \MakeUppercase{,,Киевский политехнический институт''}

    \addvspace{6pt}

    \normalsize
    Кафедра прикладной математики

    \vfill

    \textbf{Домашняя контрольная работа}

    по курсу ,,Безопасность жизнедеятельности''

    \addvspace{6pt}

    на тему:
    \texttt{,,Влияние микроклимата рабочих помещений на работников''}
  \end{center}

  \vfill

  \begin{flushright}
    Студента

    группы КМ-92

    Погоды Михаила
  \end{flushright}

  \vfill

  \begin{center}
    КИЕВ

    2012
  \end{center}
  \newpage

  \textit{Создание благоприятных условий работы является основой высокого
    уровня работоспособности трудящихся, а значит, увеличения прибыли
    предприятия.
    Одним из необходимых условий здоровой и продуктивной работы является
    обеспечение нормальных условий в рабочем помещении.
    На самочувствие, здоровье рабочего влияет микроклимат рабочих помещений,
    который определяется действием на организм человека температуры, влажности,
    теплового излучения.
    Актуальность этой проблемы обусловлена стремительный возрастанием числа и
    тяжести профессиональных заболеваний, острых отравлений, количества
    производственных аварий.
  }

  Над способами улучшения регулирования микроклимата на производстве работало
  много людей.
  Расчёту, построению и автоматизации систем вентиляции и кондиционирования
  воздуха посвящены публикации Шутька О.\,В. [1], Сукача С.\,В. [2], Бондаря
  Е.\,С. [3], однако они недостаточно описывали физические аспекты
  формирования необходимого воздушного режима в помещения.
  С другой стороны, множество литературных источников посвящены именно
  теплофизическим аспектам формирования необходимого температурного режима в
  помещении, которые учитывают динамические характеристики вентиляционных
  устройств, но не обеспечивают управление микроклиматическими параметрами в
  зоне нахождения работников [4, 5].

  Анализ этих литературных источников показал, что лучшие системы регуляции
  микроклимата --- это те, которые позволяют своевременно удаление вредоносных
  примесей и поддержку всех параметров микроклимата в границах
  регламентированных норм.
  Однако, мной было замечено отсутствие математического аппарата, с помощью
  которого можно было бы выразить параметры комфортности от заданной
  совокупности факторов микроклимата\footnote{Ощущению комфорта человека
  отвечают данные из таблицы №1}.
  Именно такой математический аппарат я предлагаю ввести.
  Он может помочь с оценкой микроклимата в целом, а это важно, т.\,к.
  параметры микроклимата связаны между собой --- изменение одного приводит к
  изменению других.
  Эта математическая модель может быть основой многофункциональных моделей
  систем управления, которые должны быть подкреплены множеством параметров
  микроклимата.
  Эти системы в свою очередь поддерживали бы параметры микроклимата на
  производстве в рамках санитарно-гигиенических норм, тем самым поддерживая
  комфортные условия работы.

  \begin{table}[h]
    \centering
    \begin{tabular}{|c|c|c|}
      \hline
      $T, ^\circ C$ & $v$, м/с & $\varphi, \%$\\
      \hline
      $18$ & $0.00\cdots0.10$ & $60\cdots75$\\
      $19$ & $0.00\cdots0.12$ & $39\cdots73$\\
      $20$ & $0.00\cdots0.15$ & $38\cdots72$\\
      $21$ & $0.00\cdots0.20$ & $37\cdots69$\\
      $22$ & $0.00\cdots0.25$ & $35\cdots67$\\
      $23$ & $0.00\cdots0.30$ & $34\cdots57$\\
      $24$ & $0.19\cdots0.36$ & $35\cdots40$\\
      \hline
    \end{tabular}
    \caption{Диапазон комфортных значений температуры, скорости воздуха и
    относительной влажности}
  \end{table}

  Выделяют оптимальные, допустимые и вредные микроклиматические условия.
  Параметры этих условий для производственных помещений и открытых территорий
  в жаркое и холодное времена года приведены в ДСН\,3.3.6\,042-99 ,,Санитарные
  нормы микроклимату помещений''.

  Рассматривая механизмы влияния метеорологических факторов производственного
  окружения (температуры, влажности, скорости движения воздуха) на человека,
  следует учитывать, что человеческий организм стремиться поддержать
  относительное динамическое постоянство своих функций при разнообразных
  метеорологических условиях.
  Это постоянство обеспечивает один из важнейших физиологических механизмов
  --- механизм терморегуляции.
  Она наблюдается при определённом соотношении тепловыделения и теплоотдачи.

  Известно, что излишняя влажность воздуха негативно влияет на механизм
  терморегуляции организма.
  Особенно вредной является влажность воздуха, которая превышает $70$--$75$\%
  при температуре $30^\circ C$ и больше.
  По данным М.\,Е. Маршакова и В.\,Г. Давидова, верхней чертой термального
  равновесия человека, пребывающего в состоянии покоя, является температура
  $30$--$31^\circ C$ при относительной влажности $85\%$ или $40^\circ C$ при
  относительной влажности $30\%$.
  Эти границы меняются при выполнении физической работы.

  Согласно с результатами исследований, человека является дееспособным и
  нормально себя чувствует, если температура окружающего воздуха не выходит за
  границы $18$--$20^\circ C$, относительная влажность --- $40$--$60\%$,
  скорость движения воздуха --- $0.1$--$0.2$ м/с.

  Физическая работа в условиях повышенной температуры приводит к ускоренному
  сердцебиению, понижению артериального давления.
  При низкой температуре может случиться переохлаждение организма, которое
  приведёт к простудным заболеваниям.

  Высокая температура ослабляет организм, вызывает сонливость, а низкая ---
  сковывает движения, что при обслуживании машин повышает опасность травм.
  При высокой температуре и влажности может случиться перегрев тела, даже
  тепловой удар.

  Тепловые аппараты, которые используются на предприятиях, являются
  источниками инфракрасного излучения.
  Оно негативно влияет на функционирование нервной системы, вызывает изменения
  в сердечно-сосудистой системе, негативно влияет на глаза.

  Понижение негативного влияния микроклимата можно достичь за счёт следующих
  мер:
  \begin{itemize}
    \item защита работников с помощью разных видов экранов;
    \item эффективного планирования и конструкторского решения
    \item механизации та автоматизации производственных процессов;
    \item рациональной тепловой изоляции оборудования;
    \item рационального размещения оборудования;
      производственных помещений;
    \item рациональной вентиляции и обогрева;
    \item использование рациональных технологических процессов (например,
      замена горячего способа обработки металла холодным);
    \item рационализации режимов работы и отдыха;
    \item специального питьевого режима;
    \item дистанционное управление, что позволяет вывести человека из-под
      неблагоприятных условий;
    \item применение спецодежды.
  \end{itemize}

  Защиту от инфракрасного излучения обеспечивают специальные устройства, а так
  же знаки опасности.

  Понижение интенсивности теплового излучения достигается за счёт применения
  различных экранов, теплоизоляционных материалов, а также индивидуальными
  средствами.

  \textbf{Вывод:} в данной работе мной был рассмотрен вопрос микроклимата
  производственных помещений, его влияние на организм человека и поддержка его
  в нормативных границах.
  Было рассмотрено несколько публикаций по этой теме, на основании которых
  была предложена идея создать математическую модель, которая бы выразила
  зависимость параметров комфортности для работников и параметров микроклимата
  производства.
  Таким образом, регуляция микроклимата на производстве должно производиться с
  учётом параметров комфортности, которые были представлены в таблице №1.
  Также был описано негативное влияние микроклимата производства на организм
  человека, а также методы понижения этого влияния.
  А его обязательно необходимо уменьшать, т.\,к. чем меньше негативное
  влияние, тем больше продуктивность работы, меньше вероятность их
  заболевания, плохого самочувствия, а значит и работа предприятия более
  стабильна, прибыльная, а также работники заинтересованы в такой работе, где
  они не повредят своё здоровье.
  Поэтому все предприятия должны быть заинтересованы в улучшении их
  микроклиматических условий.


  \newpage
  \begin{center}
    \large
    \bf
    Список литературы
  \end{center}
  \begin{itemize}
    \item[[1]] Шутька О.\,В. Модель вентиляционного комплекса в задачах
      стабилизации температурного режима в помещении // Вестник КДУ им. М.
      Остроградского. --- 2010. --- Вып. 3 3/2010 (62). Ч. 2. --- С.57-62
    \item[[2]] Сукач С.\,В. Технические решения по повышению эффективности системы
      индивидуального проветривания лабораторных помещений // Вестник КДУ им.
      М. Остроградского. --- 2010. --- Вып. 3/2010 (62). Ч.2. --- С.51--55
    \item[[3]] Бондарь Е.\,С., Гордиенко А.\,С., Михайлов В.\,А. Автоматизация
      систем вентиляции и кондиционирования воздуха // Аванност-Прим, 2005.
      --- 560с.
    \item[[4]] Богословский В.\,Н., Щеглов В.\,П., Разумов Н.\,Н. Отопление и
      вентиляция (учебник для вузов) // Стройиздат. 1980. --- 180с
    \item[[5]] Жуковский С.\,С., Возняк О.\,Т., Довбуш О.\,М., Люльчак З.\,С.
      Вентиляция помещений // Львовская политехника, 2007. --- 476с.
  \end{itemize}
\end{document}
