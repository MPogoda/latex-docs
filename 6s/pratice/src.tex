\documentclass[a4paper,12pt,notitlepage,pdftex,headsepline]{scrreprt}

\usepackage{cmap} % чтобы работал поиск по PDF
\usepackage[utf8]{inputenc}
\usepackage[russian]{babel}
\usepackage[T2A]{fontenc}

\usepackage{textcase}
\usepackage[pdftex]{graphicx}

\usepackage{lscape}

\pdfcompresslevel=9 % сжимать PDF
\usepackage{pdflscape} % для возможности альбомного размещения некоторых страниц
\usepackage[pdftex]{hyperref}
% настройка ссылок в оглавлении для pdf формата
\hypersetup{unicode=true,
            pdftitle={Курсовая по МО},
            pdfauthor={Погода Михаил},
            pdfcreator={pdflatex},
            pdfsubject={},
            pdfborder    = {0 0 0},
            bookmarksopen,
            bookmarksnumbered,
            bookmarksopenlevel = 2,
            pdfkeywords={},
            colorlinks=true, % установка цвета ссылок в оглавлении
            citecolor=black,
            filecolor=black,
            linkcolor=black,
            urlcolor=blue}

\usepackage{amsmath}
\usepackage{amssymb}
\usepackage{moreverb}
%for \includepdf
%\usepackage{pdfpages}

\author{Михаил Погода}
\title{Курсовая работа по методам оптимизации}
\date{\today}

\begin{document}
  \thispagestyle{empty}
  \begin{center}
    \large
    \MakeUppercase{Министерство образования и науки,}

    \MakeUppercase{молодёжи и спорта Украины}

    \MakeUppercase{Национальный технический университет Украины}

    \MakeUppercase{,,Киевский политехнический институт''}

    \addvspace{6pt}

    \normalsize
    Кафедра прикладной математики

    \vfill

    \textbf{Отчёт с производственной практики}

    Тема: ,,Изучение и создание видеоуроков по шаблонам проектирования''
  \end{center}

  \vfill

  \hfill Выполнил\\

  \hfill студент группы КМ-92\\

  \hfill Погода М.\,В.\\

  \vfill

  \begin{center}
    КИЕВ

    2012
  \end{center}
  \clearpage
  \tableofcontents
  \clearpage

\Large
\chapter{Постановка задачи}

Общее задание практики заключалось в освоении паттернов (шаблонов)
программирования и создание интерактивных учебных материалов по ним.
В процессе создания этих учебных материалов необходимо ознакомится с
прикладным программным обеспечением \texttt{Techsmith Camtasia Studio}.
В дальнейшем эти видеолекции будут применяться в тренинг-департаменте
компании.

Полученный в процессе практики материал должен не только содержать информацию
о паттернах программирования, но и подавать её в интерактивном виде, тем самым
повышая интерес к освоению этой темы.

\chapter{Математический смысл}

Математический смысл данной работы заключается в изучении/применении шаблонов
объектно-ориентированного программирования, которые возникли как способы
решения проблем, часто всплывающих во время написания программного продукта.

В процессе проектирования программного обеспечения перед программистом часто
возникают проблемы, связанные с тем, что в дальнейшем продукт, возможно, нужно
будет расширять.
Поэтому необходимо продумать архитектура так, чтобы расширение и сопровождение
кода было проще для программиста.

Хоть проектирование программного обеспечения и является нетривиальной задачей,
решение которой требует индивидуального подхода, но некоторые части можно
реализовать, используя общепринятые подходы.

Также использование паттернов помогает в обсуждении конкретной архитектуры с
коллегами, которые также знают паттерны программирования.

На протяжение практики были изучены некоторые из них

\section{Изученные шаблоны проектирование}
\subsection{Стратегия}
Поведенческий шаблон проектирования, предназначенный для определения семейства
алгоритмов, инкапсуляции каждого из них и обеспечения их взаимозаменяемости.
Это позволяет выбирать алгоритм путем определения соответствующего класса.
Шаблон Strategy позволяет менять выбранный алгоритм независимо от
объектов-клиентов, которые его используют.
\subsection{Наблюдатель}
Определяет зависимость типа ,,один ко многим'' между объектами таким образом,
что при изменении состояния одного объекта все зависящие от него оповещаются
об этом событии.
\subsection{Синглтон}
Гарантирует, что у класса есть только один экземпляр, и предоставляет к нему
глобальную точку доступа.
Существенно то, что можно пользоваться именно экземпляром класса, так как при
этом во многих случаях становится доступной более широкая функциональность.
Например, к описанным компонентам класса можно обращаться через интерфейс,
если такая возможность поддерживается языком.
\subsection{Декоратор}
Cтруктурный шаблон проектирования, предназначенный для динамического
подключения дополнительного поведения к объекту.
Шаблон Декоратор предоставляет гибкую альтернативу практике создания
подклассов с целью расширения функциональности.
\subsection{Команда}
Создание структуры, в которой класс-отправитель и класс-получатель не зависят
друг от друга напрямую.
Организация обратного вызова к классу, который включает в себя
класс-отправитель.

С другой стороны, помимо изучения паттернов объектно-ориентированного
проектирования, во время практики нужно было изучить программное обеспечение
Techsmit Camtasia studio, которое позволяет делать интерактивные
видео-материалы.
При этом стоит учитывать синхронизацию аудио и видео, нестатическость кадра и
другие вещи.

\chapter{Отчёт по выполненной работе}

В результате выполненной работы были изучены паттерны программирования:
декоратор, синглтон, стратегия, комманда, фабрика, наблюдатель.
Также хочется отметить то, что во время практики мы решили проблему, часто
возникающую в тренинг-департаменте, связанную с плохим касеством записи звука.

\chapter{Использованная литературы}
\begin{enumerate}
  \item \url{http://mumble.sourceforge.net}
  \item Фримен Эр., Фримен Эл., Сьерра К., Бейтс Б. --- Паттерны
    проектирования --- 2011\,г.
\end{enumerate}
\end{document}
