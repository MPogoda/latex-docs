\documentclass[a4paper,12pt,notitlepage,pdftex,headsepline]{scrartcl}

\usepackage{a4wide}
\usepackage{cmap} % чтобы работал поиск по PDF
\usepackage[utf8]{inputenc}
\usepackage[russian]{babel}
\usepackage[T2A]{fontenc}

\usepackage{textcase}
\usepackage[pdftex]{graphicx}

\usepackage{lscape}

\pdfcompresslevel=9 % сжимать PDF
\usepackage{pdflscape} % для возможности альбомного размещения некоторых страниц
\usepackage[pdftex]{hyperref}
% настройка ссылок в оглавлении для pdf формата
\hypersetup{unicode=true,
            pdftitle={Курсовая по МО},
            pdfauthor={Погода Михаил},
            pdfcreator={pdflatex},
            pdfsubject={},
            pdfborder    = {0 0 0},
            bookmarksopen,
            bookmarksnumbered,
            bookmarksopenlevel = 2,
            pdfkeywords={},
            colorlinks=true, % установка цвета ссылок в оглавлении
            citecolor=black,
            filecolor=black,
            linkcolor=black,
            urlcolor=blue}

\usepackage{amsmath}
\usepackage{amssymb}
\usepackage{moreverb}
%for \includepdf
%\usepackage{pdfpages}

\author{Михаил Погода}
\title{Курсовая работа по методам оптимизации}
\date{\today}

\begin{document}
  \thispagestyle{empty}
  \begin{center}
    \large
    \MakeUppercase{Министерство образования и науки,}

    \MakeUppercase{молодёжи и спорта Украины}

    \MakeUppercase{Национальный технический университет Украины}

    \MakeUppercase{,,Киевский политехнический институт''}

    \addvspace{6pt}

    \normalsize
    Кафедра прикладной математики

    \vfill

    \textbf{Курсовой проект}

    по дисциплине ,,Вычислительные системы''
  \end{center}

  \vfill

  Руководитель: Копычко С.\,Н.\hfill Выполнил\\
  Допущен к защите:\hfill студент группы КМ-92\\
  ``\underline{\hspace{0.5cm}}'' \underline{\hspace{5cm}} 2012 г.\hfill Погода
  М.\,В.\\
  Защищена с оценкой:\hfill Номер зачётной книжки:\\
  \underline{\hspace{7cm}} \hfill КМ-9214\\
  ``\underline{\hspace{0.6cm}}'' \underline{\hspace{5cm}} 2012 г.

  \vfill

  \begin{center}
    КИЕВ

    2012
  \end{center}
  \clearpage
  \tableofcontents
  \clearpage

\section{Постановка задачи}
  Исследовать сходимость метода Розенброка для следующих тестовых функций:
  \begin{itemize}
    \item Функция Розенброка:
      \[
        f\left(x\right) = 100 \left(x_1^2 - x_2\right)^2 + \left(x_1 - 1\right)^2
      \]
    \item Степенная функция:
      \[
        f\left(x\right) = \left(10 \left(x_1 - x_2\right)^2 + \left(x_1 - 1
        \right)^2 \right)^4
      \]
    \item Корневая функция:
      \[
        f\left(x\right) = \left(10 \left(x_1 - x_2\right)^2 + \left(x_1 - 1
        \right)^2 \right)^\frac{1}{4}
      \]
  \end{itemize}

  Использовать начальную точку $x^0 = \left(\begin{matrix}-1.2 \\
    0\end{matrix}\right)$

  Рассмотреть две обычный метод Розенброка и его версию, модифицированную
  Дэвисом, Свенном и Кемпи.

  Для немодифицированной версии исследовать зависимость от начального
  приращения $\lambda$.

  Для модифицированной рассмотреть следующие виды одномерного поиска:
  \begin{itemize}
    \item Дэвиса-Свенна-Кемпи-Пауэлла;
    \item золотого сечения.
  \end{itemize}
  Для них необходимо исследовать зависимость схождения метода Розенброка от
  точности методов одномерного поиска.

  Исследовать метод Розенброка при решение задачи условной оптимизации,
  используя метод штрафных функций.
  Рассмотреть области с линейными ограничениями, с выпуклой и невыпуклой
  областями.

  \clearpage

\section{Теоретическая часть}
  Метод Розенброка имеет сходство с этапом исследующего поиска метода
  Хука-Дживса.
  Однако, вместо непрерывного поиска по координатам, соответствующим
  направлениям независимых переменных, после каждого цикла покоординатного
  поиска строится новая ортогональная система направлений поиска.
  Причём весь шаг предыдущего этапа принимается в качестве первого блока при
  построении новой системы направлений.

  Метод Розенброка определяет местонахождение точки $x^{\left(k+1\right)}$,
  используя последовательные одномерные поиски, начиная с исходной точки
  $x^k$, вдоль системы ортогональных направлений $S_1^k, S_2^k, \dots, S_n^k$.

  Пусть $S_1^k, S_2^k, \dots, S_n^k$ --- единичные вектора, $k$ --- номер
  этапа поиска.
  На начальном этапе задаётся начальная точка $x^0$, $\lambda$ --- вектор
  приращений, $\varepsilon$ --- точность вычислений.
  На начальном этапе $k = 0$, направления $S_1^0, S_2^0, \dots, S_n^0$ обычно
  берутся параллельными осям $x_1, \dots, x_n$.

  Рассмотрим $k$-й этап и пусть $x^k$ представляет собой точку, из которой
  начат поиск, а $\lambda_1, \dots, \lambda_n$ --- соответствующие длины
  шагов, связанные с направлениями $S_1^k, S_2^k, \dots, S_n^k$

  Поиск начинается из $x^k$ путём изменения $x^k$ на шаг $\lambda_1$ в
  направлении $S_1^k$.
  Вычисляется значение функции в этой точке.
  Если значение $f\left(x^k + \lambda_1 S_1^k\right) \leqslant
  f\left(x^k\right)$, шаг считается успешным:
  \begin{enumerate}
    \item Полученная пробная точка заменяет точку $x^k$.
    \item $\lambda_1$ умножается на множитель $\alpha > 0$.
  \end{enumerate}

  Если же $f\left(x^k + \lambda_1 S_1^k\right) > f\left(x^k\right)$ ---
  шаг считается неудачным:
  \begin{enumerate}
    \item $x^k$ - не заменяется;
    \item $\lambda_1$ умножается на множитель $\beta < 0$
  \end{enumerate}

  Розенброк предложил в общем случае брать $\alpha = 3, \beta = -\frac{1}{2}$.

  Далее аналогично производятся вычисления значения функции последовательно по
  всем остальным направлениям $S_2^k, \dots, S_n^k$.

  После того, как пройдены все направления снова возвращаются к первому
  направлению.
  Эти итерации проводятся до тех пор, пока по каждому из направлений за
  успехом не последует неудача.

  При этом $k$-й этап поиска заканчивается. Последняя полученная точка
  становится начальной точкой следующего этапа $x_0^{k+1}$.
  Нормированное направление $S_1^{k+1}$ берётся параллельным $(x_0^{k+1} -
  x_0^k)$, а остальные направления выбираются ортогональными друг к другу.

  Пусть $\Lambda_i^k$ --- алгебраическая сумма всех успешных шагов (суммарное
  перемещение) в направлении $S_i^k$ на $k$-ом этапе.

  Определим $n$ векторов $A_1, \dots, A_n$ следующим образом:
  \begin{align*}
    A_1^k &= \sum_{i = 1}^k\Lambda_i^k S_i^k\\
    A_2^k &= \sum_{i = 2}^k\Lambda_i^k S_i^k\\
    \dots\\
    A_n^k &= \Lambda_n^k S_n^k
  \end{align*}

  Таким образом вектор $A_1^k$ является вектором перехода из $x_0^k$ в
  $x_0^{k+1}$, полное перемещение с $k$-го этапа на $(k+1)$-й этап.
  $A_2^k$ --- полное перемещение, но без учёта продвижения, сделанного во
  время поиска в направлении $S_1^k$ и т.\,д.

  Новые направления определяются следующим образом:
  \begin{align}
    S_i^{k+1} &= \frac{A_1^k}{\left\|A_1^k\right\|} \\
    B_n^k &= A_n^k - \sum_{i = 1}^{n-1} \left( \left( A_n^k \right)^T
    S_i^{k+1} \right)S_i^{k + 1} \\
    S_n^{k+1} &= \frac{B_n^k}{\left\| B_n^k \right\|}
  \end{align}

  Та же процедура поиска, которая проводилась на k-м этапе, повторяется затем
  на $(k+1)$-ом этапе.
  Поиск заканчивается, когда изменения значений функции и компонент вектора
  $x$ будут меньше заданной точности или $\left\| A_1 \right\| < \varepsilon$.

  Дэвис, Свенн и Кэмпи модифицировали поиск Розенброка в направлениях
  $S_1^k,\dots, S_n^k$ путём отыскания минимума $f(x)$ в каждом из направлений
  с помощью методов одномерного поиска.

  Метод Розенброка эффективно применяется для функций, имеющих узкий искривлённый гребень.
  Поиск по $n$ взаимоортогональным направлениям обеспечивает то, что
  результирующее направление стремится расположиться вдоль оси оврага.

  Задача минимизации $f(x), x\in\mathbb{R}^n$ при ограничениях $g_i(x)
  \geqslant 0$ была преобразована Розенброком таким образом, что оказался
  применим стандартный метод Розенброка.
  Решение этой задачи эквивалентно нахождению минимума не связанной никакими
  ограничениями присоединённой функции:
  \begin{equation}
    P(x) = f(x) \prod_{i = 1}^p U_i(x) g_i(x),
  \end{equation}
  где
  \[
    U_i(x) =
    \begin{cases}
      0 &, g_i(x) < 0\\
      1 &, g_i(x) \geqslant 0
    \end{cases}
  \],
  а функция $f(x)$ принимает только отрицательные значения на всей области.

  \clearpage
\section{Расчётная часть}
  \subsection{Безусловная оптимизация}
    \subsubsection{Стандартный метод Розенброка}
        \begin{table}[ht]
          \centering
          \caption{Функция Розенброка, $\lambda_0 = (1; 1)^T$}
          \begin{tabular}{|c|c|c|c|c|}
            \hline
            $\varepsilon$ & К.В.Ф. & $x_1$ & $x_2$ & $f$\\
            \hline
             $10^{-0}$ & $20$ & $-0.186842$ & $0.039839$ & $1.411024\cdot 10^{+00}$\\
             $10^{-1}$ & $20$ & $-0.186842$ & $0.039839$ & $1.411024\cdot 10^{+00}$\\
             $10^{-2}$ & $145$ & $0.784603$ & $0.614994$ & $4.643278\cdot 10^{-02}$\\
             $10^{-3}$ & $212$ & $0.999461$ & $0.998901$ & $3.354919\cdot 10^{-07}$\\
             $10^{-4}$ & $237$ & $0.999997$ & $0.999992$ & $3.407808\cdot 10^{-10}$\\
             $10^{-5}$ & $250$ & $0.999997$ & $0.999994$ & $9.713269\cdot 10^{-12}$\\
             $10^{-6}$ & $267$ & $0.999997$ & $0.999994$ & $8.521997\cdot 10^{-12}$\\
             $10^{-7}$ & $267$ & $0.999997$ & $0.999994$ & $8.521997\cdot 10^{-12}$\\
             $10^{-8}$ & $337$ & $1.000000$ & $1.000000$ & $3.951588\cdot 10^{-14}$\\
             $10^{-9}$ & $345$ & $1.000000$ & $1.000000$ & $3.934783\cdot 10^{-14}$\\
            \hline
          \end{tabular}
        \end{table}

        \begin{table}[ht]
          \centering
          \caption{Функция Розенброка, $\lambda_0 = (0.5; 0.5)^T$}
          \begin{tabular}{|c|c|c|c|c|}
            \hline
            $\varepsilon$ & К.В.Ф. & $x_1$ & $x_2$ & $f$\\
            \hline
             $10^{-0}$ & $23$ & $0.768736$ & $0.588819$ & $5.393915\cdot 10^{-02}$\\
             $10^{-1}$ & $23$ & $0.768736$ & $0.588819$ & $5.393915\cdot 10^{-02}$\\
             $10^{-2}$ & $37$ & $0.768721$ & $0.589688$ & $5.364481\cdot 10^{-02}$\\
             $10^{-3}$ & $37$ & $0.768721$ & $0.589688$ & $5.364481\cdot 10^{-02}$\\
             $10^{-4}$ & $211$ & $1.000010$ & $1.000019$ & $1.304123\cdot 10^{-10}$\\
             $10^{-5}$ & $229$ & $0.999999$ & $0.999997$ & $2.075585\cdot 10^{-12}$\\
             $10^{-6}$ & $242$ & $1.000000$ & $1.000000$ & $5.695517\cdot 10^{-14}$\\
             $10^{-7}$ & $261$ & $1.000000$ & $1.000000$ & $4.974044\cdot 10^{-15}$\\
             $10^{-8}$ & $272$ & $1.000000$ & $1.000000$ & $4.556399\cdot 10^{-15}$\\
             $10^{-9}$ & $283$ & $1.000000$ & $1.000000$ & $4.490266\cdot 10^{-15}$\\
            \hline
          \end{tabular}
        \end{table}

        \begin{table}[ht]
          \centering
          \caption{Функция Розенброка, $\lambda_0 = (0.1; 0.1)^T$}
          \begin{tabular}{|c|c|c|c|c|}
            \hline
            $\varepsilon$ & К.В.Ф. & $x_1$ & $x_2$ & $f$\\
            \hline
            $10^{-0}$ & $7$ & $-0.800000$ & $0.400000$ & $9.000000\cdot 10^{+00}$\\
            $10^{-1}$ & $34$ & $0.347830$ & $0.114861$ & $4.290766\cdot 10^{-01}$\\
            $10^{-2}$ & $34$ & $0.347830$ & $0.114861$ & $4.290766\cdot 10^{-01}$\\
            $10^{-3}$ & $128$ & $0.978056$ & $0.956338$ & $4.880663\cdot 10^{-04}$\\
            $10^{-4}$ & $202$ & $0.999710$ & $0.999417$ & $8.478705\cdot 10^{-08}$\\
            $10^{-5}$ & $236$ & $0.999991$ & $0.999983$ & $7.276444\cdot 10^{-11}$\\
            $10^{-6}$ & $236$ & $0.999991$ & $0.999983$ & $7.276444\cdot 10^{-11}$\\
            $10^{-7}$ & $245$ & $0.999992$ & $0.999983$ & $7.256107\cdot 10^{-11}$\\
            $10^{-8}$ & $353$ & $1.000000$ & $1.000000$ & $2.361478\cdot 10^{-18}$\\
            $10^{-9}$ & $353$ & $1.000000$ & $1.000000$ & $2.361478\cdot 10^{-18}$\\
            \hline
          \end{tabular}
        \end{table}

        \begin{table}[ht]
          \centering
          \caption{Степенная функция, $\lambda_0 = (1; 1)^T$}
          \begin{tabular}{|c|c|c|c|c|}
            \hline
            $\varepsilon$ & К.В.Ф. & $x_1$ & $x_2$ & $f$\\
            \hline
            $10^{-0}$ & $16$ & $-0.141859$ & $-0.037788$ & $1.973919\cdot 10^{+00}$\\
            $10^{-1}$ & $67$ & $1.000595$ & $0.999829$ & $1.210167\cdot 10^{-21}$\\
            $10^{-2}$ & $67$ & $1.000595$ & $0.999829$ & $1.210167\cdot 10^{-21}$\\
            $10^{-3}$ & $86$ & $1.001149$ & $1.001274$ & $1.005596\cdot 10^{-23}$\\
            $10^{-4}$ & $121$ & $1.000001$ & $0.999999$ & $1.131663\cdot 10^{-41}$\\
            $10^{-5}$ & $121$ & $1.000001$ & $0.999999$ & $1.131663\cdot 10^{-41}$\\
            $10^{-6}$ & $138$ & $1.000001$ & $1.000001$ & $2.616960\cdot 10^{-48}$\\
            $10^{-7}$ & $167$ & $1.000000$ & $1.000000$ & $6.116104\cdot 10^{-61}$\\
            $10^{-8}$ & $186$ & $1.000000$ & $1.000000$ & $3.769300\cdot 10^{-69}$\\
            $10^{-9}$ & $209$ & $1.000000$ & $1.000000$ & $3.507822\cdot 10^{-73}$\\
            \hline
          \end{tabular}
        \end{table}

        \begin{table}[ht]
          \centering
          \caption{Степенная функция, $\lambda_0 = (0.5; 0.5)^T$}
          \begin{tabular}{|c|c|c|c|c|}
            \hline
            $\varepsilon$ & К.В.Ф. & $x_1$ & $x_2$ & $f$\\
            \hline
            $10^{-0}$ & $5$ & $-0.700000$ & $-0.250000$ & $1.656409\cdot 10^{+02}$\\
            $10^{-1}$ & $22$ & $-0.263159$ & $-0.179688$ & $4.560394\cdot 10^{+00}$\\
            $10^{-2}$ & $63$ & $0.988353$ & $0.987364$ & $8.242238\cdot 10^{-16}$\\
            $10^{-3}$ & $105$ & $0.999943$ & $0.999947$ & $7.812551\cdot 10^{-35}$\\
            $10^{-4}$ & $125$ & $0.999997$ & $0.999997$ & $3.879384\cdot 10^{-45}$\\
            $10^{-5}$ & $132$ & $0.999999$ & $0.999999$ & $1.209450\cdot 10^{-47}$\\
            $10^{-6}$ & $147$ & $1.000000$ & $1.000000$ & $3.396068\cdot 10^{-57}$\\
            $10^{-7}$ & $159$ & $1.000000$ & $1.000000$ & $1.282180\cdot 10^{-63}$\\
            $10^{-8}$ & $194$ & $1.000000$ & $1.000000$ & $4.677116\cdot 10^{-85}$\\
            $10^{-9}$ & $194$ & $1.000000$ & $1.000000$ & $4.677116\cdot 10^{-85}$\\
            \hline
          \end{tabular}
        \end{table}

        \begin{table}[ht]
          \centering
          \caption{Степенная функция, $\lambda_0 = (0.1; 0.1)^T$}
          \begin{tabular}{|c|c|c|c|c|}
            \hline
            $\varepsilon$ & К.В.Ф. & $x_1$ & $x_2$ & $f$\\
            \hline
            $10^{-0}$ & $23$ & $0.119547$ & $0.183796$ & $2.505117\cdot 10^{-01}$\\
            $10^{-1}$ & $27$ & $0.185244$ & $0.210055$ & $1.576988\cdot 10^{-01}$\\
            $10^{-2}$ & $74$ & $0.995340$ & $0.995027$ & $4.372738\cdot 10^{-19}$\\
            $10^{-3}$ & $83$ & $0.995490$ & $0.995554$ & $1.538639\cdot 10^{-19}$\\
            $10^{-4}$ & $127$ & $1.000012$ & $1.000010$ & $2.627744\cdot 10^{-40}$\\
            $10^{-5}$ & $144$ & $1.000000$ & $1.000000$ & $4.306813\cdot 10^{-52}$\\
            $10^{-6}$ & $144$ & $1.000000$ & $1.000000$ & $4.306813\cdot 10^{-52}$\\
            $10^{-7}$ & $161$ & $1.000000$ & $1.000000$ & $4.217620\cdot 10^{-56}$\\
            $10^{-8}$ & $204$ & $1.000000$ & $1.000000$ & $1.954925\cdot 10^{-79}$\\
            $10^{-9}$ & $204$ & $1.000000$ & $1.000000$ & $1.954925\cdot 10^{-79}$\\
            \hline
          \end{tabular}
        \end{table}

        \begin{table}[ht]
          \centering
          \caption{Корневая функция, $\lambda_0 = (1; 1)^T$}
          \begin{tabular}{|c|c|c|c|c|}
            \hline
            $\varepsilon$ & К.В.Ф. & $x_1$ & $x_2$ & $f$\\
            \hline
            $10^{-0}$ & $15$ & $-0.002320$ & $-0.055230$ & $1.008062\cdot 10^{+00}$\\
            $10^{-1}$ & $62$ & $1.057236$ & $1.058977$ & $2.397923\cdot 10^{-01}$\\
            $10^{-2}$ & $84$ & $1.001712$ & $1.001504$ & $4.282080\cdot 10^{-02}$\\
            $10^{-3}$ & $106$ & $1.000209$ & $1.000220$ & $1.456571\cdot 10^{-02}$\\
            $10^{-4}$ & $124$ & $1.000006$ & $1.000004$ & $2.797184\cdot 10^{-03}$\\
            $10^{-5}$ & $137$ & $1.000004$ & $1.000003$ & $2.204949\cdot 10^{-03}$\\
            $10^{-6}$ & $152$ & $1.000001$ & $1.000002$ & $1.286883\cdot 10^{-03}$\\
            $10^{-7}$ & $186$ & $1.000000$ & $1.000000$ & $6.564279\cdot 10^{-04}$\\
            $10^{-8}$ & $263$ & $1.000000$ & $1.000000$ & $2.928505\cdot 10^{-05}$\\
            $10^{-9}$ & $275$ & $1.000000$ & $1.000000$ & $1.325684\cdot 10^{-05}$\\
            \hline
          \end{tabular}
        \end{table}

        \begin{table}[ht]
          \centering
          \caption{Корневая функция, $\lambda_0 = (0.5; 0.5)^T$}
          \begin{tabular}{|c|c|c|c|c|}
            \hline
            $\varepsilon$ & К.В.Ф. & $x_1$ & $x_2$ & $f$\\
            \hline
            $10^{-0}$ & $19$ & $0.147921$ & $0.098268$ & $9.308197\cdot 10^{-01}$\\
            $10^{-1}$ & $33$ & $0.369271$ & $0.354445$ & $7.952794\cdot 10^{-01}$\\
            $10^{-2}$ & $83$ & $0.995746$ & $0.995858$ & $6.534021\cdot 10^{-02}$\\
            $10^{-3}$ & $105$ & $0.998628$ & $0.998430$ & $3.883157\cdot 10^{-02}$\\
            $10^{-4}$ & $142$ & $0.999943$ & $0.999939$ & $7.663579\cdot 10^{-03}$\\
            $10^{-5}$ & $142$ & $0.999943$ & $0.999939$ & $7.663579\cdot 10^{-03}$\\
            $10^{-6}$ & $221$ & $1.000000$ & $1.000000$ & $5.952182\cdot 10^{-04}$\\
            $10^{-7}$ & $241$ & $1.000000$ & $1.000000$ & $1.280182\cdot 10^{-04}$\\
            $10^{-8}$ & $257$ & $1.000000$ & $1.000000$ & $2.851789\cdot 10^{-05}$\\
            $10^{-9}$ & $270$ & $1.000000$ & $1.000000$ & $2.312888\cdot 10^{-05}$\\
            \hline
          \end{tabular}
        \end{table}

        \begin{table}[ht]
          \centering
          \caption{Корневая функция, $\lambda_0 = (0.1; 0.1)^T$}
          \begin{tabular}{|c|c|c|c|c|}
            \hline
            $\varepsilon$ & К.В.Ф. & $x_1$ & $x_2$ & $f$\\
            \hline
            $10^{-0}$ & $19$ & $0.147921$ & $0.098268$ & $9.308197\cdot 10^{-01}$\\
            $10^{-1}$ & $33$ & $0.369271$ & $0.354445$ & $7.952794\cdot 10^{-01}$\\
            $10^{-2}$ & $83$ & $0.995746$ & $0.995858$ & $6.534021\cdot 10^{-02}$\\
            $10^{-3}$ & $105$ & $0.998628$ & $0.998430$ & $3.883157\cdot 10^{-02}$\\
            $10^{-4}$ & $142$ & $0.999943$ & $0.999939$ & $7.663579\cdot 10^{-03}$\\
            $10^{-5}$ & $142$ & $0.999943$ & $0.999939$ & $7.663579\cdot 10^{-03}$\\
            $10^{-6}$ & $221$ & $1.000000$ & $1.000000$ & $5.952182\cdot 10^{-04}$\\
            $10^{-7}$ & $241$ & $1.000000$ & $1.000000$ & $1.280182\cdot 10^{-04}$\\
            $10^{-8}$ & $257$ & $1.000000$ & $1.000000$ & $2.851789\cdot 10^{-05}$\\
            $10^{-9}$ & $270$ & $1.000000$ & $1.000000$ & $2.312888\cdot 10^{-05}$\\
            \hline
          \end{tabular}
        \end{table}

        \clearpage
    \subsubsection{Модифицированный метод с ДСК-Пауэлла}
        \begin{table}[ht]
          \centering
          \caption{Функция Розенброка, $\Delta = 1$}
          \begin{tabular}{|c|c|c|c|c|}
            \hline
            $\varepsilon$ & К.В.Ф. & $x_1$ & $x_2$ & $f$\\
            \hline
            $10^{-0}$ & $25$ & $0.208367$ & $0.043364$ & $6.266828\cdot 10^{-01}$\\
            $10^{-1}$ & $27$ & $0.208367$ & $0.043364$ & $6.266828\cdot 10^{-01}$\\
            $10^{-2}$ & $203$ & $0.907025$ & $0.818619$ & $1.030592\cdot 10^{-02}$\\
            $10^{-3}$ & $317$ & $0.999804$ & $0.999552$ & $3.599786\cdot 10^{-07}$\\
            $10^{-4}$ & $406$ & $0.999993$ & $0.999976$ & $1.009787\cdot 10^{-08}$\\
            $10^{-5}$ & $492$ & $0.999996$ & $0.999997$ & $2.872348\cdot 10^{-09}$\\
            $10^{-6}$ & $552$ & $1.000000$ & $1.000000$ & $1.152505\cdot 10^{-14}$\\
            $10^{-7}$ & $608$ & $1.000000$ & $1.000000$ & $1.147806\cdot 10^{-14}$\\
            $10^{-8}$ & $659$ & $1.000000$ & $1.000000$ & $2.207523\cdot 10^{-16}$\\
            $10^{-9}$ & $773$ & $1.000000$ & $1.000000$ & $1.244921\cdot 10^{-30}$\\
            \hline
          \end{tabular}
        \end{table}

        \begin{table}[ht]
          \centering
          \caption{Функция Розенброка, $\Delta = 0.5$}
          \begin{tabular}{|c|c|c|c|c|}
            \hline
            $\varepsilon$ & К.В.Ф. & $x_1$ & $x_2$ & $f$\\
            \hline
            $10^{-0}$ & $25$ & $0.132554$ & $0.017473$ & $7.524635\cdot 10^{-01}$\\
            $10^{-1}$ & $28$ & $0.160250$ & $0.025630$ & $7.051798\cdot 10^{-01}$\\
            $10^{-2}$ & $159$ & $0.780579$ & $0.603399$ & $5.163253\cdot 10^{-02}$\\
            $10^{-3}$ & $298$ & $0.998052$ & $0.995669$ & $2.296535\cdot 10^{-05}$\\
            $10^{-4}$ & $393$ & $0.999993$ & $0.999975$ & $1.002948\cdot 10^{-08}$\\
            $10^{-5}$ & $482$ & $0.999999$ & $1.000000$ & $3.274505\cdot 10^{-10}$\\
            $10^{-6}$ & $548$ & $1.000000$ & $1.000000$ & $1.153210\cdot 10^{-14}$\\
            $10^{-7}$ & $608$ & $1.000000$ & $1.000000$ & $1.152972\cdot 10^{-14}$\\
            $10^{-8}$ & $671$ & $1.000000$ & $1.000000$ & $3.818815\cdot 10^{-15}$\\
            $10^{-9}$ & $768$ & $1.000000$ & $1.000000$ & $0.000000\cdot 10^{+00}$\\
            \hline
          \end{tabular}
        \end{table}

        \begin{table}[ht]
          \centering
          \caption{Функция Розенброка, $\Delta = 0.1$}
          \begin{tabular}{|c|c|c|c|c|}
            \hline
            $\varepsilon$ & К.В.Ф. & $x_1$ & $x_2$ & $f$\\
            \hline
            $10^{-0}$ & $27$ & $0.139435$ & $0.000000$ & $7.783720\cdot 10^{-01}$\\
            $10^{-1}$ & $27$ & $0.158041$ & $0.000000$ & $7.712798\cdot 10^{-01}$\\
            $10^{-2}$ & $192$ & $0.968303$ & $0.935514$ & $1.444570\cdot 10^{-03}$\\
            $10^{-3}$ & $265$ & $0.999958$ & $0.999937$ & $4.041542\cdot 10^{-08}$\\
            $10^{-4}$ & $321$ & $0.999993$ & $0.999976$ & $1.009854\cdot 10^{-08}$\\
            $10^{-5}$ & $392$ & $0.999999$ & $1.000000$ & $6.078351\cdot 10^{-10}$\\
            $10^{-6}$ & $435$ & $1.000000$ & $1.000000$ & $1.150164\cdot 10^{-14}$\\
            $10^{-7}$ & $485$ & $1.000000$ & $1.000000$ & $1.148389\cdot 10^{-14}$\\
            $10^{-8}$ & $523$ & $1.000000$ & $1.000000$ & $8.160856\cdot 10^{-16}$\\
            $10^{-9}$ & $604$ & $1.000000$ & $1.000000$ & $2.454388\cdot 10^{-26}$\\
            \hline
          \end{tabular}
        \end{table}

        \begin{table}[ht]
          \centering
          \caption{Степенная функция, $\Delta = 1$}
          \begin{tabular}{|c|c|c|c|c|}
            \hline
            $\varepsilon$ & К.В.Ф. & $x_1$ & $x_2$ & $f$\\
            \hline
            $10^{-0}$ & $28$ & $0.207210$ & $0.396765$ & $2.735562\cdot 10^{-01}$\\
            $10^{-1}$ & $81$ & $0.986985$ & $1.010333$ & $9.542616\cdot 10^{-10}$\\
            $10^{-2}$ & $101$ & $0.996530$ & $1.001222$ & $2.414886\cdot 10^{-15}$\\
            $10^{-3}$ & $123$ & $1.000156$ & $0.999583$ & $1.427558\cdot 10^{-22}$\\
            $10^{-4}$ & $174$ & $1.000011$ & $0.999984$ & $2.668299\cdot 10^{-33}$\\
            $10^{-5}$ & $193$ & $0.999998$ & $0.999997$ & $2.731421\cdot 10^{-43}$\\
            $10^{-6}$ & $208$ & $1.000000$ & $1.000000$ & $5.566580\cdot 10^{-54}$\\
            $10^{-7}$ & $223$ & $1.000000$ & $1.000000$ & $1.087890\cdot 10^{-57}$\\
            $10^{-8}$ & $301$ & $1.000000$ & $1.000000$ & $1.207079\cdot 10^{-70}$\\
            $10^{-9}$ & $320$ & $1.000000$ & $1.000000$ & $1.623805\cdot 10^{-76}$\\
            \hline
          \end{tabular}
        \end{table}

        \begin{table}[ht]
          \centering
          \caption{Степенная функция, $\Delta = 0.5$}
          \begin{tabular}{|c|c|c|c|c|}
            \hline
            $\varepsilon$ & К.В.Ф. & $x_1$ & $x_2$ & $f$\\
            \hline
            $10^{-0}$ & $27$ & $0.119009$ & $0.134863$ & $3.180561\cdot 10^{-01}$\\
            $10^{-1}$ & $75$ & $0.996220$ & $0.985170$ & $4.311335\cdot 10^{-12}$\\
            $10^{-2}$ & $95$ & $0.996822$ & $1.002093$ & $6.345545\cdot 10^{-15}$\\
            $10^{-3}$ & $118$ & $1.000159$ & $0.999586$ & $1.422048\cdot 10^{-22}$\\
            $10^{-4}$ & $167$ & $1.000010$ & $0.999985$ & $1.841569\cdot 10^{-33}$\\
            $10^{-5}$ & $185$ & $0.999998$ & $0.999997$ & $5.863951\cdot 10^{-43}$\\
            $10^{-6}$ & $201$ & $1.000000$ & $1.000000$ & $6.222517\cdot 10^{-65}$\\
            $10^{-7}$ & $215$ & $1.000000$ & $1.000000$ & $1.047149\cdot 10^{-57}$\\
            $10^{-8}$ & $290$ & $1.000000$ & $1.000000$ & $3.322610\cdot 10^{-70}$\\
            $10^{-9}$ & $303$ & $1.000000$ & $1.000000$ & $9.147221\cdot 10^{-77}$\\
            \hline
          \end{tabular}
        \end{table}

        \begin{table}[ht]
          \centering
          \caption{Степенная функция, $\Delta = 0.1$}
          \begin{tabular}{|c|c|c|c|c|}
            \hline
            $\varepsilon$ & К.В.Ф. & $x_1$ & $x_2$ & $f$\\
            \hline
            $10^{-0}$ & $29$ & $0.172522$ & $0.234442$ & $1.520219\cdot 10^{-01}$\\
            $10^{-1}$ & $56$ & $0.990787$ & $0.979981$ & $6.053477\cdot 10^{-12}$\\
            $10^{-2}$ & $84$ & $1.000015$ & $0.997320$ & $4.053914\cdot 10^{-17}$\\
            $10^{-3}$ & $127$ & $0.999889$ & $0.999997$ & $1.890096\cdot 10^{-28}$\\
            $10^{-4}$ & $147$ & $1.000012$ & $1.000013$ & $1.042252\cdot 10^{-39}$\\
            $10^{-5}$ & $165$ & $1.000001$ & $0.999999$ & $6.727702\cdot 10^{-42}$\\
            $10^{-6}$ & $178$ & $1.000000$ & $1.000000$ & $7.418560\cdot 10^{-52}$\\
            $10^{-7}$ & $198$ & $1.000000$ & $1.000000$ & $3.641494\cdot 10^{-57}$\\
            $10^{-8}$ & $263$ & $1.000000$ & $1.000000$ & $4.948449\cdot 10^{-64}$\\
            $10^{-9}$ & $287$ & $1.000000$ & $1.000000$ & $4.075654\cdot 10^{-76}$\\
            \hline
          \end{tabular}
        \end{table}

        \begin{table}[ht]
          \centering
          \caption{Корневая функция, $\Delta = 1$}
          \begin{tabular}{|c|c|c|c|c|}
            \hline
            $\varepsilon$ & К.В.Ф. & $x_1$ & $x_2$ & $f$\\
            \hline
            $10^{-0}$ & $22$ & $-0.015417$ & $0.000000$ & $1.008259\cdot 10^{+00}$\\
            $10^{-1}$ & $27$ & $0.110707$ & $0.049015$ & $9.541698\cdot 10^{-01}$\\
            $10^{-2}$ & $46$ & $0.190048$ & $0.100191$ & $9.264722\cdot 10^{-01}$\\
            $10^{-3}$ & $135$ & $0.999888$ & $0.999878$ & $1.078652\cdot 10^{-02}$\\
            $10^{-4}$ & $163$ & $0.999997$ & $0.999997$ & $1.783311\cdot 10^{-03}$\\
            $10^{-5}$ & $201$ & $1.000000$ & $1.000000$ & $1.131792\cdot 10^{-04}$\\
            $10^{-6}$ & $218$ & $1.000000$ & $1.000000$ & $3.423233\cdot 10^{-05}$\\
            $10^{-7}$ & $239$ & $1.000000$ & $1.000000$ & $3.216867\cdot 10^{-06}$\\
            $10^{-8}$ & $329$ & $1.000000$ & $1.000000$ & $3.475771\cdot 10^{-07}$\\
            $10^{-9}$ & $353$ & $1.000000$ & $1.000000$ & $3.747444\cdot 10^{-08}$\\
            \hline
          \end{tabular}
        \end{table}

        \begin{table}[ht]
          \centering
          \caption{Корневая функция, $\Delta = 0.5$}
          \begin{tabular}{|c|c|c|c|c|}
            \hline
            $\varepsilon$ & К.В.Ф. & $x_1$ & $x_2$ & $f$\\
            \hline
            $10^{-0}$ & $23$ & $0.182428$ & $0.192595$ & $9.045462\cdot 10^{-01}$\\
            $10^{-1}$ & $28$ & $0.156228$ & $0.129750$ & $9.208237\cdot 10^{-01}$\\
            $10^{-2}$ & $88$ & $0.997659$ & $0.997166$ & $5.303303\cdot 10^{-02}$\\
            $10^{-3}$ & $138$ & $1.000006$ & $1.000027$ & $8.156093\cdot 10^{-03}$\\
            $10^{-4}$ & $165$ & $1.000000$ & $1.000000$ & $6.843736\cdot 10^{-04}$\\
            $10^{-5}$ & $190$ & $1.000000$ & $1.000000$ & $2.332516\cdot 10^{-04}$\\
            $10^{-6}$ & $223$ & $1.000000$ & $1.000000$ & $8.011146\cdot 10^{-06}$\\
            $10^{-7}$ & $237$ & $1.000000$ & $1.000000$ & $7.919820\cdot 10^{-07}$\\
            $10^{-8}$ & $327$ & $1.000000$ & $1.000000$ & $6.196378\cdot 10^{-08}$\\
            $10^{-9}$ & $345$ & $1.000000$ & $1.000000$ & $0.000000\cdot 10^{+00}$\\
            \hline
          \end{tabular}
        \end{table}

        \begin{table}[ht]
          \centering
          \caption{Корневая функция, $\Delta = 0.1$}
          \begin{tabular}{|c|c|c|c|c|}
            \hline
            $\varepsilon$ & К.В.Ф. & $x_1$ & $x_2$ & $f$\\
            \hline
            $10^{-0}$ & $28$ & $0.255348$ & $0.248731$ & $8.631025\cdot 10^{-01}$\\
            $10^{-1}$ & $58$ & $1.007387$ & $1.012525$ & $1.336039\cdot 10^{-01}$\\
            $10^{-2}$ & $81$ & $0.997641$ & $0.997543$ & $4.877890\cdot 10^{-02}$\\
            $10^{-3}$ & $99$ & $0.999861$ & $0.999936$ & $1.653975\cdot 10^{-02}$\\
            $10^{-4}$ & $154$ & $1.000000$ & $1.000000$ & $5.469622\cdot 10^{-04}$\\
            $10^{-5}$ & $182$ & $1.000000$ & $1.000000$ & $4.169904\cdot 10^{-05}$\\
            $10^{-6}$ & $193$ & $1.000000$ & $1.000000$ & $5.368007\cdot 10^{-05}$\\
            $10^{-7}$ & $208$ & $1.000000$ & $1.000000$ & $2.746928\cdot 10^{-06}$\\
            $10^{-8}$ & $305$ & $1.000000$ & $1.000000$ & $7.408502\cdot 10^{-08}$\\
            $10^{-9}$ & $318$ & $1.000000$ & $1.000000$ & $3.880696\cdot 10^{-08}$\\
            \hline
          \end{tabular}
        \end{table}

        \clearpage
    \subsubsection{Модифицированный метод с золотым сечением}

        \begin{table}[ht]
          \centering
          \caption{Функция Розенброка, $\Delta = 1$}
          \begin{tabular}{|c|c|c|c|c|}
            \hline
            $\varepsilon$ & К.В.Ф. & $x_1$ & $x_2$ & $f$\\
            \hline
            $10^{-0}$ & $31$ & $-0.032457$ & $0.004171$ & $1.066939\cdot 10^{+00}$\\
            $10^{-1}$ & $51$ & $0.216162$ & $0.041958$ & $6.166750\cdot 10^{-01}$\\
            $10^{-2}$ & $513$ & $0.998846$ & $0.995810$ & $3.563947\cdot 10^{-04}$\\
            $10^{-3}$ & $750$ & $1.000037$ & $1.000193$ & $1.438000\cdot 10^{-06}$\\
            $10^{-4}$ & $887$ & $1.000004$ & $1.000018$ & $8.392043\cdot 10^{-09}$\\
            $10^{-5}$ & $1119$ & $0.999999$ & $0.999999$ & $2.223234\cdot 10^{-10}$\\
            $10^{-6}$ & $1371$ & $1.000000$ & $0.999999$ & $7.407322\cdot 10^{-12}$\\
            $10^{-7}$ & $1478$ & $1.000000$ & $1.000000$ & $2.228001\cdot 10^{-18}$\\
            $10^{-8}$ & $1658$ & $1.000000$ & $1.000000$ & $1.699808\cdot 10^{-16}$\\
            $10^{-9}$ & $1903$ & $1.000000$ & $1.000000$ & $1.853118\cdot 10^{-18}$\\
            \hline
          \end{tabular}
        \end{table}

        \begin{table}[ht]
          \centering
          \caption{Функция Розенброка, $\Delta = 0.5$}
          \begin{tabular}{|c|c|c|c|c|}
            \hline
            $\varepsilon$ & К.В.Ф. & $x_1$ & $x_2$ & $f$\\
            \hline
            $10^{-0}$ & $29$ & $0.028525$ & $0.001037$ & $9.437690\cdot 10^{-01}$\\
            $10^{-1}$ & $49$ & $0.231343$ & $0.063969$ & $6.017535\cdot 10^{-01}$\\
            $10^{-2}$ & $456$ & $0.987124$ & $0.972493$ & $5.346531\cdot 10^{-04}$\\
            $10^{-3}$ & $685$ & $0.999368$ & $0.999032$ & $9.139243\cdot 10^{-06}$\\
            $10^{-4}$ & $855$ & $0.999996$ & $0.999986$ & $3.466505\cdot 10^{-09}$\\
            $10^{-5}$ & $1085$ & $1.000003$ & $1.000008$ & $1.562749\cdot 10^{-10}$\\
            $10^{-6}$ & $1265$ & $1.000000$ & $0.999999$ & $9.127516\cdot 10^{-13}$\\
            $10^{-7}$ & $1409$ & $1.000000$ & $1.000000$ & $2.761110\cdot 10^{-17}$\\
            $10^{-8}$ & $1589$ & $1.000000$ & $1.000000$ & $3.308784\cdot 10^{-16}$\\
            $10^{-9}$ & $1867$ & $1.000000$ & $1.000000$ & $7.202872\cdot 10^{-20}$\\
            \hline
          \end{tabular}
        \end{table}

        \begin{table}[ht]
          \centering
          \caption{Функция Розенброка, $\Delta = 0.1$}
          \begin{tabular}{|c|c|c|c|c|}
            \hline
            $\varepsilon$ & К.В.Ф. & $x_1$ & $x_2$ & $f$\\
            \hline
            $10^{-0}$ & $30$ & $0.149579$ & $0.035197$ & $7.396586\cdot 10^{-01}$\\
            $10^{-1}$ & $39$ & $0.242208$ & $0.054916$ & $5.756540\cdot 10^{-01}$\\
            $10^{-2}$ & $389$ & $0.995772$ & $0.988219$ & $1.134876\cdot 10^{-03}$\\
            $10^{-3}$ & $607$ & $1.000103$ & $1.000232$ & $8.270946\cdot 10^{-08}$\\
            $10^{-4}$ & $766$ & $0.999989$ & $0.999977$ & $3.762098\cdot 10^{-10}$\\
            $10^{-5}$ & $1020$ & $1.000006$ & $1.000013$ & $1.014339\cdot 10^{-10}$\\
            $10^{-6}$ & $1145$ & $1.000000$ & $1.000000$ & $3.148726\cdot 10^{-12}$\\
            $10^{-7}$ & $1325$ & $1.000000$ & $1.000000$ & $7.931547\cdot 10^{-14}$\\
            $10^{-8}$ & $1494$ & $1.000000$ & $1.000000$ & $4.744233\cdot 10^{-17}$\\
            $10^{-9}$ & $1766$ & $1.000000$ & $1.000000$ & $1.693201\cdot 10^{-18}$\\
            \hline
          \end{tabular}
        \end{table}

        \begin{table}[ht]
          \centering
          \caption{Степенная функция, $\Delta = 1$}
          \begin{tabular}{|c|c|c|c|c|}
            \hline
            $\varepsilon$ & К.В.Ф. & $x_1$ & $x_2$ & $f$\\
            \hline
            $10^{-0}$ & $31$ & $0.163858$ & $0.307892$ & $2.220709\cdot 10^{-01}$\\
            $10^{-1}$ & $127$ & $1.011145$ & $0.992082$ & $1.866917\cdot 10^{-10}$\\
            $10^{-2}$ & $175$ & $1.000015$ & $1.000252$ & $1.528132\cdot 10^{-25}$\\
            $10^{-3}$ & $267$ & $1.000060$ & $1.000131$ & $2.011511\cdot 10^{-29}$\\
            $10^{-4}$ & $317$ & $0.999996$ & $1.000016$ & $3.481534\cdot 10^{-34}$\\
            $10^{-5}$ & $377$ & $1.000000$ & $0.999999$ & $2.904809\cdot 10^{-48}$\\
            $10^{-6}$ & $437$ & $1.000000$ & $1.000000$ & $3.997378\cdot 10^{-55}$\\
            $10^{-7}$ & $495$ & $1.000000$ & $1.000000$ & $1.732721\cdot 10^{-64}$\\
            $10^{-8}$ & $647$ & $1.000000$ & $1.000000$ & $1.877895\cdot 10^{-70}$\\
            $10^{-9}$ & $705$ & $1.000000$ & $1.000000$ & $1.647839\cdot 10^{-73}$\\
            \hline
          \end{tabular}
        \end{table}

        \begin{table}[ht]
          \centering
          \caption{Степенная функция, $\Delta = 0.5$}
          \begin{tabular}{|c|c|c|c|c|}
            \hline
            $\varepsilon$ & К.В.Ф. & $x_1$ & $x_2$ & $f$\\
            \hline
            $10^{-0}$ & $29$ & $0.258221$ & $0.183452$ & $2.727240\cdot 10^{-01}$\\
            $10^{-1}$ & $117$ & $1.001652$ & $1.001298$ & $7.458737\cdot 10^{-23}$\\
            $10^{-2}$ & $161$ & $0.999991$ & $1.001238$ & $8.546977\cdot 10^{-20}$\\
            $10^{-3}$ & $249$ & $0.999745$ & $0.999566$ & $6.670732\cdot 10^{-26}$\\
            $10^{-4}$ & $309$ & $0.999996$ & $0.999999$ & $1.501424\cdot 10^{-40}$\\
            $10^{-5}$ & $369$ & $0.999999$ & $1.000000$ & $2.844431\cdot 10^{-46}$\\
            $10^{-6}$ & $429$ & $1.000000$ & $1.000000$ & $4.167821\cdot 10^{-54}$\\
            $10^{-7}$ & $477$ & $1.000000$ & $1.000000$ & $1.065918\cdot 10^{-60}$\\
            $10^{-8}$ & $625$ & $1.000000$ & $1.000000$ & $1.638247\cdot 10^{-67}$\\
            $10^{-9}$ & $695$ & $1.000000$ & $1.000000$ & $7.959911\cdot 10^{-76}$\\
            \hline
          \end{tabular}
        \end{table}

        \begin{table}[ht]
          \centering
          \caption{Степенная функция, $\Delta = 0.1$}
          \begin{tabular}{|c|c|c|c|c|}
            \hline
            $\varepsilon$ & К.В.Ф. & $x_1$ & $x_2$ & $f$\\
            \hline
            $10^{-0}$ & $33$ & $0.176671$ & $0.262290$ & $1.454153\cdot 10^{-01}$\\
            $10^{-1}$ & $82$ & $0.999709$ & $0.987974$ & $5.361199\cdot 10^{-12}$\\
            $10^{-2}$ & $146$ & $0.999019$ & $0.999210$ & $9.508046\cdot 10^{-25}$\\
            $10^{-3}$ & $230$ & $1.000062$ & $1.000047$ & $4.094276\cdot 10^{-34}$\\
            $10^{-4}$ & $286$ & $0.999996$ & $0.999989$ & $1.006155\cdot 10^{-37}$\\
            $10^{-5}$ & $339$ & $1.000001$ & $1.000000$ & $7.595626\cdot 10^{-48}$\\
            $10^{-6}$ & $398$ & $1.000000$ & $1.000000$ & $2.996838\cdot 10^{-55}$\\
            $10^{-7}$ & $458$ & $1.000000$ & $1.000000$ & $2.923031\cdot 10^{-67}$\\
            $10^{-8}$ & $596$ & $1.000000$ & $1.000000$ & $3.560688\cdot 10^{-69}$\\
            $10^{-9}$ & $666$ & $1.000000$ & $1.000000$ & $4.585630\cdot 10^{-78}$\\
            \hline
          \end{tabular}
        \end{table}

        \begin{table}[ht]
          \centering
          \caption{Корневая функция, $\Delta = 1$}
          \begin{tabular}{|c|c|c|c|c|}
            \hline
            $\varepsilon$ & К.В.Ф. & $x_1$ & $x_2$ & $f$\\
            \hline
            $10^{-0}$ & $31$ & $-0.032457$ & $0.004171$ & $1.019281\cdot 10^{+00}$\\
            $10^{-1}$ & $101$ & $0.989870$ & $0.997728$ & $1.638124\cdot 10^{-01}$\\
            $10^{-2}$ & $175$ & $1.000349$ & $1.000008$ & $3.368103\cdot 10^{-02}$\\
            $10^{-3}$ & $267$ & $1.000033$ & $1.000014$ & $8.412298\cdot 10^{-03}$\\
            $10^{-4}$ & $317$ & $0.999992$ & $1.000013$ & $8.160954\cdot 10^{-03}$\\
            $10^{-5}$ & $377$ & $1.000001$ & $1.000000$ & $2.153282\cdot 10^{-03}$\\
            $10^{-6}$ & $437$ & $1.000000$ & $1.000000$ & $7.653897\cdot 10^{-04}$\\
            $10^{-7}$ & $495$ & $1.000000$ & $1.000000$ & $1.198839\cdot 10^{-04}$\\
            $10^{-8}$ & $647$ & $1.000000$ & $1.000000$ & $4.340209\cdot 10^{-05}$\\
            $10^{-9}$ & $705$ & $1.000000$ & $1.000000$ & $1.710467\cdot 10^{-05}$\\
            \hline
          \end{tabular}
        \end{table}

        \begin{table}[ht]
          \centering
          \caption{Корневая функция, $\Delta = 0.5$}
          \begin{tabular}{|c|c|c|c|c|}
            \hline
            $\varepsilon$ & К.В.Ф. & $x_1$ & $x_2$ & $f$\\
            \hline
            $10^{-0}$ & $29$ & $0.028525$ & $0.001037$ & $9.876013\cdot 10^{-01}$\\
            $10^{-1}$ & $115$ & $0.991967$ & $0.985664$ & $1.465898\cdot 10^{-01}$\\
            $10^{-2}$ & $161$ & $0.999179$ & $0.999683$ & $4.235802\cdot 10^{-02}$\\
            $10^{-3}$ & $249$ & $0.999971$ & $0.999747$ & $2.664950\cdot 10^{-02}$\\
            $10^{-4}$ & $309$ & $0.999995$ & $1.000007$ & $5.952199\cdot 10^{-03}$\\
            $10^{-5}$ & $369$ & $0.999999$ & $1.000000$ & $1.782014\cdot 10^{-03}$\\
            $10^{-6}$ & $429$ & $1.000000$ & $1.000000$ & $5.537944\cdot 10^{-04}$\\
            $10^{-7}$ & $477$ & $1.000000$ & $1.000000$ & $2.110390\cdot 10^{-04}$\\
            $10^{-8}$ & $625$ & $1.000000$ & $1.000000$ & $8.101790\cdot 10^{-05}$\\
            $10^{-9}$ & $695$ & $1.000000$ & $1.000000$ & $1.854895\cdot 10^{-05}$\\
            \hline
          \end{tabular}
        \end{table}

        \begin{table}[ht]
          \centering
          \caption{Корневая функция, $\Delta = 0.1$}
          \begin{tabular}{|c|c|c|c|c|}
            \hline
            $\varepsilon$ & К.В.Ф. & $x_1$ & $x_2$ & $f$\\
            \hline
            $10^{-0}$ & $33$ & $0.227037$ & $0.211264$ & $8.800971\cdot 10^{-01}$\\
            $10^{-1}$ & $94$ & $0.996722$ & $0.986966$ & $1.761448\cdot 10^{-01}$\\
            $10^{-2}$ & $144$ & $1.001135$ & $0.998952$ & $8.364837\cdot 10^{-02}$\\
            $10^{-3}$ & $227$ & $1.000006$ & $0.999918$ & $1.663027\cdot 10^{-02}$\\
            $10^{-4}$ & $283$ & $0.999987$ & $0.999997$ & $6.001345\cdot 10^{-03}$\\
            $10^{-5}$ & $337$ & $0.999999$ & $1.000001$ & $2.497451\cdot 10^{-03}$\\
            $10^{-6}$ & $396$ & $1.000000$ & $1.000000$ & $4.703622\cdot 10^{-04}$\\
            $10^{-7}$ & $456$ & $1.000000$ & $1.000000$ & $2.115841\cdot 10^{-04}$\\
            $10^{-8}$ & $593$ & $1.000000$ & $1.000000$ & $2.397934\cdot 10^{-05}$\\
            $10^{-9}$ & $663$ & $1.000000$ & $1.000000$ & $1.692080\cdot 10^{-05}$\\
            \hline
          \end{tabular}
        \end{table}

        \clearpage
  \subsection{Условная оптимизация}
    \subsubsection{Выпуклая область}
      \[
        \left(x_1-1\right)^2 + x_2 \leqslant 4
      \]

      \paragraph{Функция Розенброка}
        Метод останавливается в точке $x = \left(\begin{matrix} 0.9926\\
          0.00373\end{matrix}\right)$ для любой точности $\varepsilon
          \leqslant 10^{-3}$.
      \paragraph{Степенная функция}
        Для этой функции метод тоже останавливается в точке
        $\left(\begin{matrix} 1\\ 0\end{matrix}\right)$, которая не является
          минимумом в допустимой области.
      \paragraph{Корневая функция}
        В силу того, что функция принимает небольшие значения в этой области,
        результат был получен более точный: $x = \left(\begin{matrix}
          0.82924\\
          0.78436\end{matrix}\right)$.
    \subsubsection{Линейное ограничение}
      \[
        x_2 \leqslant -1 - x_1
      \]

      \paragraph{Функция Розенброка}
      Чтобы достигнуть точности $\varepsilon = 10^{-6}$ необходимо $170$
      вычислений функции.
      В результате получается точка $x = \left(\begin{matrix} -0.49006\\
        -0.50994\end{matrix}\right)$, $f(x) = 5.8501 \cdot 10^{1}$.
      \paragraph{Степенная функция}
      Чтобы достигнуть точности $\varepsilon = 10^{-6}$ необходимо $186$
      вычислений функции.
      В результате получается точка $x = \left(\begin{matrix} -0.46341\\
        -0.53657\end{matrix}\right)$, $f(x) = 2.3237\cdot 10^{1}$.
      \paragraph{Корневая функция}
      Чтобы достигнуть точности $\varepsilon = 10^{-6}$ необходимо $203$
      вычислений функции.
      В результате получается точка $x = \left(\begin{matrix} -0.46342\\
        -0.53657\end{matrix}\right)$, $f(x) = 1.2172$.

  \clearpage
\section{Выводы}
  В данной работе проведены исследования тестовых функций методом Розенброка
  для задач условной и безусловной оптимизаций.
  Все функции тестировались для точности в пределах $1$ до $10^{-9}$, а
  параметры метода ($\lambda_0$ для стандартного метода Розенброка и $\Delta$
  для его модифицированной версии) были испробованы значения от $1$ до
  $10^{-2}$ (в отчёте предоставлены только три варианта: $1$, $0.5$ и $0.1$).
  В случаях безусловных оптимизаций точность и параметр были выбраны
  оптимальными исходя из предшествующего тестирования безусловной оптимизации
  для этих функций.

  Как показали тесты, использование модифицированного метода Розенброка с
  методом золотого сечения в качестве метода одномерного поиска даёт самые
  плохие результаты для все тестовых функций.

  \paragraph{Функция Розенброка}
    Точность $10^{-9}$ при использовании стандартного метода с $\lambda_0 =
    \left(\begin{matrix} 0.1\\ 0.1\end{matrix} \right)$ была достигнута всего за
    $353$ вычисления функции, полученное значение оптимума $2.361476\cdot
    10^{-18}$.
    Оптимальной начальной $\lambda$ является $\left(\begin{matrix} 0.5 \\ 0.5
    \end{matrix}\right)$, т.\,к. при использовании её метод приходит в точку
    минимума за 242 вычисления функции.
    Нетрудно заметить скачок в количестве вычислений функций при изменении
    точности с $10^{-3}$ на $10^{-4}$.
    На мой взгляд, это связано с тем, что овражность функции Розенброка имеет
    приблизительно такую ширину.

    При использовании модифицированной версии алгоритма результаты получились
    хуже, однако использовании $\Delta = 0.5$ и точности $10^{-9}$ была
    достигнута точка оптимума со значением функции в ней $0$.

  \paragraph{Степенная функция}
    Для степенной функции метод сходится в точку оптимума достаточно быстро,
    без существенных рывков в количестве вычислений функции при увеличении
    точности.
    Опять же, модифицированный метод показывает немного хуже результаты, чем
    немодифированный.
    Так же, как и в случае с функцией Розенброка, оптимальными параметрами для
    обоих видов метода были, соотвественно, $\lambda_0 = \left(\begin{matrix}
      0.5\\ 0.5\end{matrix}\right)$ и $\Delta = 0.5$.
  \paragraph{Корневая функция}
    Для этой тестовой функции лучше показал себя модифицированный метод, где
    был достигнут оптимум за $345$ вычислений функции.
    Для немодифицированного метода заметно то, что уменьшение начального шага
    $\lambda_0$ отрицательно сказывается на сходимости метода: легко заметить,
    что значения полученных оптимумов для $\lambda_0 = \left(\begin{matrix} 1\\
      1\end{matrix}\right)$ меньше соответствующих минимумов, полученных
      используя $\lambda_0 = \left(\begin{matrix} 0.5\\
        0.5\end{matrix}\right)$.

  \paragraph{Условная оптимизация}
    Идея Розенброка плохо применяется на практике:
    метод не справляется ни с вогнутыми ограничениями, ни с нелинейными
    выпуклыми ограничениями.
    При таких ограничениях метод Розенброка справился только с корневой
    функцией.
    Это обусловлено тем, что эта функция принимает значения, не сильно
    различающиеся по величине.

    При линейном же ограничении метод справился удовлетворительно для всех
    тестовых функций.

  Таким образом, исследование показало, что метод Розенброка очень хорошо
  справляется с безусловной оптимизацией овражных функций, причём модификация
  Дэвиса, Свена, Кемпи зачастую проигрывает немодифицированной версии метода
  по достигнутой точности за определённое количество вычислений функций.

  Однако, метод Розенброка плохо подходит для решения задачи условной
  оптимизации.
  Можно предложить использовать следующую методику для решения такой задачи:
  \begin{enumerate}
    \item Запустить метод Розенброка для безусловной оптимизации;
    \item из последней допустимой рабочей точки, полученной в ходе этого этого
      метода запустить другой, более подходящий для задачи условной
      оптимизации, метод.
      Например, метод Вейсмана.
  \end{enumerate}
  \clearpage
\section{Список литературы}
\begin{enumerate}
  \item Д.\,Химмельблау ,,Прикладное нелинейное программирование''
  \item Конспект лекций по предмету ,,Методы оптимизаций''
\end{enumerate}
\end{document}
