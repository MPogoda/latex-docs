\documentclass[a4paper,12pt,notitlepage,pdftex,headsepline]{scrartcl}

\usepackage{a4wide}
\usepackage{cmap} % чтобы работал поиск по PDF
\usepackage[utf8]{inputenc}
\usepackage[russian]{babel}
\usepackage[T2A]{fontenc}

\usepackage{textcase}
\usepackage[pdftex]{graphicx}

\usepackage{lscape}

\pdfcompresslevel=9 % сжимать PDF
\usepackage{pdflscape} % для возможности альбомного размещения некоторых страниц
\usepackage[pdftex]{hyperref}
% настройка ссылок в оглавлении для pdf формата
\hypersetup{unicode=true,
            pdftitle={Операционные системы},
            pdfauthor={Погода Михаил},
            pdfcreator={pdflatex},
            pdfsubject={},
            pdfborder    = {0 0 0},
            bookmarksopen,
            bookmarksnumbered,
            bookmarksopenlevel = 2,
            pdfkeywords={},
            colorlinks=true, % установка цвета ссылок в оглавлении
            citecolor=black,
            filecolor=black,
            linkcolor=black,
            urlcolor=blue}

\usepackage{amsmath}
\usepackage{amssymb}
\usepackage{moreverb}
%for \includepdf
%\usepackage{pdfpages}

\author{Михаил Погода}
\title{Конспект по операционным системам}
\date{\today}

\begin{document}
\section{Список литературы}
  \begin{itemize}
    \item Таненбаум Э. ,,Современные операционные системы''
    \item Дейтел Г. ,,Введение в операционные системы''
    \item ,,Основы операционных системы. Конспект лекций'' (intuit)
    \item Макаров ,,Операционные системы. Конспект лекций''
  \end{itemize}
\section{26.01.12}
  \subsection{Структура компьютерной системы (hardware + software)}
  \subsection{Функциональное устройство компьютера}
  \subsection{Структура системного програмного обеспечения}
  \subsection{Место и роль операционной системы в составе компьютерной системы}
  \subsection{Две основные функции операционной системы. Временное и пространственное распределение ресурсов}
  \subsection{Режимы функционирования программного обеспечения (режим ядра, пользовательский), их использование, привелигированные команды}
  \subsection{Эволюция операционных систем}
  \subsection{Пирамида операционных систем}
  \subsection{Прерывания, их типы. Механизм прерываний}
  \subsection{Архитектура операционных систем}
  \subsection{Зашита памяти в ЭВМ}
  \subsection{Таймер и часы}
  \subsection{Интерливинг, буферизация, периферийные устройства}
  \subsection{Работа устройств в режиме он-лайн и офф-лайн}
  \subsection{Виртуальная память}
  \subsection{Мультипроцессораная обработка и конвейеризация}
  \subsection{Программное обеспечение, машинные языки, языки высокого уровня (их классификация)}
  \subsection{Ассемблеры и макросы}
  \subsection{Компиляторы и интерпретаторы}
  \subsection{Загрузчики и редакторы связей}
  \subsection{Микропрограммирование и ОС}
\end{document}
