\documentclass[a4paper,12pt,notitlepage,pdftex,headsepline]{scrartcl}

\usepackage{a4wide}
\usepackage{cmap} % чтобы работал поиск по PDF
\usepackage[utf8]{inputenc}
\usepackage[russian]{babel}
\usepackage[T2A]{fontenc}

\usepackage{textcase}
\usepackage[pdftex]{graphicx}

\usepackage{lscape}

\pdfcompresslevel=9 % сжимать PDF
\usepackage{pdflscape} % для возможности альбомного размещения некоторых страниц
\usepackage[pdftex]{hyperref}
% настройка ссылок в оглавлении для pdf формата
\hypersetup{unicode=true,
            pdftitle={Операционные системы},
            pdfauthor={Погода Михаил},
            pdfcreator={pdflatex},
            pdfsubject={},
            pdfborder    = {0 0 0},
            bookmarksopen,
            bookmarksnumbered,
            bookmarksopenlevel = 2,
            pdfkeywords={},
            colorlinks=true, % установка цвета ссылок в оглавлении
            citecolor=black,
            filecolor=black,
            linkcolor=black,
            urlcolor=blue}

\usepackage{amsmath}
\usepackage{amssymb}
\usepackage{moreverb}
%for \includepdf
%\usepackage{pdfpages}

\author{Михаил Погода}
\title{Конспект по операционным системам}
\date{\today}

\begin{document}
\section{Список литературы}
  \begin{itemize}
    \item Таненбаум Э. ,,Современные операционные системы''
    \item Дейтел Г. ,,Введение в операционные системы''
    \item ,,Основы операционных системы. Конспект лекций'' (intuit)
    \item Макаров ,,Операционные системы. Конспект лекций''
  \end{itemize}
\section{26.01.12}
  \subsection{Структура компьютерной системы (hardware + software)}
  \subsection{Функциональное устройство компьютера}
  \subsection{Структура системного програмного обеспечения}
  \subsection{Место и роль операционной системы в составе компьютерной системы}
  \subsection{Две основные функции операционной системы. Временное и пространственное распределение ресурсов}
  \subsection{Режимы функционирования программного обеспечения (режим ядра, пользовательский), их использование, привелигированные команды}
  \subsection{Эволюция операционных систем}
  \subsection{Пирамида операционных систем}
  \subsection{Прерывания, их типы. Механизм прерываний}
  \subsection{Архитектура операционных систем}
  \subsection{Зашита памяти в ЭВМ}
  \subsection{Таймер и часы}
  \subsection{Интерливинг, буферизация, периферийные устройства}
  \subsection{Работа устройств в режиме он-лайн и офф-лайн}
  \subsection{Виртуальная память}
  \subsection{Мультипроцессораная обработка и конвейеризация}
  \subsection{Программное обеспечение, машинные языки, языки высокого уровня (их классификация)}
  \subsection{Ассемблеры и макросы}
  \subsection{Компиляторы и интерпретаторы}
  \subsection{Загрузчики и редакторы связей}
  \subsection{Микропрограммирование и ОС}

  \section{15.03.12. Взаимоблокировки (тупики, дедлоки, линчи)}
  \subsection{Примеры возникновения тупиковых ситуаций в компьютерных системах}
  \subsection{Концепция ресурсов: типы ресурсов(выгружаемые и не выгружаемые, разделяемые и выделенные)}
  \subsection{Последовательность событий для получения ресурсов (3 пункта)}
  \subsection{Определение дедлока}
  \subsection{Условия возникновения тупиков (4)}
  \subsection{Моделирование дедлоков}
  \subsection{Стратегии поведения при возникновении дедлоков: пренебрежение проблемы в целом ( windows-стайл), обнаружение и восстановление после дедлока, динамическое избежание тупиковых ситуаций путём грамотного распределения ресурсов: траектории ресурсов, безопасные (надёжные) и ненадёжные состояния, алгоритм банкира для одного вида ресурсов, для нескольких видов ресурсов; предотвращение с помощью опровержения одного из 4 условий, необходимых для дедлока}
  \subsection{Двухвазное блокирование}
  \subsection{Тупики без ресурсов}
  \subsection{Голодание}


  ТАНЕНБАУМ с.185--212
  интуит.ру л7 с1--8
  Дейтел т1 с.158--195




%  { COMPUTER SYSTEMS} 
  \section{13.03.12. Устройство управления ЭВМ}
    \subsection{Декомпозиция вычислительного устройства на операционные и управляющие блоки. Принцип В.\,М. Глушкова. Иерархия языков, описания функционирования вычислительных устройств}
    Каган с.115--125, Вашкевич с. 110--113
    \subsection{Назначение УУ}
    Стригин с. 128--131
    Волкогон 
    \subsection{Схема УУ одноадресной ЭВМ и его функционирование}
    Стригин с. 131--134
    Волкогон
    \subsection{Функции УУ}
    Стригин с. 133--134
    Волкогон
    \subsection{Классификаций УУ: по способу выполнения команд в ЭВМ (синхронные, асинхронные, смешанные), по своим возможностям (универсальные, с жётской логикой), по принципу формирования управляющих сигналов (с жёсткой логикой, микропрограммные --- с хранимой логикой)}
    Стригин с. 134--136
    Волкогон
    \subsection{Структура устройств управления: универсальные УУ, микропрограммные УУ}
    Стригин с. 136--139
    \subsection{Управляющие автоматы с жёсткой логикой. Синтез УУ с жёсткой логикой и его функционирование}
    Волкогон
    Вашкевич с.113--118
    \subsection{Управляющие автоматы с программируемой логикой (МПА): с естественной адресацией, с принудительной адресацией}
    Волкогон
    \subsection{Способы кодирования управляющих сигналов при микропрограммировании}
    Волкогон
\end{document}
