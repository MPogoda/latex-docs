\documentclass[paper=a4,10pt,pagesize]{scrartcl}
\usepackage{ucs}
\usepackage[utf8x]{inputenc}
\usepackage[russian]{babel}
\begin{document}
\begin{enumerate}
  \item \textit{Что такое магнитное поле? Что называется напряжённостью и индукцией магнитного поля и какая связь между ними?}

Магнитное поле~--- особая форма материи, с помощью которой осуществляется взаимодействие между двигающимися электрически-заряженными частицами.

Напряжённость магнитного поля $\vec{H}$~--- векторная физическая величина, которая является количественной характеристикой магнитного поля, выражающая силу, с которой поле действует на единицу длины прямолинейного проводника с силой тока в одну единицу, расположенного перпендикулярно к направлению магнитных силовых линий.

Магнитная индукция $\vec{B}$~--- векторная величина, которая является силовой характеристикой магнитного поля в данной точке пространства. Показывает с какой силой магнитное поле действует на двигающийся заряд.

Магнитная индукция связана с напряжённостью магнитного поля следующей формулой:
$$\vec{B}=\mu\vec{H}$$
\item \textit{\small Что происходит при намагничивании магнетика? Какой физический смысл вектора намагничивания?}

Внутри магнетиков при включении внешнего магнитного поля возникает явление электромагнитной инерции.
ЭДС индукции изменяет движение электронов, в атомах и молекулах появляются дополнительные магнитные моменты.
По правилу Ленца эти моменты направлены так, что вещество намагничиваются против внешнего поля так, что возникает диамагнетизм.

Вектор намагничивания $\vec{j}$ является количественной характеристикой намагничивания магнетика; он представляет собой магнитный момент единицы объёма, который возникает во внешнем магнитном поле $\vec{H}$:
$$\vec{j}=\chi\vec{H}$$
\item \textit{Какие существуют виды магнетиков? Какие свойства диа- и парамагнетиков?}

Существуют такие виды магнетиков:
\begin{itemize}
    \item $\chi_\mu<0$~--- диамагнетики;
    \item $\chi_\mu>0$~--- парамагнетики;
    \item $\chi_\mu\gg1$~--- ферромагнетики.
\end{itemize}
В диамагнетиках вектор намагничивания направлен навстречу вектору напряжённости внешнего поля, поэтому у них $\mu<1$.

У парамагнетиков средний магнитный момент вещества равен нулю.
У них общая намагниченность совпадает с направлением внешнего поля и $\mu >1$.

У диа- и парамагнетиков намагниченность небольшая, появляется только при воздействии внешнего поля и исчезает после прекращения воздействия этого поля.
\item \textit{Что такое ферромагнетики? В чём состоит явление магнитного гистерезиса?}

Ферромагнетики~--- сильно-магнитные вещества, в которых магнитная проницаемость $\mu\gg1$.

Явление ~магнитного ~гистерезиса ~состоит в необратимости магнитных свойств ферромагнетика под влиянием тех магнитных процессов, которые раньше влияли на него.
\item \textit{Какая природа ферромагнетизма? Что такое точка Кюри для ферромагнетика? }

Появление магнитных свойств у ферромагнетиков связано с их доменной структурой.
При воздействии на ферромагнетик внешним магнитным полем, векторы намагниченности доменов начинают поворачиваться за силовыми линиями магнитного поля.
Чем сильнее магнитное поле, тем меньше углы между векторами магнитных моментов доменов и силовыми линиями, что вызывает нелинейное возрастание намагниченности.

Точка Кюри~--- температура $T_\kappa$, при превышении которой ферромагнетик становится обычным парамагнетиком.
При этой температуре области спонтанной намагниченности распадаются, и ферромагнетик теряет свои характерные свойства.
\item \textit{В чём заключается явление электромагнитной индукции? Как формулируется закон Фарадея для электромагнитной индукции? }

Явление электромагнитной индукции заключается в появлении в замкнутом контуре проводника электрического тока из-за изменения потока магнитной индукции через площадь, ограниченную этим контуром.

Закон ~электромагнитной ~индукции ~формулируется так: \texttt{ЭДС индукции пропорциональна скорости изменения магнитного потока:}
$$\varepsilon = -\frac{d\Phi}{dt}$$
\item \textit{Как формулируется закон Ома для переменного тока? Запишите его формулу.}

Сила тока в цепи переменного тока прямо пропорциональна напряжению и обратно пропорциональна полному сопротивлению поля переменного тока:
$$\mathcal{I} = \frac{U}{\sqrt{R^2+\left(\frac{1}{\omega C} - \omega L\right)^2}}$$
\item \textit{В чём состоит предварительная настройка та регулировка электронного осциллографа?}

Необходимо включить осциллограф и блок питания в сеть на 5--10 минут.
Ручку потенциометра выставить в крайнюю левую позицию.
Ручками управления лучом установить в центр экрана светлое пятно.
Ручки ,,усиление X'' и ,,усиление Y'' установить в крайние левые позиции.
\item \textit{Что такое чувствительность осциллографа к напряжению?}

Чувствительность осциллографа по напряжению~--- величина, которая определяется экспериментально для горизонтального и вертикального каналов осциллографа через число делений сетки экрана, на которые отклонился электронный луч вдоль осей OX и OY от центра:

$$C_x=\frac{x}{U_x}, ~~~~~~~~~~~~~~~~~~~~C_y=\frac{y}{U_y}$$
\end{enumerate}

\end{document}
