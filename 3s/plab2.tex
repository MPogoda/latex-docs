\documentclass[a4paper,10pt]{article}
\usepackage[utf8]{inputenc}
\usepackage[russian]{babel}
\usepackage{indentfirst}
\usepackage{amsmath}
\usepackage{amssymb}

\begin{document}
\begin{enumerate}
\item \textit{Что такое электрическая ёмкость и в каких единицах она измеряется?}

Электрическая ёмкость --- способность тела накапливать электрический заряд.
Единица измерения ёмкости в системе СИ --- Фарад.
\item \textit{В чём заключается суть предложенного метода измерения ёмкости конденсатора?}

Смысл предложенного метода измерения ёмкости конденсатора состоит в том, что мы знаем ёмкость другог конденсатора и соотношение зарядов
этих конденсаторов (может быть заменено определением максимальным угловым отклонений рамки гальванометра,
которое вызывается протеканием тока через рамку) мы можем вычислить ёмкость другого конденсатора:
$$C_2 = C_1\frac{q_2}{q_1}, \text{ або  } C_2 = C_1\frac{\varphi_0^{\left(2\right)}}{\varphi_0^{\left(1\right)}}$$
, где $\frac{\varphi_0^{\left(2\right)}}{\varphi_0^{\left(1\right)}}$ --- отношение максимальных угловых отклонений рамки гальванометра.
\item \textit{Выведите формулу, которая подтверждает, что величина первого отклонения баллистического гальванометра пропорциональна
заряда, который прошёл по его рамке}

Уравнение движения рамки ($\tau < T_0$, $T_0$ --- период собственных колебаний рамки):
\begin{equation}\label{a1}
\mathcal{I}\ddot\varphi = BnSi \Rightarrow BnS\int\limits_0^\tau i\, dt = BnSq = \mathcal{I}\dot\varphi
\end{equation}
Кинетическая энергия, которую получит рамка: $\frac{\mathcal{I}\dot\varphi^2}{2}$, она пойдёт на закручивание подвеса на угол $\varphi$.

Момент сил кручения при повороте на угол $\varphi$ --- $D\cdot\varphi$, т.\,е. при закручивании на угол $d\varphi$ работа сил
кручения будет $D\cdot\varphi d\varphi$.
Вся работа, которая тратится на отклонение рамки на угол $\varphi_0$:
$$\int\limits_0^{\varphi_0} D\varphi\,d\varphi = \frac{D\varphi_0^2}{2}$$
\begin{equation}\label{a2}
\frac{D\varphi_0^2}{2} = \frac{\mathcal{I}\dot\varphi^2}{2}, \text{ или } \mathcal{I}\dot\varphi^2 = D\varphi_0^2
\end{equation}

Из (\ref{a1}) и (\ref{a2}) имеем:
\begin{equation}\label{a3}
\mathcal{I} = \frac{B^2n^2s^2q^2}{D\varphi_0^2}
\end{equation}
Период собственных колебаний рамки: $T_0 = 2\pi\sqrt\frac{\mathcal{I}}{D} \Rightarrow \mathcal{I} = \frac{T_0^2D}{4\pi^2}$.
Подставив это в (\ref{a3}), получим:
$$\frac{T_0^2D}{4\pi^2} = \frac{B^2n^2S^2q^2}{D\varphi_0^2} \Leftrightarrow T_0^2D^2\varphi_0^2 = 4\pi^2B^2n^2S^2q^2$$
$$q = \frac{T_0D\varphi_0}{2\pi B n S} = \frac{T_0}{\pi} \frac{D}{B n S} \frac{\varphi_0}{2} = \frac{T_0}{2\pi} c \varphi_0 = b \varphi_0$$
, где $b = \frac{T_0}{2\pi}c$
\item \textit{Как устроен баллистический гальванометр и какое его предназначение?}

Баллистический гальванометр используется для измерения количества электричества, которое проходит через цепь за промежуток времени,
маленький по сравнению с периодом собственных колебаний рамки.

Баллистический гальванометр относится к приборам магнитоэлектрической системы и от обычного отличается искуственно увеличенным моментом
инерции его подвижной части.

Существуют стрелочные и зеркальные гальванометры.
В зеркальном (подвижная) рамка подвешивается в магнитном поле постоянного магнита на тонкой упругой подвеске.
В рамке есть замкнутый сердечник (циллиндрический).
Ток подводится к рамке через подвес и тонкую металлическую нитку, которая оттягивает книзу подвижную систему.
К подвесу возле рамки прикрепляется лёгкое зеркало.
Луч света от светильника падает на зеркало и, отражаясь, попадает на шкалу с отметками.

\item \textit{Что такое динамическая и баллистическая постоянные гальванометра?}

Динамическая постоянная гальванометра --- это величина $c = \frac{D}{B n S}$.
Баллистическая постоянная гальванометра --- это величина $b = c\frac{T_0}{2\pi}$.

\item \textit{Какое предназначение кнопочного выключателя $K_2$ в схеме?}

Кнопочный переключатель $K_2$ замыкает цепь индукционных токов, которые возникают в рамке гальванометра во время её колебаний.

\item \textit{Какой принцип работы схемы на рис.\,3.3?}

Постоянный ~ток ~от ~источника питания $\mathcal{E}$ подводится к реостату, подключённому по схеме делителя напряжения.
Напряжение, которое снимается с делителя, подаётся через переключатель $K_1$ (позиция 1) на конденсатор.
Это напряжение можно изменять, перемещая ползунок реостата.
При переключении $K_1$ в позицию 2 конденсатор замыкается на гальванометр.
Параллельно к гальванометру поключён кнопочный выключатель $K_2$, который замыкает цепь индукционных токов.
Торможение колебаний рамки осуществляется действием магнитного поля магнита гальванометра на индукционные токи.
Выключатель $K_2$ замыкаем в момент обратного прохождения светового штриха через нулевую отметку шкалы.

\end{enumerate}

\end{document}
