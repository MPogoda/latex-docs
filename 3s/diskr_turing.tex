\documentclass[titlepage,paper=a4,pagesize,draft]{scrartcl}
\usepackage{ucs}
\usepackage[utf8x]{inputenc}
\usepackage[russian]{babel}
\usepackage{amssymb}
\usepackage{indentfirst}

\author{Погода Михаил}
\title{Расчётная работа по дискретной математике, 3 семестр}
\date{\today}
\begin{document}
\maketitle
\tableofcontents
\section{Постановка задачи}
Реализовать машину Тьюринга и нормальный алгоритм Маркова, переводящие слово из унарной системы исчисления в десятичную.
\subsection{Примеры входных и выходных слов}
\begin{tabular}{cc}
Ввод & Вывод\\
$||||||$ & $6$\\
$||||||||||||||||$ & $16$\\
$|$ & $1$
\end{tabular}
\section{Машина Тьюринга}
\subsection{Краткое словесное описание алгоритма}
Алгоритм реализован по следующей схеме:
\begin{enumerate}
\item сдвинуться в конец слова, перейти к пункту 2;
\item двигаться влево от текущей позиции до тех пор, пока не увидим палочку (перейти к пункту 3) либо пустой символ (перейти к пункту 5);
\item если встретили палочку, то заменяем её на ноль и движемся в конец слова, перейти к пункту 4;
\item если в текущей позиции стоит цифра от 0 до 8, то заменяем её на одну больше и перейти к пункту 2; если встретили 9, то заменить её на 0, сдвинуться влево и повторить текущий пункт;
\item двигаться вправо до первого ненулевого символа (останов алгоритма), затирая все нули.
\end{enumerate}
\subsection{Таблица состояний}
\begin{tabular}{|c|c|c|c|c|}
\hline
 & $q_0$ & $q_1$ & $q_2$ & $q_3$\\
\hline
$|$ & $|\Rightarrow q_0$ & $0\Rightarrow q_2$ & & \\
$0$ & $\lambda\Rightarrow q_0$ & $0\Leftarrow q_1$ & $0\Rightarrow q_2$ & $1\Leftarrow q_1$\\
$1$ & $1 !$ & $1\Leftarrow q_1$ & $1\Rightarrow q_2$ & $2\Leftarrow q_1$\\
$2$ & $2 !$ & $2\Leftarrow q_1$ & $2\Rightarrow q_2$ & $3\Leftarrow q_1$\\
$3$ & $3 !$ & $3\Leftarrow q_1$ & $3\Rightarrow q_2$ & $4\Leftarrow q_1$\\
$4$ & $4 !$ & $4\Leftarrow q_1$ & $4\Rightarrow q_2$ & $5\Leftarrow q_1$\\
$5$ & $5 !$ & $5\Leftarrow q_1$ & $5\Rightarrow q_2$ & $6\Leftarrow q_1$\\
$6$ & $6 !$ & $6\Leftarrow q_1$ & $6\Rightarrow q_2$ & $7\Leftarrow q_1$\\
$7$ & $7 !$ & $7\Leftarrow q_1$ & $7\Rightarrow q_2$ & $8\Leftarrow q_1$\\
$8$ & $8 !$ & $8\Leftarrow q_1$ & $8\Rightarrow q_2$ & $9\Leftarrow q_1$\\
$9$ & $9 !$ & $9\Leftarrow q_1$ & $9\Rightarrow q_2$ & $0\Leftarrow q_3$\\
$\lambda$ & $\lambda\Leftarrow q_1$ & $\lambda\Rightarrow q_0$ & $\lambda\Leftarrow q_3$ & \\
\hline
\end{tabular}
\subsection{Примеры работы машины Тьюринга}
\begin{enumerate}
\item $\lambda\underline||||||\lambda\rightarrow(5\times q_0)\rightarrow\lambda||||||\underline\lambda\rightarrow(q_0)\rightarrow\lambda|||||\underline|\lambda\rightarrow(q_1)\rightarrow\lambda|||||0\underline\lambda\rightarrow(q_2)\rightarrow\lambda|||||\underline0\lambda\rightarrow(q_3)\rightarrow\lambda||||\underline|1\lambda\rightarrow(q_1)\rightarrow\lambda||||0\underline1\lambda\rightarrow(q_2)\rightarrow\lambda||||01\underline\lambda\rightarrow(q_2)\rightarrow\lambda||||0\underline1\lambda\rightarrow(q_3)\rightarrow\lambda||||\underline02\lambda\rightarrow(q_1)\rightarrow\lambda|||\underline|02\lambda\rightarrow(q_1)\rightarrow\lambda|||0\underline02\lambda\rightarrow(2\times q_2)\rightarrow\lambda|||002\underline\lambda\rightarrow(q_2)\rightarrow\lambda|||00\underline2\lambda\rightarrow(q_3)\rightarrow\lambda|||0\underline03\lambda\rightarrow(2\times q_1)\rightarrow\lambda||\underline|003\lambda\rightarrow(q_1)\rightarrow\lambda||0\underline003\lambda\rightarrow(3\times q_2)\rightarrow\lambda||0003\underline\lambda\rightarrow(q_2)\rightarrow\lambda||000\underline3\lambda\rightarrow(q_3)\rightarrow\lambda||00\underline04\lambda\rightarrow(3\times q_1)\rightarrow\lambda|\underline|0004\lambda\rightarrow(q_1)\rightarrow\lambda|0\underline0004\lambda\rightarrow(4\times q_2)\rightarrow\lambda|00004\underline\lambda\rightarrow(q_2)\rightarrow\lambda|0000\underline4\lambda\rightarrow(q_3)\rightarrow\lambda|000\underline05\lambda\rightarrow(4\times q_1)\rightarrow\lambda\underline|00005\lambda\rightarrow(q_2)\rightarrow\lambda0\underline00005\lambda\rightarrow(5\times q_2)\rightarrow\lambda000005\underline\lambda\rightarrow(q_2)\rightarrow\lambda00000\underline5\lambda\rightarrow(q_3)\rightarrow\lambda0000\underline06\lambda\rightarrow(5\times q_1)\rightarrow\underline\lambda000006\lambda\rightarrow(q_1)\rightarrow\lambda\underline000006\lambda\rightarrow(5\times q_0)\rightarrow\lambda\lambda\lambda\lambda\lambda\lambda\underline6\lambda\rightarrow(q_0)\rightarrow\lambda6\lambda$
\end{enumerate}

\section{Нормальный алгоритм Маркова}
\begin{enumerate}
\item $\aleph\lambda\rightarrow\xi$
\item $\aleph x\rightarrow x\aleph, ~~~~~~0\leqslant x\leqslant9$
\item $9\xi\rightarrow\xi0$
\item $x\xi\rightarrow x+1, ~~~~~0\leqslant x\leqslant8$
\item $\xi\rightarrow1$
\item $|\lambda\rightarrow1$
\item $|x\rightarrow x\aleph, ~~~~~0\leqslant x\leqslant9$
\item $\lambda\rightarrow.$
\end{enumerate}

\end{document}
