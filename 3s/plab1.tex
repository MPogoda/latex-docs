\documentclass[a4paper,10pt]{article}
\usepackage[utf8]{inputenc}
\usepackage[russian]{babel}
\usepackage{indentfirst}
\usepackage{amsmath}
\usepackage{amssymb}

\begin{document}
\begin{enumerate}
\item \textit{Что такое напряжённость и потенциал электрического поля?}

Напряжённость $E$ --- это силовая характеристика поля, которая численно равна силе,
с которой поле действует на единичный заряд (точечный положительный) $q_0$, который в находится в данной точке поля:
$$\vec{E} = \frac{\vec{F}}{q_0}$$

Потенциал --- энергитическая характеристика поля; для поля точечного заряда это величина работы,
которую выполняют силы поля перемещая единичный точечный положительный заряд из данной точки в бесконечность:
$$\varphi = \frac{A}{q_0}$$
\item \textit{Какая связь между напряжённостью и потенциалом в данной точке электростатического поля?}
$$\vec{E} = -\mathop{grad} \varphi$$
\item \textit{Какое поле называется потенциальным? Докажите потенциальный характер электрического поля}

Потенциальное поле --- векторное поле, которое можно представить как градиент некоторой скалярной функции координат (потенциала).

Электрическое поле имеет потенциальный характер т.\,к. $\mathop{rot}\vec{E} = 0$.
\item \textit{В чём заключается метод моделирования электростатических полей с помощью токов в слабопроводимой среде?}

Мы моделируем электростатическое поле заряженных тел в вакууме, т.\,к. в вакууме мы исследовать электростатическое поле не можем.
Поэтому вместо заряженных тел мы берём электроды, как слабопроводимая среда используется электропроводящая бумага,
а граничат с ней воздух и изолированная подкладка.
Преимуществом такого моделирования является то, что измерять поле в проводимой среде намного проще, чем в непроводимом;
эта простата обусловлена тем, что в проводной среде измеряется потенциал вместо напряжённости поля.
Для таких измерений используются зонды (электроды), который вводят в поле.
\item \textit{Как доказать то, что линии тока ортогональны к эквипотенциальным поверхностям?}

Эквипотенциальная поверхность --- $\varphi\left(x, y, z\right) = Const$
$$\varphi_1 - \varphi_2 = 0$$
$$\varphi_1 - \varphi_2 = \int\limits_{T_1}^{T_2}\vec{E}\,d\vec{r} = \int\limits_{T_1}^{T_2}|\vec{E}|\,|d\vec{r}|
\cos\bigl(\widehat{\vec{E},\,d\vec{r}}\bigr) = 0$$
$$|\vec{E}| \not = 0, |d\vec{r}| \not = 0 \Rightarrow \cos\bigl(\widehat{\vec{E},\,d\vec{r}}\bigr) = 0
\Rightarrow \vec{E}\bot\,d\vec{r}$$
\item \textit{Как выводят формулы для $E\left(r\right)$ и $U\left(r\right)$ в данной работе?}

Плотность тока на расстоянии $r$ от оси системы --- $j\left(r\right)$ находим из условия непрерывности:
$$j\left(r\right) = \frac{\mathcal I}{2\pi r d}$$
Напряжённость поля на расстоянии $r$:
$$E\left(r\right)=\frac{j\left(r\right)}{\sigma}, \text{ или  } E\left(r\right) = \frac{\mathcal I}{2\pi r d\sigma}$$
$$\vec{E} = -\mathop{grad}\varphi$$
Перейдём в полярную систему координат:
$$E\left(r\right) = -\frac{d\varphi\left(r\right)}{dr} = -\frac{dU\left(r\right)}{dr}$$
$$U\left(r\right) = -\int\limits_{r_0}^r E\left(r\right)\,dr = \frac{\mathcal I}{2\pi\sigma d}\ln\frac{r_0}{r}$$
, где $r_0$ --- радиус внешнего электрода.
\item \textit{Как производятся измеренияв данной работе?}

Падение ~напряжения ~$U\left(r\right)$ ~измеряется ~по ~показаниям ~амперметра ($U\left(r\right) = \mathcal{I}\left(r\right)R_g$)
на расстояниях $r = 1, 2,\dots 8$ см.
\item \textit{Как вычислить $E\left(r\right)$ по измеренным значениям $U\left(r\right)$?}

$$U_0 = U\left(r_1\right) = \frac{\mathcal I}{2\pi\sigma d}\ln\frac{r_0}{r_1}$$
$$U\left(r\right) = \frac{U_0}{\ln\frac{r_0}{r_1}}\ln\frac{r_0}{r}$$
$$E\left(r\right) = \frac{U_0}{\ln\frac{r_0}{r_1}}\frac1r$$
\end{enumerate}
\end{document}
