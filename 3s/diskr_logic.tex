\documentclass[titlepage,10pt,a4paper]{article}

\usepackage[utf8]{inputenc}
\usepackage[russian]{babel}

\usepackage{indentfirst}

\usepackage{geometry}
\geometry{left=2cm}
\geometry{bottom=2cm}
\usepackage{amsmath}
\usepackage{amssymb}
\title{Расчётная работа по дискретной математике}
\date{\today}
\author{Погода Михаил, КМ-92}
\begin{document}
\maketitle
\newpage
\tableofcontents
\newpage
\section{ПОСТАНОВКА ЗАДАЧИ}
\paragraph{Область определения: }
\textit{люди, предметы, страны.}
\paragraph{Высказывания: }
\begin{enumerate}
\item Закон гласит, что продажа оружия недружественным странам, осуществляемая любым американским гражданином, является преступлением.
\item В государстве Н., враждебным по отношению к Америке, имеются некоторые ракеты, и все ракеты этого государства были проданы ему полковником У., который является американским гражданином.
\end{enumerate}
\paragraph{Вывод: }
полковник У. совершил преступление.

\section{ФОРМАЛИЗАЦИЯ ЗАДАЧИ}
\subsection{Области определения}
\begin{itemize}
\item $x\in X$ --- люди;
\item $y\in Y$ --- предметы;
\item $z\in Z$ --- страны.
\end{itemize}
\subsection{Предикаты}
\begin{tabular}{ll}
$\mathcal{W}\left(y\right)$: &  предмет $y$ является оружие;\\
$\mathcal{E}\left(z\right)$: &  $z$ является недружественной страной по отношению к Америке;\\
$\mathcal{A}\left(x\right)$: & человек $x$ является американский гражданин;\\
$\mathcal{S}\left(x,y,z\right)$: & человек $x$ продал предмет $y$ стране $z$;\\
$\mathcal{C}\left(x\right)$: & человек $x$ совершил преступление.
\end{tabular}
\subsection{Посылки}
\begin{enumerate}
\item $\forall x\forall y\forall z \bigl( \mathcal{A} \left(x\right) \& \mathcal{W} \left(y\right) \& \mathcal{E} \left(z\right) \& \mathcal{S}\left(x,y,z\right)\rightarrow\mathcal{C}\left(x\right)\bigr)$
\item $\mathcal{E}\bigl(\text{Н.}\bigr) \& \mathcal{W}\bigl(\text{Ракеты}\bigr) \& \mathcal{S}\bigl(\text{Полковник У., Ракеты, Н.}\bigr) \& \mathcal{A}\bigl(\text{Полковник У.}\bigr)$
\end{enumerate}
\subsection{Вывод}
$\mathcal{C}\bigl(\text{Полковник У.}\bigr)$

\section{НЕФОРМАЛЬНОЕ ДОКАЗАТЕЛЬСТВО (метод редукции)}
Предположим, что все посылки истинны, а вывод ложен.
В таком случае справедлива следующая запись:
\begin{enumerate}
\item $\Bigl|\forall x\forall y\forall z \bigl( \mathcal{A} \left(x\right) \& \mathcal{W} \left(y\right) \& \mathcal{E} \left(z\right) \& \mathcal{S}\left(x,y,z\right)\rightarrow\mathcal{C}\left(x\right)\bigr)\Bigr| = T$
\item $\Bigl|\mathcal{E}\bigl(\text{Н.}\bigr) \& \mathcal{W}\bigl(\text{Ракеты}\bigr) \& \mathcal{S}\bigl(\text{Полковник У., Ракеты, Н.}\bigr) \& \mathcal{A}\bigl(\text{Полковник У.}\bigr)\Bigr| = T$
\item[В.] $\Bigl|\mathcal{C}\bigl(\text{Полковник У.}\bigr)\Bigr| = F$
\end{enumerate}
Отсюда, на интерпретации
$\begin{cases}
x = \text{Полковник У.;}\\
y = \text{Ракеты;}\\
z = \text{Н..}
\end{cases}$ первая посылка имеет вид $\bigl|T \rightarrow F\bigr|=F\not = T$.

Полученное противоречие доказывает неверность предположения, из чего следует, что вывод истинен, а не ложен.

\section{ДОКАЗАТЕЛЬСТВО ЛОГИЧЕСКОГО СЛЕДОВАНИЯ В\\ ИСЧИСЛЕНИИ ПРЕДИКАТОВ ПЕРВОГО ПОРЯДКА\\ (построение формального вывода)}
\begin{enumerate}
\item $\forall x\forall y\forall z \bigl( \mathcal{A} \left(x\right) \& \mathcal{W} \left(y\right) \& \mathcal{E} \left(z\right) \& \mathcal{S}\left(x,y,z\right)\rightarrow\mathcal{C}\left(x\right)\bigr)$\hfill Г1
\item $\mathcal{E}\bigl(\text{Н.}\bigr) \& \mathcal{W}\bigl(\text{Ракеты}\bigr) \& \mathcal{S}\bigl(\text{Полковник У., Ракеты, Н.}\bigr) \& \mathcal{A}\bigl(\text{Полковник У.}\bigr)$ \hfill Г2
\item $\forall y\forall z \big( \mathcal{A} \left(t\right) \& \mathcal{W} \left(y\right) \& \mathcal{E} \left(z\right) \& \mathcal{S} \left(t, y, z\right) \rightarrow \mathcal{C} \left(t\right)\bigr)$ \hfill УК 1
\item $\forall z \big( \mathcal{A} \left(t\right) \& \mathcal{W} \left(k\right) \& \mathcal{E} \left(z\right) \& \mathcal{S} \left(t, k, z\right) \rightarrow \mathcal{C} \left(t\right)\bigr)$ \hfill УК 3
\item $\mathcal{A} \left(t\right) \& \mathcal{W} \left(k\right) \& \mathcal{E} \left(\zeta\right) \& \mathcal{S} \left(t, k, \zeta\right) \rightarrow \mathcal{C} \left(t\right)$ \hfill УК 4
\item $\mathcal{C}\bigl(\text{Полковник У.}\bigr)$ \hfill GMP 2, 5
\end{enumerate}

\section{ДОКАЗАТЕЛЬСТВО МЕТОДОМ РЕЗОЛЮЦИЙ}
\subsection{Множество дизъюнктов для данного логического следования}
\begin{enumerate}
\item $\forall x\forall y\forall z \bigl( \mathcal{A} \left(x\right) \& \mathcal{W} \left(y\right) \& \mathcal{E} \left(z\right) \& \mathcal{S}\left(x,y,z\right)\rightarrow\mathcal{C}\left(x\right)\bigr) =
\forall x\forall y\forall z \bigl( \neg{\mathcal{A}}\left(x\right) \lor \neg{\mathcal{W}}\left(y\right) \lor \neg{\mathcal{E}}\left(z\right) \lor \neg\mathcal{S}\left(x, y, z\right) \lor \mathcal{C}\left(x\right)\bigr)$
\item $\mathcal{E}\bigl(\text{Н.}\bigr) \& \mathcal{W}\bigl(\text{Ракеты}\bigr) \& \mathcal{S}\bigl(\text{Полковник У., Ракеты, Н.}\bigr) \& \mathcal{A}\bigl(\text{Полковник У.}\bigr) = \begin{cases}
\mathcal{A}\bigl(\text{Полковник У.}\bigr)\\
\mathcal{W}\bigl(\text{Ракеты}\bigr)\\
\mathcal{E}\bigl(\text{Н.}\bigr)\\
\mathcal{S}\bigl(\text{Полковник У., Ракеты, Н.}\bigr) \end{cases}$
\item[$\neg$В.] $\neg\mathcal{C}\bigl(\text{Полковник У.}\bigr)$
\end{enumerate}
\subsection{Применение метода резолюций}
\begin{enumerate}
\item $\neg\mathcal{A}\left(x\right) \lor \neg\mathcal{W}\left(y\right) \lor \neg\mathcal{E}\left(z\right) \lor \neg\mathcal{S}\left(x, y, z\right) \lor \mathcal{C}\left(x\right)$
\item $\mathcal{A}\bigl(\text{Полковник У.}\bigr)$
\item $\mathcal{W}\bigl(\text{Ракеты}\bigr)$
\item $\mathcal{E}\bigl(\text{Н.}\bigr)$
\item $\mathcal{S}\bigl(\text{Полковник У., Ракеты, Н.}\bigr)$
\item $\neg\mathcal{C}\bigl(\text{Полковник У.}\bigr)$
\item $\neg\mathcal{W}\left(y\right) \lor \neg\mathcal{E}\left(z\right) \lor \neg\mathcal{S}\bigl(\text{Полковник У.}, y, z\bigr) \lor \mathcal{C}\bigl(\text{Полковник У.}\bigr)$ \hfill $\bigl\{\text{Полковник У.}/x\bigr\}$ \hfill Р 1,2
\item $\neg\mathcal{E}\left(z\right) \lor \neg\mathcal{S}\bigl(\text{Полковник У., Ракеты}, z\bigr) \lor \mathcal{C}\bigl(\text{Полковник У.}\bigr)$ \hfill $\bigl\{\text{Ракеты}/y\bigr\}$ \hfill Р 7,3
\item $\neg\mathcal{S}\bigl(\text{Полковник У., Ракеты, Н.}\bigr) \lor \mathcal{C}\bigl(\text{Полковник У.}\bigr)$ \hfill $\bigl\{\text{Н.}/z\bigr\}$ \hfill Р 8,4
\item $\mathcal{C}\bigl(\text{Полковник У.}\bigr)$ \hfill Р 9,5
\item $\square$ \hfill Р 10,6
\end{enumerate}

\newpage
\section{ПОСТРОЕНИЕ ЛОГИЧЕСКОЙ ПРОГРАММЫ}
\subsection{Логическая программа}
\begin{verbatim}
Predicates:
A(x)
W(y)
E(z)
S(x,y,z)
C(x)

Clauses:
A(Полковник У.)
W(Ракеты)
E(Н.)
S(Полковник У., Ракеты, Н.)
C(x) :- A(x), W(y), E(z), S(x, y, z)

Goal ?- C(Полковник У.)
\end{verbatim}
\subsection{Дерево логического вывода}
\begin{center}
\verb'not(C(Полковник У.))'\\
\verb'|'\\
\verb'C(x) :- A(x), W(y), E(z), S(x, y, z)'\\
\verb'{Полковник У./x}'\\
\verb'|'\\
\verb':- A(Полковник У.), W(y), E(z), S(Полковник У., y, z)'\\
\verb'|'\\
\verb'A(Полковник У.)'\\
\verb'|'\\
\verb':- W(y), E(z), S(Полковник У., y, z)'\\
\verb'{Ракеты/y}'\\
\verb'|'\\
\verb'W(Ракеты)'\\
\verb':- E(z), S(Полковник У., Ракеты, z)'\\
\verb'{Н./z}'\\
\verb'|'\\
\verb'E(Н.)'\\
\verb':- S(Полковник У., Ракеты, Н.)'\\
\verb'|'\\
\verb'S(Полковник У., Ракеты, Н.)'\\
\verb'|'\\
$\square$

\end{center}
\end{document}
