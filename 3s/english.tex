\documentclass[a4paper,10pt,notitlepage,pdftex,headsepline]{scrartcl}

\usepackage{geometry}
\geometry{left=2cm}
\geometry{right=2cm}
\geometry{top=1cm}
\geometry{bottom=1cm}

\usepackage{cmap} % чтобы работал поиск по PDF
\usepackage[utf8]{inputenc}
\usepackage[english,russian]{babel}
\usepackage[T2A]{fontenc}

\usepackage{textcase}
\usepackage[pdftex]{graphicx}

\pdfcompresslevel=9 % сжимать PDF
\usepackage{pdflscape} % для возможности альбомного размещения некоторых страниц
\usepackage[pdftex]{hyperref}
% настройка ссылок в оглавлении для pdf формата
\hypersetup{
	unicode=true,
    pdftitle={English Copybook},
    pdfauthor={Michael Pogoda},
    pdfcreator={pdflatex},
    pdfsubject={English},
    pdfborder    = {0 0 0},
    bookmarksopen,
    bookmarksnumbered,
    bookmarksopenlevel = 2,
    pdfkeywords={},
    colorlinks=true, % установка цвета ссылок в оглавлении
    citecolor=black, 
    filecolor=black,
    linkcolor=black,
    urlcolor=blue
}

\usepackage{amsmath}
\usepackage{amssymb}
\usepackage{moreverb}
\author{Michael Pogoda}
\title{english}
\date{\today}
\begin{document}
\section{Ex.\,7 Unit 7}
\begin{enumerate}
\item Закончив исследование, учёные провели анализ полученных данных.
\item Проектировщик ушёл из офиса после того, как просмотрел все документы.
\item Обсудив функции сберегательных устройств, мы перешли к обсуждению процессора.
\item Ограничив объём информации битом, который имеет два альтернативных состояния, разработчики компьютеров представляют информацию в виде комбинации битов.
\item После перевода программы на машинный язык, разработчик архитектуры компьютерной системы размещает программу в компьютере.
\item Закодированная инструкция была передана в процессор.
\item Переданная в процессор инструкция заставила арифметико-логическое устройство выполнить некоторые вычисления.
\end{enumerate}
\section{Ex.\,13 p.\,93}
Запоминающее устройства разделяют на основные и внешние, в зависимости от комбинации цены, объёма и времени доступа.
Цена запоминающих устройств выражается через цену одного бита сохранённых данных.
Время, необходимое компьютеру для того, чтобы найти и переместить данные, называется временем доступа.
Объём может изменятся от нескольких сотен байтов основной памяти для очень маленького компьютера до многих миллиардов байтов внешней памяти для больших компьютерных систем.
\section{Grammarway 3, Unit 10}
\subsection{Ex.\,10}
\begin{enumerate}
\item suggested;
\item begged;
\item told;
\item ordered.
\end{enumerate}
\subsection{Ex.\,11}
\begin{enumerate}
\item the ballet teacher asked Rachel to lift her leg higher.
\item The ballet teacher told her to turn her head a little more.
\item The ballet teacher told her not to lean back.
\end{enumerate}
\subsection{Ex.\,12}
\begin{enumerate}
\item The guard ordered the driver to stop.
\item He suggested going for a walk.
\item She begged him no to leave her.
\item Jenny asked Jave to help her with it.
\item She asked him to open window.
\item Mother suggested going for a drive.
\item She suggested eating then immediately.
\end{enumerate}
\subsection{Ex.\,13}
\begin{enumerate}
\item boasted;
\item agreed;
\item exclaimed;
\item complained;
\item suggested;
\item accused;
\item reminded;
\item promised;
\item insisted;
\item warned;
\item explained.
\end{enumerate}
\subsection{Ex.\,26}
\begin{enumerate}
\item The tyable was covered with a cloth.
\item Who was the book written by?
\item It is said that she is very clever.
\item The car was sold by Martin.
\item Cathy wants to be liked.
\item It is expected that the letter will arrive soon.
\item Were those decorations made by Andy?
\item It is said that a ring was given to Julie by Rick.
\end{enumerate}
\subsection{Ex.\,27}
\begin{enumerate}
\item has;
\item was;
\item not;
\item its;
\item it;
\item to;
\item being;
\item of.
\end{enumerate}
\section{Ex.\,6 p.\,101}
\begin{enumerate}
\item Информационная ёмкость бита ограничена двумя альтернативами, на комбинации которых базируется код.
\item Основная память функционально похожа на человеческий мозг.
\item После того, как электрон покинет поверхность, металл становится позитивно заряженным.
\item Все инструкции должны проходить через основную память, вокруг которой организована архитектура компьютерной системы.
\item Архитектура компьютерной системы организована вокруг основной памяти, все инструкции проходят через неё.
\item Электромеханическая память зависит от подвижных механических частей, время доступа которых больше, чем в электронной памяти.
\end{enumerate}
\section{Ex.\,13 p.\,106}
\begin{enumerate}
\item Результаты арифметических операций, которые вернулись в аккумулятор, были переданы в основную память.

Результаты арифметических операций, которые регистр памяти передаёт в основную память, были возвращены аккумулятору.

Результаты арифметических операций, которые возвращаются аккумулятору, регистр памяти переносит в основную память.
\item Проходя через кондуктор, свободные электроны образуют электрический ток.

Свободные элементы, которые проходят через кондуктор, генерируют электрический ток.

Свободные элементы проходят через кондуктор, генерирую при этом электрический ток.
\end{enumerate}
\section{Unit 9, Ex.\,9}
\begin{enumerate}
\item Принтер --- пример устройства, которое выводит информацию в понятному человеку формате.
\item Скоростные устройства, которые используются как внешние запоминающие устройства, являются устройствами вводу и вывода.
\item Прогресс электроники, результатом которого стало изобретение электронного компьютера, был научным прорывом второй половины двадцатого века.
\item Периодический закон Менделеева, который был принят как универсальный закон природы, имеет большое значение в наше время.
\item Память сохраняет инструкции и данные, чтобы быстро их восстанавливать по запросу процессора.
\end{enumerate}
\section{Ex.\,20 p.\,126}
Devices, which can perform both the input and output functions, such as magnetic discs, diskettes \& magnetic tapes, are called magnetic media devices.

Data are recorded on magnetic discs and magnetic tapes either by outputing the data from primary storage or by using a data recorder.

Ket-to-disc devices are used as data recording stations in multistation shared-processor systems.

Key-to-diskette systems store data on flexible discs, called diskettes.
\section{Unit 9 Ex.\,19}
\begin{enumerate}
\item Принтеры, как известно, сильно различаются по продуктивности и дизайну.
\item Ожидается, что они станут наиболее часто используемыми устройствами.
\item Струйный принтер считается одним из самых новых типов посимвольно-печатающих устройств.
\item Принтер ударного действия печатает символы ударяя символьным шрифтом по бумаге.
\item Качество печати матричных принтеров хуже.
\item Наиболее часто в больших системах используется строчный принтер.
\end{enumerate}
\section{Unit 7, Ex.\,1}
\begin{enumerate}
\item b
\item a
\item c
\item a
\end{enumerate}
\section{Unit 7, Ex.\,2}
\begin{enumerate}
\item j
\item h
\item d
\item c
\item b
\end{enumerate}
\section{Unit 8, Ex.\,1}
\begin{enumerate}
\item a
\item c
\item c
\item b
\end{enumerate}
\section{Unit 8, Ex.\,2}
\begin{enumerate}
\item a
\item d
\item c
\item i
\item f
\end{enumerate}
\section{Unit 10, Ex.\,3}
\begin{enumerate}
\item c
\item b
\item c
\end{enumerate}
\section{Unit 7, Ex.\,14}
\begin{enumerate}
\item A control unit
\item The algorithm
\item The program
\item The programmed-control principle.
\end{enumerate}
\section{p.\,140, Ex.\,16}
A modem is a device that allows a computer to communicate with other computers.
It allows the individual to access information from all over the world and use that information in everyday life.

A modem takes computer information and changes it into a signal that can be sent over telephone lines.
The modem is a bridge between digital and analog signals.

Modem and telephone can share one line.

There are three kinds of modems: internal, external and fax.

Most computer modems are modems with faxing capabilities.
\section{Unit 10, Ex.\, 11}
rate-esteem, analyse-developer, use-application, possibility-probability, plays-games, control-restraint, post-mail, protession-job, consultant-adviser, teacher-tutor, director-supervisor, bookkeeper-accountant, fight-combat, amateur-dilettante, attack-offence

flexible-supple, thrilling-exciting, main-chief, little-small, general-universal.
\section{p.\, 141, Ex.\, 2}
ЭЛТ --- экран, который используется в персональных компьютерах (монитор).
Клавиатура и монитор могут быть частями одного устройства, который также содержит микрокомпьютер и дисковые устройства, или могут быть отдельными элементами.

Кроме монитора, наиболее распространёнными устройствами вывода являются матричные и посимвольные принтеры.
Матричные принтеры подходят для большинства применений.
Посимвольные принтеры обычно используются для деловой переписки.

Флоппи-диски, или дискеты, --- это самое распространённое запоминающее устройство.
Вони хранят наборы битом на гибком магнитном диске.
\section{p.\,144, Ex.\,2} 
\begin{enumerate}
\item typing, inputting
\item operates, enables, provides
\item held
\item being supplied
\item have been worked
\item having been constructed
\end{enumerate}
\section{p.\,153, Ex.\,12}
\begin{enumerate}
\item Было б хорошо, если бы ты не курил.
\item Если бы Земля не вращалась бы, она не имела бы форму шара.
\item Если бы у меня было время, я бы помог тебе решить проблему.
\item Я бы перевёл статью без затруднений, если бы я хорошо знал английский язык.
\item Если бы я был на твоём месте, я бы научился разговаривать бегло на английском.
\item Если бы тебя попросили объяснить почему сложение выполняется так, как оно выполняется, тебе, наверно, необходимо было бы подумать , прежде чем отвечать
\end{enumerate}
\end{document}